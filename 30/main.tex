\documentclass[a4paper,11pt]{article}
\makeatletter
\usepackage[top=2cm, left=1.5cm, bottom=1.5cm, right=1.5cm]{geometry}
\usepackage{amsmath,amssymb}
\usepackage{tcolorbox}
\usepackage[export]{adjustbox}
\usepackage{graphicx}
\usepackage{wrapfig}
\usepackage{pgf}
\usepackage{tikz}% graphic drawing
\usetikzlibrary{arrows}
\pagestyle{empty}
\usepackage{wasysym}
\usepackage{mathpazo}% change math font
\usepackage{enumitem}% change list environment like enumerate, itemize and description
\usepackage{multicol}% multi columns
\usepackage{mathpazo}% change math font
\usepackage{xcolor}
\usepackage[no-math]{fontspec}% font specfication
\newcommand{\heart}{\ensuremath\heartsuit}
\newcommand{\butt}{\rotatebox[origin=c]{180}{\heart}}
\newcommand*\circled[1]{\tikz[baseline=(char.base)]{
\node[shape=circle,draw,inner sep=2pt] (char) {#1};}}
%
\makeatletter %line1-8 make Khmer known
\def\@khmernum#1{\expandafter\@@khmernum\number#1\@nil} 
\def\@@khmernum#1{%
\ifx#1\@nil 
\else 
\char\numexpr#1+"17E0\relax 
\expandafter\@@khmernum\fi 
} 
\def\knum#1{\expandafter\@khmernum\csname c@#1\endcsname}
\def\khmernumeral#1{\@@khmernum#1\@nil}
\AddEnumerateCounter{\knum}{\@knum}{}
\makeatother
\makeatletter
\newcommand*{\kalph}[1]{%
\expandafter\@kalph\csname c@#1\endcsname%
}
\newcommand*{\@kalph}[1]{%
\ifcase#1\or ក\or ខ\or គ\or ឃ\or ង\or ច\or ឆ\or ជ\or ឈ\or ញ\or ដ\or ឋ\or ឌ\or ឍ\or ណ\or ត\or ថ\or ទ\or ធ\or ន\or ប\or ផ\or ព\or ភ\or ម\or យ\or រ\or ល\or វ\or ស\or ហ\or ឡ\or អ%
\else\@ctrerr\fi%
}
\AddEnumerateCounter{\kalph}{\@kalph}{}
\makeatother

\SetEnumitemKey{I}{%
leftmargin=*,
label={\protect\tikz[baseline=-0.9ex]\protect\node[text=blue,fill=blue!10!white]{\en\Roman*.};},%
font=\small\sffamily\bfseries,%
labelsep=1ex,%
topsep=0pt}
%
\SetEnumitemKey{a}{%
leftmargin=*,%
label={\protect\tikz[baseline=-0.9ex]\protect\node[text=blue,fill=magenta!10!white]{\kalph*.};},%
font=\small\kbk\bfseries,%
labelsep=1ex,%
topsep=0pt}
%
\SetEnumitemKey{1}{leftmargin=*,%
label={\protect\tikz[baseline=-0.9ex]\protect\node[text=red,fill=cyan!20!white]{\knum*.};},%
font=\small\kbk\bfseries,%
labelsep=1ex,%
topsep=0pt}
%
\def\hard{\leavevmode\makebox[0pt][r]{\large\ensuremath{\star}\hspace{2em}}}
%
\def\hhard{\leavevmode\makebox[0pt][r]{\large\ensuremath{\star\star}\hspace{2em}}}
%
\everymath{\protect\displaystyle\protect\color{black}}
%
\pagecolor{cyan!1!white}
%
\usepackage{amsmath}
\usepackage{amssymb}
\usepackage{wasysym}
\makeatother
\setmainfont{Khmer OS Content}% set default font to Khmer OS
\newcommand{\ko}{\fontspec[Script=Khmer]{Khmer OS}\selectfont}
\newcommand{\kml}{\fontspec[Script=Khmer]{Khmer OS Muol Light}\selectfont}
\newcommand{\kos}{\fontspec[Script=Khmer]{Khmer OS System}\selectfont}
\newcommand{\kb}{\fontspec[Script=Khmer]{Khmer OS Battambang}\selectfont}
\newcommand{\kbk}{\fontspec[Script=Khmer]{Khmer OS Bokor}\selectfont}
\newcommand{\en}{\fontspec{Liberation Serif}\selectfont}

\usepackage{fancyhdr}
\pagestyle{fancy}
\fancyhf{}
\lhead{\kbk វិទ្យាល័យមេតូឌីស្តកម្ពុជា}
\chead{\kbk គណិតវិទ្យាទី១២(វិទ្យាសាស្ត្រ)}
\rhead{\kbk វិញ្ញាសាត្រៀមប្រឡង}
\lfoot{\kbk ដោយ៖ ស៊ំុ សំអុន}
\cfoot{\en\thepage}
\rfoot{\en Tel: 096 94 05 840}
\renewcommand{\headrulewidth}{0.7pt}
\renewcommand{\footrulewidth}{0.7pt}
\headheight = 25pt
\headsep = 5pt
\footskip = 25pt

\begin{document}
\begin{center}

{\kml វិញ្ញាសាទី១}
\end{center}
\vspace{-0.3cm}
\begin{enumerate}[I]
\item \begin{enumerate}[1]
\item ដោះស្រាយសមីការ $Z^2-2\sqrt{2} Z+4=0 \quad(1)$ ក្នុងសំណុំចំនួនកុំផ្លិច។ រកម៉ូឌុល និងអាគុយម៉ង់នៃឬសនីមួយៗ របស់សមីការ $(1)$ ។
\item សរសេរ $W=\left(\frac{\sqrt{2}+i\sqrt{2}}{\sqrt{2}-i\sqrt{2}}\right)^2$ ជាទម្រង់ត្រីកោណមាត្រ។
\end{enumerate}
\item ចូរគណនាលីមីតខាងក្រោម៖
\begin{multicols}{3}
\begin{enumerate}[a]
\item $A=\lim_{x\to 0}{\frac{\sin x+\sin 2x}{\sin 3x+\sin 4x}}$
\item $B=\lim_{x\to 0}{\frac{e^x-\sin x-1}{1-\sqrt{x+1}}}$
\item $C=\lim_{x\to0}{\frac{(2e^x-2)(1-\cos 2x)}{x^3}}$
\end{enumerate}
\end{multicols}
\item \begin{enumerate}[1]
\item ចតុកោណកែងមួយមានបរិមាត្រ $400 ~ m^2$ ។ រកប្រវែងជ្រុងដើម្បីឱ្យចតុកោណនេះមានផ្ទៃក្រឡាធំបំផុត។
\item ចតុកោណកែងមួយមានផ្ទៃក្រឡា $1600 ~ m^2$ ។ រកប្រវែងជ្រុងដើម្បីឱ្យចតុកោណកែងនេះមានបរិមាត្រតូចបំផុត។
\end{enumerate}
\item ក្នុងតម្រុយអរតូណម៉ាល់មានទិសដៅវិជ្ជមាន $(o,\vec{i}, \vec{j} ,\vec{k})$ មួយគេឱ្យចំណុច $A(1,-2,0) ~;~ B(1,0,4)$ និង $C(0,3,3)$ ។
\begin{enumerate}[1]
\item ចូរសង់ត្រីកោណ $ABC$ ក្នុងតម្រុយ $(o,\vec{i}, \vec{j} ,\vec{k})$ ។
\item រកកូអរដោនេនៃវ៉ិចទ័រ $\overrightarrow{AB}$ រួចរកសមីការប្លង់ $(P)$ ជាប្លង់មេដ្យាទ័រនៃ $[AB]$ ។
\item រកសមីការស្វ៊ែ $(S)$ ដែលមានអង្កត់ផ្ចិត $[AB]$ ។
\item គណនា $\vec{n}=\overrightarrow{AB}\times \overrightarrow{AC}$ ។ ទាញរកផ្ទៃក្រឡានៃត្រីកោណ $ABC$ ។
\item គណនា $t=(\overrightarrow{AB}\times \overrightarrow{AC}).\overrightarrow{AO}$ ។ ទាញរកមាឌនៃតេត្រាអ៊ែត $OABC$ ។ ទាញរកចម្ងាយពី $O$ ទៅប្លង់ $ABC$ ។
\end{enumerate}
\item អនុគមន៍ $f$ កំណត់ដោយ $y=f(x)=\frac{x^2-x+1}{x-1}$ និងមានក្រាប $(C)$ ។
\begin{enumerate}[1]
\item រកដែនកំណត់នៃអនុគមន៍ $f$ ហើយគណនា$\lim_{x\to\pm \infty }f(x)$ និង $\lim_{x\to 1}f(x)$ ។ 
\item កំណត់តម្លៃ $a,b$ និង $c$ ដើម្បីឱ្យ $f(x)=ax+b+\frac{c}{x-1}$ ។
\item បង្ហាញថា $f$ មានអតិបរមាមួយ និងអប្បបរមាមួយ។ គណនាតម្លៃនៃបរមាទាំងពីរ។ សង់តារាងអថេរភាពនៃ $f$ ។
\item រកសមីការអាស៊ីមតូតទាំងពីរនៃក្រាប $(C)$ ។ សិក្សាទីតាំងរវាងក្រាប $(C)$ ធៀបនឹងអាស៊ីមតូតទ្រេត។
\item សង់ក្រាប $(C)$ ក្នុងតម្រុយអរតូណម៉ាល់ $(o,\vec{i},\vec{j})$ ។
\item ដោយប្រើក្រាភិច ចូរពិភាក្សាទៅតាមតម្លៃនៃប៉ារ៉ាម៉ែត្រ $m$ នូវអត្ថិភាព និងសញ្ញាឬសនៃសមីការ $x^2-(m+1)x+m+1=0$
\end{enumerate}

\item $f$ ជាអនុគមន៍កំណត់ដោយ $f(x)=x-1-2\ln\left(1-\frac{1}{x}\right)$ និង $(C)$ ជាក្រាបនៃ $f$ ។
\begin{enumerate}[1]
\item ចូរសិក្សាទិសដៅអថេរភាពនៃ $f$ ។
\item គណនា $\lim_{x\to 1^+}{f(x)}$ និង $\lim_{x\to +\infty}{f(x)}$ រួចរកអាស៊ីមតូតនៃក្រាប $(C)$។
\item គូសតារាងអថេរភាពនៃ $f$ ។ រួចសង់ក្រាប $(C)$ ក្នុងតម្រុយ $(o,\vec{i},\vec{j})$ ។
\item បង្ហាញថាសមីការ $f(x)=3$ មានឬសតែមួយគត់លើចន្លោះ $[2,+\infty)$ ។
\item គណនាក្រឡាផ្ទៃ $A$ ដែលនៅចន្លោះខ្សែកោង $(C)$ និងបន្ទាត់ $L: y=x-1$ ត្រូវនឹងចន្លោះ $2\leq x \leq 4$ ។ គេឱ្យ $\ln 2=0.7$ និង $\ln 3=1.1$ ។
\end{enumerate}
\end{enumerate}
\end{document}