\documentclass[12pt, a5paper]{article}
%%import package named hightest
\usepackage{hightest}
\usepackage{amsmath}
\usepackage{cases}
\usepackage[export]{adjustbox}
\usepackage{wrapfig}
\usepackage{tkz-tab}
%\usepackage{mathpazo}% change math font
%\usepackage[no-math]{fontspec}% font specfication
\header{រៀនគណិតវិទ្យាទាំងអស់គ្នា}{គណិតវិទ្យា}{២៧/០៣/២០១៨}
\footer{រៀបរៀង និងបង្រៀនដោយ ស៊ុំ សំអុន}{ទំព័រ \thepage}{០៩៦ ៩៤០ ៥៨៤០}
\everymath{\protect\displaystyle\protect\color{magenta}}
\begin{document}
\maketitle
\begin{enumerate}[m]
	\item គេមានអនុគមន៍ $f(x)=2e^x-e^{2x}$ កំណត់ចំពោះគ្រប់ $x\in \mathbb{R}$ ។
	\begin{enumerate}[k]
		\item គណនាលីមីត $\lim\limits_{x\to -\infty}f(x)$ និង $\lim\limits_{x\to +\infty} f(x)$ 
		\item គណនាដេរីវេ និងរកបរមាធៀបនៃអនុគមន៍ $f$
		\item គណនាដេរីវេទីពីរ និងរកចំនុចរបត់នៃអនុគមន៍ $f$
		\item សង់តារាងអថេរភាព និងក្រាបនៃអនុគមន៍ $f$ ។\\ (គេឲ្យ៖ $e=2.7~~e^2=7.4~~\ln2=0.7$) 
	\end{enumerate}
	\item គេមានអនុគមន៍ $f(x)=e^x-x$ កំណត់ចំពោះគ្រប់ $x\in \mathbb{R}$ ។
	\begin{enumerate}[k]
		\item គណនាលីមីត $\lim\limits_{x\to -\infty}f(x)$ និង $\lim\limits_{x\to +\infty} f(x)$ 
		\item រកអាស៊ីមតូតទ្រេតនៃក្រាបតាងអនុគមន៍ $f$
		\item គណនា និងសិក្សាសញ្ញានៃដេរីវេ និងរកបរមាធៀបនៃអនុគមន៍ $f$
		\item គណនា $f(1)$ និង $f'(-1)$
		\item សង់តារាងអថេរភាព និងក្រាបនៃអនុគមន៍ $f$ ។
		(គេឲ្យ៖ $e=2.7~~e^2=7.4$ )
	\end{enumerate}
	\item $f$ ជាអនុគមន៍កំណត់ដោយ $f(x)=e^{1-x}$ ហើយ $C$ ជាក្រាបនៃ $f$ ។
	\begin{enumerate}[k]
		\item បញ្ជាក់ដែនកំណត់នៃ $f$ រួចស្រាយបំភ្លឺជាអនុគមន៍ចុះជានិច្ចលើ $\mathbb{R}$
		\item គណនា $\lim\limits_{x\to -\infty}f(x)$ និង $\lim\limits_{x\to +\infty} f(x)$ រួចទាញរកអាស៊ីមតូតដេកនៃក្រាប $C$
		\item ចូរគូសតារាងអថេរភាពនៃ $f$
		\item កំណត់កូអរដោនេនៃចំណុចប្រសព្វរវាងក្រាប $C$ និងអក្ស័អរដោនេ។
		\item សរសេរមីការបន្ទាត់ $T$ ដែលប៉ះក្រាប $C$ ត្រង់ចំណុចអាស៊ីស $x=1$។
		\item សង់ក្រាប $C$ និងបន្ទាត់ប៉ះ $T$ នៅក្នុងតម្រុយអរតូនរមេតែមួយ។
	\end{enumerate}
	\item គេឲ្យអនុគមន៍ $f$ កំណត់ដោយ $y=f(x)=\frac{2}{e^x+1}-1$ មានក្រាបតាង $C$។ 
	\begin{enumerate}[k]
		\item បញ្ជាក់ដែនកំណត់នៃអនុគមន៍ $f$
		\item គណនាលីមីតនៃ$f$ត្រង់ចុងដែនកំណត់ រួចទាញរកសមីការអាស៊ីមតូតនៃក្រាប$C$។
		\item គណនា $f(-x)+f(x)$ រួចទាញថា $f$ ជាអនុគមន៍សេស។
		\item គណនា និងសិក្សាសញ្ញានៃដេរីវេ $f'(x)$ រួចសង់តារាងអថេរភាពនៃ $f$
		\item សរសេរសមីការបន្ទាត់ប៉ះនិងក្រាប $C$ ត្រង់ចំណុចដែលមានអាប់ស៊ីស $x=0$
		\item សង់ក្រាប $C$ និងបន្ទាត់ប៉ះ ក្នុងតម្រុយអរតូណរម៉ាល់ $(O; \vec{i}; \vec{j})$
	\end{enumerate}
	\item គេឲ្យអនុគមន៍ $f(x)=\frac{e^x}{1-e^{2x}}$ ដែលមានក្រាបតាង $C$។
	\begin{enumerate}[k]
		\item រកដែនកំណត់នៃអនុគមន៍ $f$
		\item គណនាលីមីតត្រង់ចុងដែនកំណត់ រួចទាញរកអាស៊ីមតូតឈរ និងដេកនៃ $f$
		\item បង្ហាញថាចំណុច $O(0, 0)$ ជាផ្ចិតឆ្លុះនៃក្រាបតាងអនុគមន៍ $f$
		\item បង្ហាញថាអនុគមន៍ $f$ ជាអនុគមន៍កើនដាច់ខាត
		\item សង់តារាងអថេរភាព និងសង់ក្រាបនៃអនុគមន៍ $f$
		\item សិក្សាអថេរភាព និងសិក្សាសញ្ញាប្ញសនៃសមីការ $me^{2x}+e^x=m$ តាមក្រាប។
	\end{enumerate}
	\begin{center}
		\sffamily\color{black}
		សូមសំណាងល្អ!
	\end{center}
\end{enumerate}
\end{document}