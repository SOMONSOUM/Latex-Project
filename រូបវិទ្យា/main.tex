\documentclass[12pt, a4paper]{article}
%%import package named techno
\usepackage{techno}
\usepackage[export]{adjustbox}
\usepackage{wrapfig}
\usepackage{tkz-tab}
%សរសេរគីមីវិទ្យា
%\usepackage[version=3]{mhchem}
%\usepackage{mathpazo}% change math font
%\usepackage[no-math]{fontspec}% font specfication
\everymath{\protect\displaystyle\protect\color{blue}}
\begin{document}
\maketitle\koc
	\begin{enumerate}[m]
		\item គណនាបម្រែបម្រួលថាមពលក្នុងនៃឧស្ម័នបរិសុទ្ធ $3~mol$ នៅពេលវាដុតកម្តៅពី $273~K$ ទៅ $293~K$ ។
		\item កម្មន្តសរុបធ្វើលើឧស្ម័នស្មើ $135~J$ ។ ឧបមាក្នុងរយៈពេលបម្លែង ថាមពលក្នុងវាកើន $104~J$ ។ តើបរិមាណកម្តៅមានតម្លៃប៉ុន្មាន?
		\item ប្រព័ន្ធទែម៉ូឌីណាមិចកម្មន្តដែល
	\end{enumerate}
\end{document}