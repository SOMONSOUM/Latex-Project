\documentclass[12pt, a5paper]{article}
%%import package named hightest
\usepackage{hightest}
\usepackage[export]{adjustbox}
\usepackage{wrapfig}
\usepackage{tkz-tab}
\usepackage[version=3]{mhchem}
%\usepackage{mathpazo}% change math font
%\usepackage[no-math]{fontspec}% font specfication
\header{រៀនគណិតវិទ្យាទាំងអស់គ្នា}{គីមីវិទ្យា}{០៣/០៤/២០១៨}
\footer{រៀបរៀង និងបង្រៀនដោយ ស៊ុំ សំអុន}{ទំព័រ \thepage}{០៩៦ ៩៤០ ៥៨៤០}
\everymath{\protect\displaystyle\protect\color{blue}}
\begin{document}
\maketitle
\begin{enumerate}[m]
	\item តើអ្វីទៅដែលហៅថាលំនឹងគីមី?
	\item តើថេរលំនឹង $\ce{K}$ សម្គាល់ទៅលើអ្វី?
	\item តើកត្តាអ្វីដែលធ្វើឲ្យថេរលំនឹង $\ce{K}$ ប្រែប្រួល?
	\item សរសេរកន្សោមថេរលំនឹងគីមីនៃប្រតិកម្មខាងក្រោម៖
	\begin{enumerate}[k]
		\item $ \ce{4HCl_{(g)} + O2_{(g)} <=> 2Cl2_{(g)} + 2H2O_{(g)}} $
		\item 
		$ \ce{2HI_{(g)} <=> H2_{(g)} + I2_{(g)}} $
	\end{enumerate}
	\item អាស៊ីតក្លរីឌ្រិចជាអាស៊ីតខ្លាំង នៅក្នុងទឹកវាបំបែកជាអ៊ីយ៉ុង $\ce{H3O+}$ និង $\ce{Cl-}$ បានសព្វល្អៈ~~
	$\ce{HCl_{(aq)} + H2O_{(l)} <=> H3O^+_{(aq)} + Cl^-_{(aq)}} $ ។\\
	ចូរអ្នកសាកល្បងពិចារណាតម្លៃថេរលំនឹង $\ce{K}$ នៃប្រតិកម្មខាងលើ តើតម្លៃរបស់ $\ce{K}$ អាចត្រូវនឹងតម្លៃណាមួយៈ  $1\times10^{-2}~;~1\times10^{-3}~;~1\times10^{-5}$ ឬធំណាស់ ?
	\item គេមានប្រតិម្ម $\ce{H2_{(g)} + Cl2_{(g)} <=> 2HCl_{(g)}}$ ដែលមានលំនឹងនៅសីតុណ្ហភាព $1227^\circ C$ ។ កំហាប់អង្គធាតុនៅពេលលំនឹងគីមីគឺ៖ $\ce{[H2]=[Cl2]}=4.5\times10^{-3}\ce{M}$ និង $\ce{[HCl]}=62.5\times10^{-3}\ce{M}$ ។\\
	ចូរគណនាថេរលំនឹង $\ce{K}$ ? \text{\sffamily\color{magenta}ចម្លើយ $\ce{K=192.9}$}
	\item គេមានប្រតិម្ម $\ce{H2_{(g)} + I2_{(g)} <=> 2HI_{(g)}}$ ដែលមានលំនឹងនៅសីតុណ្ហភាព $425^\circ C$ ។ កំហាប់អង្គធាតុនៅពេលលំនឹងគីមីគឺ៖ $\ce{[H2]}=1.83\times10^{-1}\ce{M}\\\ce{[I2]}=3.13\times10^{-3}\ce{M}$ និង $\ce{[HI]}=1.77\times10^{-2}\ce{M}$ ។\\
	ចូរគណនាថេរលំនឹង $\ce{K}$ ? \text{\sffamily\color{magenta}ចម្លើយ $\ce{K=0.54}$}
	\begin{center}
		\sffamily\color{black}
		សូមសំណាងល្អ!
	\end{center}\newpage
	\begin{center}
		\sffamily\color{black}
		\circled{០២}\\
		ជំពូក ៤ លំនឹងគីមី(លំហាត់សុទ្ធ)
	\end{center} 
	\begin{center}
		\sffamily\color{black}
		សូមសំណាងល្អ!
	\end{center}\newpage
\end{enumerate}
\end{document}