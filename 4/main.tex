\documentclass[12pt ,a4paper]{book}

\usepackage[ top=1cm ,bottom =1cm ,right=0.8cm ,left=0.8cm]{geometry}

\usepackage[T1]{fontenc}

\usepackage[utf8]{inputenc}

\usepackage{ucs}

\usepackage{amssymb}


\usepackage{xltxtra}

\everymath{\displaystyle}

\usepackage{mathptmx}  

\usepackage{xcolor}

\usepackage{graphicx}

\usepackage{tikz}

\usepackage{enumerate}

\usepackage{enumitem}

\usepackage{tocloft}

\usepackage{multicol}

\setlength{\columnsep}{1.5cm}

\usepackage{caption}

%\captionsetup[figure]{name= រូបទី}

\usepackage{wrapfig}




\usepackage[sumlimits]{amsmath}

\usepackage{enumitem}

\usepackage{tocloft}

\usepackage{sectsty}

\makeatletter

\def\@khmernum#1{\expandafter\@@khmernum\number#1\@nil}

\def\@@khmernum#1{%

\ifx#1\@nil

\else

\char\numexpr#1+”17E0\relax

\expandafter\@@khmernum\fi

}

\def\knum#1{\expandafter\@khmernum\csname c@#1\endcsname}

\def\khmernumeral#1{\@@khmernum#1\@nil}

\AddEnumerateCounter{\knum}{\@knum}{}

\makeatother

\makeatletter

\newcommand*{\kalph}[1]{%

\expandafter\@kalph\csname c@#1\endcsname%

}

\newcommand*{\@kalph}[1]{%

\ifcase#1\or ក\or ខ\or គ\or ឃ\or ង\or ច\or ឆ\or ជ\or ឈ\or ញ\or ដ\or ឋ\or ឌ%

\or ឍ\or ណ\or ត\or ថ\or ទ\or ធ\or ន\or ប\or ផ\or ព\or ភ\or ម\or យ\or រ\or ល%

\or វ\or ស\or ហ\or ឡ\or អ%

\else\@ctrerr\fi%

}

\AddEnumerateCounter{\kalph}{\@kalph}{}

\makeatother





\defaultfontfeatures{Mapping=tex-text}

\XeTeXlinebreaklocale"khm"

\XeTeXlinebreakskip=0pt plus 1pt minus 1pt

\setmainfont[ Scale=0.9 , Script=Khmer, AutoFakeBold=3.5,AutoFakeSlant=0.4]{Kh Preyveng}\setmathrm{Ubuntu}

\setmathrm{Ubuntu}

\newcommand{\en}{\fontspec{Ubuntu}\selectfont}

\newcommand{\kos}{\fontspec[Scale=0.9, Script=Khmer]{Khmer OS Content}\selectfont}

\newcommand{\kml}{\fontspec[Scale=1, Script=Khmer]{Khmer OS Muol Light}\selectfont}

\newcommand{\kl}{\fontspec[Scale=1.3, Script=Khmer]{Koulen}\selectfont}




\newcommand{\f}{\frac}

\newcommand{\df}{\dfrac}

\newcommand{\s}{\sqrt}

\newcommand{\C}{\mathbb{C}}

\newcommand{\N}{\mathbb{N}}

\newcommand{\Z}{\mathbb{Z}}

\newcommand{\R}{\mathbb{R}}

\renewcommand{\(}{\left(}

\renewcommand{\)}{\right)}

\newcommand{\lb}{\left|}

%\newcommand{\rb}{\right|}

\renewcommand{\[}{\left[}

\renewcommand{\]}{\right]}

\newcommand{\rb}{\right|}


\renewcommand{\labelenumi}{\textcolor{blue}{\Roman*.}}

\renewcommand{\labelenumii}{\textcolor{blue}{\kalph*.}}

\renewcommand{\labelenumiii}{\textcolor{blue}{\alph*.}}

\renewcommand{\chaptername}{ មេរៀន }


%\usepackage{fancyhdr}

%\pagestyle{fancy}

%\fancyhf{}

%\lhead{ រៀបរៀងដោយ xxxxxxxxxx }

%\chead{ Tel : xxxxxxxx}

%\rhead{ សម្រាប់ថ្នាក់ទី  xxxxxxx}


\begin{document}

\begin{center}

{\kl វិញ្ញសារ ទី​ ០៤}\\  {\kl សម្រាប់ត្រៀមប្រលងឆមាសលើកទី​១}\\ { \kl រៀបរៀង  និង បង្រៀនដោយ \; យឹម  ភារុន}\\

\end{center} 

\begin{enumerate}

\item  គណនាលីមីត $M=\lim_{x\to -\infty}\ln\[ \df {(e^{3x}-e)(1-ex)  }{(e^x+1)(x+1)}\]$ ~និង ~$N=\lim_{x\to 0}\df{e^{2x^2}-\cos4x}{5x^2}$ ។

\item  គេមានអនុគមន៍ $f(x)=\df{1-\sin x}{1+\sin x}$  ។\\

ចូរគណនា $f'(x)$ ~រួចគណនាតម្លៃនៃ $f'\( 2014\pi\) , f'\( 2015\pi\)$ និង $f'\( \df{2015\pi}{2}\)$~។ 

\item  

\begin{enumerate}

\item  ដោយដឹងថា $i^2=-1$  ចូរគណនា $i^{2014}$ និង $i^{2015}$  រួចសរសេរចំនួនកុំផ្លិច $z=i^{2014}-i^{2015}$  ជាទម្រង់ត្រីកោណមាត្រ។

\item  កំណត់តម្លៃ $a$  និង  $b$  ដោយដឹងថា $az+b\bar{z}=|z|^2$ ។

\item  បង្ហាញថា $A=\df{i^{2014}-i^{2015}}{i^{2014}+i^{2015}}$  ជាចំនួននិមិត្តសុទ្ធ ។

\end{enumerate}

\item  គេឲ្យអនុគមន៍មួយកំណត់ដោយ $g(x)=\df{x^2-x-1}{x+1}~, x\ne -1$ ។

\begin{enumerate}

\item  រកចំនួនពិត $a,b$ និង $c$ ដើម្បីឲ្យ $g(x)=ax+b+\df{c}{x+1}$  ចំពោះគ្រប់ $x\ne -1$ ។\\ រួចរកសមីការអាស៊ីមតូតទ្រេតនៃក្រាបតាងអនុគមន៍ $g$  ។

\item  គណនាដេរីវេ $g'(x), g''(x)$ និង$g'''(x)$ ។

\item  បង្ហាញថាចំណុច $I(-1,-3)$  ជាផ្ចិតឆ្លុះនៃក្រាបតាងអនុគមន៍នេះ ។

\end{enumerate}

\item  ក្នុងតម្រុយអតូណរម៉ាល់ $\(O, \vec i,\vec j , \vec k\)$  គេមានចំណុច $A(0,1,0),B(3,0,0)$ និង $C(0,0,2)$  ។

\begin{enumerate}

\item    គណនា $\vec n=\overrightarrow{AB}\times \overrightarrow{AC}$   រួចទាញថាបីចំណុច $A,B$ និង $C$ មិនស្ថិតនៅលើបន្ទាត់តែមួយ ។

\item  បង្ហាញថា ប្លង់ $(ABC)$  កំណត់ដោយសមីការ ~$\df x3+\df y1+\df 2z=1$ ។

\item  សរសេរសមីការប៉ារ៉ាម៉ែត្រនៃបន្ទាត់  $(BC)$~។ រួចគណនាចម្ងាយពីចំណុច 	$A$ ទៅបន្្ទាត់ $(BC)$  ។

\item  គណនាផ្ទៃក្រឡាត្រីកោណ $ABC$  និង មាឌចតុមុខ $OABC$  ។ រួចទាញរកចម្ងាយពីគល់  $O$  ទៅប្លង់ $(ABC)$ ។

\end{enumerate}

\item  អនុគមន៍មួយកំណត់ចំពោះគ្រប់ចំនួនពិត $x$ ដោយ $y=f(x)=\df{(x+1)^2}{x^2+1}$ និង មានក្រាប $C$ ។

\begin{enumerate}

\item   គណនា $\lim_{x\to \pm \infty}f(x)$~រួចទាញរកសមីការអាស៊ីមតូតដេកនៃក្រាប $C$ ។

\item    គណនា និង សិក្សាសញ្ញាដេរីវេនៃ $f'(x)$ ។

\item   បង្ហាញថាអនុគមន៍ $f$  មានតម្លៃបរមាពីរផ្សេងគ្នា ។  គណនាតម្លៃបរមាទាំងនោះ ។ គូសតារាងអថេរភាពនៃ $f$ ។

\item   រកកូអរដោនេចំណុចប្រសព្វរវាងខ្សែកោង $C$  និង អ័ក្សទាំងពីរនៃតម្រុយ  ។ សង់ខ្សែកោង $C$ ។

\item   ដោយប្រើក្រាប $C$  ចូរពិភាក្សាតម្លៃ $m$ នូវអត្ថិភាព  និង សញ្ញាឫសរបស់សមីការ  $(m-1)x^2-2x+m-1=0$  ។

\end{enumerate}

\end{enumerate}


\end{document}



