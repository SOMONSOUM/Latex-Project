\documentclass{officialexam} 
\usepackage{chemfig}
\usepackage{tikz}
\usepackage{array}
\usepackage{circuitikz}
\usepackage{graphicx}
\graphicspath{ {./images/} }
\usepackage[version=4]{mhchem}
\usepackage{chemfig}
\begin{document}
	{\maketitle}\\
	\borderline{ប្រធាន ០១}
	\begin{enumerate}[I]
		\item {\color{khtug}(១២ ពិន្ទុ)} ចូរសរសេរសមីការអុីយ៉ុងសព្វ និងអុីយ៉ុងសម្រួលសម្រាប់ប្រតិកម្មខាងក្រោម៖
		\begin{enumerate}[k]
			\item $\ce{Zn(NO3)_2(aq) + (NH4)_2S(aq) -> }$
			\item $\ce{(NH4)_2CO_3(aq) + CaCl2_(aq) -> }$
			\item $\ce{BaCl2_(aq) + ZnSO4_(aq) -> }$
			\item $\ce{Na2S_(aq) + ZnCl2_(aq) -> }$
		\end{enumerate}
	 	\item {\color{khtug}(១២ ពិន្ទុ)} សមាសធាតុគីមីទាំងនេះជាសមាសធាតុអំផូទែៈ $\ce{H2O,NH3,HCO3^{-}}$ និង $\ce{HSO4^{-}}$
	 	\begin{enumerate}[k]
	 		\item ដូចម្តេចដែលហៅថាសមាសធាតុអំផូទែ?
	 		\item ចូរសរសេរគូទាំងពីររបស់សមាសធាតុនីមួយៗ។
	 	\end{enumerate}
 		\item {\color{khtug}(១៥ ពិន្ទុ)} ផ្មកំបោរអាចមានអំពើជាមួយអាស៊ីតក្លរីឌ្រិចតាមសមីការតុល្យការ\\ $\ce{CaCO3_(s) + 2H^{+}_(aq) -> Ca^{2+}_(aq) + CO2_(g) + H2O_(l)}$ ។ នៅខណៈ $t=0$ កំហាប់អុីយ៉ុង $\ce{Ca^{2+}}$ មានតម្លៃស្មើសូន្យ។ នៅខណៈ $t=15s$ កំហាប់អុីយ៉ុង $\ce{Ca^{2+}}$ កើតឡើងស្មើនឹង $1.8\times10^{-3}mol.L^{-1}$ និងនៅខណៈ $t=30s$ មានតម្លៃស្មើ\\ $3.13\times10^{-3}mol.L^{-1}$ ។
 		\begin{enumerate}[k]
 			\item តើប្រភេទគីមីណាខ្លះជាអង្គធាតុប្រតិករ និងប្រភេទគីមីណាខ្លះជាអង្គធាតុកកើត?
 			\item ចូរគណនាល្បឿនមធ្យមកំណអុីយ៉ុង $Ca^{2+}$ នៅចន្លោះពេល $15s$ និង $30s$។
 			\item ចូរទាញរកល្បឿនមធ្យមបំបាត់អុីយ៉ុង $H^{+}$។
 		\end{enumerate}
 		\item {\color{khtug}(១៨ ពិន្ទុ)} ការវិភាគម៉ូលេគុលអាមីនមួយ បានលទ្ធផលដូចតទៅ៖ កាបូន $61.02\%$ ~ អាសូត $23.73\%$ និងអុីដ្រូសែន $15.25\%$ គិតជាម៉ាស។
 		\begin{enumerate}[k]
 			\item កំណត់រូបមន្តដុលនៃអាមីននោះ។
 			\item សរសេររូបមន្តស្ទើរលាតដែលមានអាចមាន និងហៅឈ្មោះរបស់វា ។\\
 			គេឲ្យ $\ce{H=1,C=12,N=14}$ ។
 		\end{enumerate}
 		\item {\color{khtug}(១៨ ពិន្ទុ)} គេលាយសូលុយស្យុង $\ce{HCl}$ ចំនួន $\ce{10 mL}$ កំហាប់ $\ce{0.002M}$ ជាមួយសូលុយស្យុង $\ce{NaOH}$ ចំនួន $\ce{10mL}$ កំហាប់ $\ce{0.003M}$។
 		\begin{enumerate}[k]
 			\item គណនា $\ce{pH}$ របស់ល្យាយសូលុយស្យុងក្រោយប្រតិកម្ម។
 			\item តើគេត្រូវបន្ថែមអាសុីត ឬ បាសប៉ុន្មាន $\ce{mL}$ ដើម្បីឲ្យល្យាយទទួលបានសមមូលអាសុីត-បាស?
 		\end{enumerate}
	\end{enumerate}
\borderline{ចម្លើយ}\\
{\color{white}.}\dotfill\\
{\color{white}.}\dotfill\\
{\color{white}.}\dotfill
\\
{\color{white}.}\dotfill\\
{\color{white}.}\dotfill\\
{\color{white}.}\dotfill
\\
{\color{white}.}\dotfill\\
{\color{white}.}\dotfill\\
{\color{white}.}\dotfill
\\
{\color{white}.}\dotfill\\
{\color{white}.}\dotfill\\
{\color{white}.}\dotfill
\\
{\color{white}.}\dotfill\\
{\color{white}.}\dotfill\\
{\color{white}.}\dotfill
\\
{\color{white}.}\dotfill\\
{\color{white}.}\dotfill\\
{\color{white}.}\dotfill
\\
{\color{white}.}\dotfill\\
{\color{white}.}\dotfill\\
{\color{white}.}\dotfill
\\
{\color{white}.}\dotfill\\
{\color{white}.}\dotfill\\
\begin{center}
	\sffamily\color{blue}
	សូមសំណាងល្អ!
\end{center}\newpage
{\maketitle}\\
\borderline{ប្រធាន ០២}
	\begin{enumerate}[I]
		\item {\color{khtug}(១០ ពិន្ទុ)} គេឲ្យប្រតិកម្មគីមីមួយដូចតទៅ៖ $\ce{Fe(s) + 2HCl(aq) -> FeCl2(aq) + H2(g)}$។ ចូរពន្យល់ ហេតុអ្ចីបានជាប្រតិកម្មរវាង $\ce{Fe}$ និង $\ce{HCl}$ កើនឡើងល្បឿនកាលណា៖
		\begin{enumerate}[k,2]
			\item $\ce{Fe}$ ស្ថិតក្នុងភាពជាម្សៅ
			\item សីតុណ្ហភាពខ្ពស់
		\end{enumerate}
		\item {\color{khtug}(១០ ពិន្ទុ)} ចូរសរសេរសមីការសម្រាប់ការបំបែកសមាសធាតុអុីយ៉ុងក្នុងទឹក និងប្រាប់ពីចំនួនម៉ូលសរុបនៃអុីយ៉ុងដែលកើតឡើង៖ 
		\begin{enumerate}[k,2]
			\item $\ce{0.25}$ ម៉ូល អាលុយមីញ៉ូមក្លរូ
			\item $\ce{0.75}$ ម៉ូល សូដ្យូមស៊ុលផាត
		\end{enumerate}
		\item {\color{khtug}(១៥ ពិន្ទុ)} គេដាក់ម៉ាញេស្យូមឲ្យមានប្រតិ​កម្មជាមួយសូលុយស្យុងអាសុីតស៊ុលផរិច $\ce{H2SO4}$ (រាវ) ចំនួន $\ce{100mL}$ នៅកំហាប់ $\ce{3.00M}$ ។
		\begin{enumerate}[k]
			\item គណនាម៉ាសម៉ាញេស្យូមស៊ុលផាតដែលទទួលបាន។
			\item គណនាមាឌឧស្ម័នអុីដ្រូសែនដែលភាយនៅលក្ខខណ្ឌធម្មតា។\\
			គេឲ្យ៖ $S=32,Mg=24,O=16, Vm=22.4L.mol^{-1}$ ។
		\end{enumerate}
		\item {\color{khtug}(២០ ពិន្ទុ)} ចូរសរសេរទម្រង់សមាសធាតុខាងក្រោម ព្រមទាំលើកឧទារណ៍នីមួយៗមកបញ្ជាក់ផង៖
		\begin{enumerate}[k]
			\item អាល់កុលថ្នាក់ទី $\ce{I}$ អាល់កុលថ្នាក់ទី $\ce{II}$ អាល់កុលថ្នាក់ទី $\ce{III}$ 
			\item អាមីតថ្នាក់ទី $\ce{I}$ អាមីតថ្នាក់ទី $\ce{II}$ អាមីតថ្នាក់ទី $\ce{III}$
			\item អេស្ទែ
		\end{enumerate}
		\item  {\color{khtug}(២០ ពិន្ទុ)} \begin{enumerate}[k]
			\item ចូរគណនាម៉ាសជាក្រាមរបស់ស៊ូតចំាបាច់ដើម្បីធ្វើសូលុយស្យុង $\ce{NaOH 546mL}$ ដែលមាន $\ce{pH=10}$ ។\\គេឲ្យ៖ $O=16, Na=23, H=1$។
			\item រកកំហាប់អុីយ៉ុង $\ce{H3O^{+}(aq)}$ និង $\ce{OH^{-}(aq)}$ ក្នុងសូលុយស្យុងមួយដែលរៀបចំដោយ $\ce{0.200mol}$ នៃអាសុីត $\ce{HNO3}$ រលាយក្នុងទឹក $250mL$ ។ គេឲ្យ៖ $Ke=1\times10^{-14}, T=25^\circ C$ ។
			\item សូលុយស្យុងអាសុីតក្លរីឌ្រិចមួយធ្វើឡើងដោយរំលាយអាសុីតសុទ្ធ $18.4g$ ទៅក្នុងទឹក $662mL$ ។ ចូរគណនា $pH$ របស់សូលុយស្យុងនេះ។(ឧបមាថាមាឌសូលុយស្យុងនៅថេរ)។\\
			គេឲ្យ៖ $Cl=35.5, H=1, \log7.50=0.88$។
		\end{enumerate}
	\end{enumerate}
\borderline{ចម្លើយ}\\
{\color{white}.}\dotfill\\
{\color{white}.}\dotfill\\
{\color{white}.}\dotfill
\\
{\color{white}.}\dotfill\\
{\color{white}.}\dotfill\\
{\color{white}.}\dotfill
\\
{\color{white}.}\dotfill\\
{\color{white}.}\dotfill\\
{\color{white}.}\dotfill
\\
{\color{white}.}\dotfill\\
{\color{white}.}\dotfill\\
{\color{white}.}\dotfill
\\
{\color{white}.}\dotfill\\
{\color{white}.}\dotfill\\
{\color{white}.}\dotfill
\\
{\color{white}.}\dotfill\\
{\color{white}.}\dotfill\\
{\color{white}.}\dotfill
\\
{\color{white}.}\dotfill\\
{\color{white}.}\dotfill\\
{\color{white}.}\dotfill
\\
{\color{white}.}\dotfill\\
{\color{white}.}\dotfill\\
\begin{center}
	\sffamily\color{blue}
	សូមសំណាងល្អ!
\end{center}\newpage
{\maketitle}\\
\borderline{ប្រធាន ០៣}
\begin{enumerate}[I]
	\item {\color{khtug}(១០ ពិន្ទុ)} គេសំយោគអេស្ទែមួយ ដោយឲ្យអាសុីតប្រូប៉ាណូអុិច មានប្រតិកម្មជាមួយអេតាណុល។\\
	ចូរសរសេរសមីការគីមីតាងប្រតិកម្ម និងប្រាប់ឈ្មោះអេស្ទែនោះ។
	\item {\color{khtug}(១០ ពិន្ទុ)} សូលុយស្យុងអាម៉ូញាក់ក្នុងទឹកគឺជាបាស។ ចូរពន្យល់ ព្រមទាំងសរសេរសមីការគីមីបញ្ជាក់។
	\item {\color{khtug}(១៥ ពិន្ទុ)} តើនឹងមានអ្វីកើតឡើង នៅពេលដែលសូលុយស្យុងអាម៉ូញ៉ូមស៊ុលផួ និងកាត់ម្ញ៉ូមនីត្រាតត្រូវបានដាក់លាយបញ្ចូលគ្នា? ចូរសរសេរសមីការគីមី សមីការអុីយ៉ុងសព្វ សមីការអុីយ៉ុងសម្រួលសម្រាប់ប្រតិកម្មនេះ។
	\item  {\color{khtug}(១៥ ពិន្ទុ)} គេឲ្យម្សៅដែកមានប្រតិកម្មជាមួយសូលុយស្យុងអាសុីតក្លរីទ្រិច។ គេទទួលបានសូលុយស្យុងដែក $(II)$ ក្លរួ និងឧស្ម័នអុីដ្រូសែនភាយឡើង។
	\begin{enumerate}[k]
		\item ចូរសរសេរសមីការគីមី តាងប្រតិកម្មខាងលើ។
		\item ចូររៀបរាប់វិធីបួនយ៉ាង ដែលគេអាចប្រើដើម្បីវាស់ល្បឿននៃប្រតិកម្មនេះបាន។
		\item ក្នុងចំណោមវិធីទាំងបួននេះ តើវិធីណាមួយដែលងាយស្រួលជាងគេ? ចូរពន្យល់។
	\end{enumerate}
	\item {\color{khtug}(១៥ ពិន្ទុ)} សូលុយស្យុងមួយមាន $pH=10.70$ ។ ចូរគណនា៖
	\begin{enumerate}[k,2]
		\item កំហាប់អុីយ៉ុង $\ce{[H3O^{+}]}$
		\item កំហាប់អុីយ៉ុង $\ce{[OH^{-}]}$
		\item តើវាជាសូលុយស្យុងអាសុីត ឬសូលុយស្យុងបាស?\\
		គេឲ្យ $10^{0.3}=2~,~K_w$ នៅ $25^\circ C=10^{-14}$
	\end{enumerate}
	\item {\color{khtug}(១០ ពិន្ទុ)} \begin{enumerate}[k]
		\item តើទិន្នន័យអ្វីដែលគេត្រូវការ ដើម្បីគណនាកំហាប់របស់បាសដែលគេមិនស្គាល់?
		\item ចូរគណនាកំហាប់របស់អុីយ៉ុង $\ce{[OH^{-}]}$ ដែលមានក្នុងសូលុយស្យុង កាលណាគេដាក់ $59.0mL$ សូលុយស្យុង $\ce{HCl 0.3M}$ ឲ្យធ្វើប្រតិកម្មបន្សាបជាមួយសូលុយស្យុងបាស $50.0mL$ ។ 
	\end{enumerate}
\end{enumerate}
\borderline{ចម្លើយ}\\
{\color{white}.}\dotfill\\
{\color{white}.}\dotfill\\
{\color{white}.}\dotfill
\\
{\color{white}.}\dotfill\\
{\color{white}.}\dotfill\\
{\color{white}.}\dotfill
\\
{\color{white}.}\dotfill\\
{\color{white}.}\dotfill\\
{\color{white}.}\dotfill
\\
{\color{white}.}\dotfill\\
{\color{white}.}\dotfill\\
{\color{white}.}\dotfill
\\
{\color{white}.}\dotfill\\
{\color{white}.}\dotfill\\
{\color{white}.}\dotfill
\\
{\color{white}.}\dotfill\\
{\color{white}.}\dotfill\\
{\color{white}.}\dotfill
\\
{\color{white}.}\dotfill\\
{\color{white}.}\dotfill\\
{\color{white}.}\dotfill
\\
{\color{white}.}\dotfill\\
{\color{white}.}\dotfill\\
\begin{center}
	\sffamily\color{blue}
	សូមសំណាងល្អ!
\end{center}\newpage
{\maketitle}\\
\borderline{ប្រធាន ០៤}
\begin{enumerate}[I]
	\item  {\color{khtug}(១០ ពិន្ទុ)} ចូរសរសេររូបមន្តរបស់សមាសធាតុដូចខាងក្រោម៖
	\begin{enumerate}[k,2]
		\item មេទីលអេទីលប្រូប៉ាណូអាត
		\item ប្រូពីលមេតាណូអាត
		\item ទ្រីអេទីលឡាមីន
		\item ផេនីលអេតាណូអាត
	\end{enumerate}
	\item {\color{khtug}(១០ ពិន្ទុ)} កាល់ស្យូមកាបូណាតជាសមាសធាតុអុីយ៉ុងមិនរលាយក្នុងទឹក។ វាមានប្រតិកម្មជាមួយសូលុយស្យុង\\អាស៊ីតក្លរីឌ្រិចរាវ។
	\begin{enumerate}[k]
		\item ចូរសរសេរសមីការគីមី សមីការអុីយ៉ុងសព្វ និងសមីការអុីយ៉ុងសម្រួលនៃប្រតិកម្មនេះ។
		\item តើអុីយ៉ុងណាដែលគ្មានការប្រែប្រួលក្នុងពេលប្រតិកម្ម?
	\end{enumerate}
	\item {\color{khtug}(១២ ពិន្ទុ)} សូលុយស្យុងអាសុីតស៊ុលផួរិចមួយមានដង់សុីតេស្មើនឹង $1.198g/cm^3$ និងមានកំហាប់ភាគរយជាម៉ាសស្មើនឹង $27\%$។ គណនាកំហាប់ជាម៉ូលនៃសូលុយស្យុងអាសុីតនោះ។\\
	(ម៉ាសម៉ូល $H=1; S=32; O=16$)
	\item ក្នុង $100mL$ នៃសូលុយស្យុងស៊ូតដែលទទួលបាន គេឃើញមានស៊ូត $10^{-3} mol$ រលាយ។ គេបន្ថែមទឹក $400cm^3$ ទៅក្នុងសូលុយស្យុងនោះទៀត។ កំណត់តម្លៃនៃកំហាប់ជាម៉ូល របស់សូលុយស្យុងក្រោយនេះ។
	\item {\color{khtug}(១៥ ពិន្ទុ)} គេឲ្យប្រតិកម្មគីមីមួយដូចខាងក្រោម៖\\
	$\ce{Zn + 2HCl -> ZnCl2 + H2}$​\\
	 ចូរបកស្រាយថាប្រតិកម្មនេះជាប្រតិកម្មអុកស៊ីដូរេដុកម្ម។
	\item {\color{khtug}(២០ ពិន្ទុ)} គ្រូរបស់អ្នក ចង់ផលិតឧស្ម័នអុីដ្រូសែននៅក្នុងមន្ទីរពិសោធន៍ ដោយឲ្យអាសុីតស៊ុលផួរិចមានប្រតិកម្មជាមួយដុំលោហៈស័ង្កសី។\\ សូមផ្តល់គំនិតបីរបៀប ថាតើត្រូវធ្វើដូចម្តេច ដើម្បីឲ្យល្បឿននៃការផលិតឧស្ម័នអុីដ្រូសែនកាន់តែលឿនជាងមុន? ចូរពន្យល់។
\end{enumerate}
\borderline{ចម្លើយ}\\
{\color{white}.}\dotfill\\
{\color{white}.}\dotfill\\
{\color{white}.}\dotfill
\\
{\color{white}.}\dotfill\\
{\color{white}.}\dotfill\\
{\color{white}.}\dotfill
\\
{\color{white}.}\dotfill\\
{\color{white}.}\dotfill\\
{\color{white}.}\dotfill
\\
{\color{white}.}\dotfill\\
{\color{white}.}\dotfill\\
{\color{white}.}\dotfill
\\
{\color{white}.}\dotfill\\
{\color{white}.}\dotfill\\
{\color{white}.}\dotfill
\\
{\color{white}.}\dotfill\\
{\color{white}.}\dotfill\\
{\color{white}.}\dotfill
\\
{\color{white}.}\dotfill\\
{\color{white}.}\dotfill\\
{\color{white}.}\dotfill
\\
{\color{white}.}\dotfill\\
{\color{white}.}\dotfill\\
\begin{center}
	\sffamily\color{blue}
	សូមសំណាងល្អ!
\end{center}\newpage
{\maketitle}\\
\borderline{ប្រធាន ០៥}
\begin{enumerate}[I]
	\item {\color{khtug}(១២ ពិន្ទុ)} សិស្យម្នាក់ធ្វើអត្រាកម្មសូលុយស្យុងអាសុីតនីទ្រិចមិនស្គាល់កំហាប់ចំនួន $250mL$ ជាមួយសូលុយស្យុងសូល្យូមអុីដ្រុកសុីតកំហាប់ $0.20M$ មាឌ $200mL$។
	\begin{enumerate}[k]
		\item តើគេត្រូវប្រើអង្គធាតុចង្អុលពណ៍អ្វីសម្រាប់អត្រាកម្មនេះ?
		\item ចូរសរសេរសមីការតាងប្រតិកម្មនេះ។ តើប្រតិកម្មនេះជាប្រតិកម្មអ្វី?
		\item រកកំហាប់ជាម៉ូលរបស់សូលុយស្យុងអាសុីតនីទ្រិចដែលប្រើ។
	\end{enumerate}
	\item {\color{khtug}(១២ ពិន្ទុ)} គេយក $0.15mol$ នៃ $\ce{Cl2}$ និង $0.30mol$ នៃ $\ce{NO2}$ ដាក់ក្នុងប្រអប់បិទជិតដែលមានចំណុះ $1.50L$។ គេទុកឲ្យប្រព័ន្ធមានលំនឹងនៅសីតុណ្ហភាពកំណត់មួយ។ កំហាប់ $\ce{NO2Cl}$ ពេលមានលំនឹងគឺ $0.054mol.L^{-1}$។ ចូរគណនាតម្លៃ $K$ នៅសីតុណ្ហភាពនោះ។ គេឲ្យសមីការតុល្យការលំនឹង៖ $\ce{2NO2_(g) + Cl2_(g)} \rightleftharpoons \ce{2NO2Cl_(g)}$
	\item {\color{khtug}(១៥ ពិន្ទុ)} គេលាយសូលុយស្យុង $\ce{H2SO4}$ ចំនួន $10mL$ កំហាប់ $0.0025M$ ជាមួយសូលុយស្យុង $NaOH$ ចំនួន $10mL$ កំហាប់ $0.003M$ ។
	\begin{enumerate}[k]
		\item តើល្បាយដែលទទួលបានមានភាពជាអាស៊ីត ឬជាបាស ឬជាណឺត?
		\item ចូរគណនា $pH$ របស់ល្បាយនោះ។
	\end{enumerate}
	\item {\color{khtug}(១៨ ពិន្ទុ)}​\begin{enumerate}[k]
		\item នៅសីតុណ្ហភាពជាក់លាក់មួយ អាសុីតក្លរីឌ្រិច $\ce{HCl}$ មានប្រតិកម្មជាមួយថ្មម៉ាប់ ឬ$\ce{CaCO3}$។\\
		ចូរពណ៌នាពីវិធីពីរយ៉ាងដែលធ្វើឲ្យល្បឿននៃប្រតិកម្មនេះកាន់តែលឿន។
		\item គេឲ្យសូលុយស្យុងសូដ្យូមអុីដ្រុកសុីតមានប្រតិកម្មជាមួយសូលុយស្យុងស័ង្កសីនីត្រាត គេសង្កេតឃើញមានកករពណ៍សកើតឡើង។ ចូរសរសេរសមីការតាងប្រតិកម្ម សមីការអុីយ៉ុងសព្វ និងសមីការអុីយ៉ុងសម្រួលនៃប្រតិកម្មនេះ។
		\item ចូរបង្ហាញថាប្រតិកម្មខាងក្រោមនេះ ជាប្រតិកម្មឌីស្មូតកម្ម។
		 $\ce{S2O3^{2-} + 2H^{+} -> S + SO2 + H2O}$ 
	\end{enumerate}
	\item {\color{khtug}(១៨ ពិន្ទុ)} ចំហេះសព្វអេស្ទែឆ្អែតមួយ ចំនួន $1.02g$ បានផ្តល់ឧស្ម័នកាបូនឌីអុកសុីត $\ce{(CO2)}$ ចំនួន $2.20g$។
	\begin{enumerate}[k]
		\item ចូរកំណត់រូបមន្តរបស់អេស្ទែនោះ។
		\item ចូរសរសេររូបមន្តស្ទើលាត និងហៅឈ្មោះរបស់អេស្ទែដែលអាចមាន។\\
		គេឲ្យ $H=1, C=12, 0=16$ ។
	\end{enumerate}
\end{enumerate}
\borderline{ចម្លើយ}\\
{\color{white}.}\dotfill\\
{\color{white}.}\dotfill\\
{\color{white}.}\dotfill
\\
{\color{white}.}\dotfill\\
{\color{white}.}\dotfill\\
{\color{white}.}\dotfill
\\
{\color{white}.}\dotfill\\
{\color{white}.}\dotfill\\
{\color{white}.}\dotfill
\\
{\color{white}.}\dotfill\\
{\color{white}.}\dotfill\\
{\color{white}.}\dotfill
\\
{\color{white}.}\dotfill\\
{\color{white}.}\dotfill\\
{\color{white}.}\dotfill
\\
{\color{white}.}\dotfill\\
{\color{white}.}\dotfill\\
{\color{white}.}\dotfill
\\
{\color{white}.}\dotfill\\
{\color{white}.}\dotfill\\
{\color{white}.}\dotfill
\\
{\color{white}.}\dotfill\\
{\color{white}.}\dotfill\\
\begin{center}
	\sffamily\color{blue}
	សូមសំណាងល្អ!
\end{center}\newpage
{\maketitle}\\
\borderline{ប្រធាន ០៦}
\begin{enumerate}[I]
	\item {\color{khtug}(\kml{១០ ពិន្}ទុ)} គេឲ្យប្រតិកម្មរវាងឧស្ម័នស្ពាន់ធ័រឌីអុកសុីត និងអុកសុីសែនឲ្យផលជាឧស្ម័នស្ពាន់ធ័រទ្រីអុកសុីត។\\ គេឲ្យប្រព័ន្ធនេះលំនឹងនៅសីតុណ្ហភាព $873^\circ C$ កំហាប់នៃសារធាតុនីមួយៗនៅពេលមានលំនឹងគឺ $\left[\ce{SO2}\right]=1.50M, \left[\ce{O2}\right]=1.25M$ និង $\left[\ce{SO3}\right]=3.50M$។ 
	\begin{enumerate}[k,2]
		\item ដូចម្តេចដែលហៅថាថេរលំនឹងគីមី?
		\item គណនាថេរលំនឹងនៃប្រព័ន្ធ។
	\end{enumerate}
	\item {\color{khtug}(\kml{១៥ ពិន្ទុ})} ក្នុងមជ្ឈដ្ឋានអាសុីតអុីយ៉ុងត្យូស៊ុលផាតធ្វើប្រតិកម្មយឺតប្លែងជាស្ពាន់ធ័រ និងស្ពាន់ធ័រឌីអុកសុីត។\\
	សមីការតុល្យការតាងប្រតិកម្មៈ $\ce{S2O3^{2-} + 2H+ -> S + SO2 + H2O}$។ 
	\begin{enumerate}[m]
		\item ចូរសរសេរគូរេដុកដែលចូររួបប្រតិកម្ម និងកន្លះសមីការអេឡិចត្រូនិចនៃគូរេដុកនីមួយៗ
		\item តើប្រតិកម្មខាងលើអាចចាត់ទុកជាប្រតិកម្មអុកសុីដូរេដុកកម្មបានដែរ ឬទេ? ព្រោះអ្វី?
		\item តើល្បឿនបំបាត់អុីយ៉ុង $\ce{S2O3^{2-}}$ ប្រែប្រួលដូចម្តេច កាលណា៖
		\begin{enumerate}[k,2]
			\item $\ce{S2O3^{2-}}$ កើន? 
			\item ពង្រាវសូលុយស្យុងដើម?
		\end{enumerate}
		\item បើល្បឿនបំបាត់ $\ce{S2O3^{2-}}$ ខណៈ $t$ គឺ $10^{-4}mol\cdot L^{-1}\cdot s^{-1}$។ ចូរគណនាល្បឿនបំបាត់ $\ce{H+}$ ខណៈ $t$។
	\end{enumerate}
	\item {\color{khtug}(\kml{១៥ ពិន្ទុ})} សូលុយស្យុងកាល់ស្យូមអុីដ្រុកសុីត $\ce{Ca(OH)2}$ មួយមានកំហាប់ $C_B=5\times10^{-2}M$ ចំនួន $400mL$។
	\begin{enumerate}[k]
		\item ចូរសរសេរសមីការតាងប្រតិកម្ម $\ce{Ca(OH)2}$ ក្នុងទឹក និងគណនា $\ce{pH}$ នៃសូលុយស្យុងនេះ។
		\item គណនាម៉ាស $\ce{Ca(OH)2}$ សម្រាប់រំលាយទៅក្នុងទឹក ដើម្បីទទួលបានសូលុយស្យុងខាងលើ។\\
		គេឲ្យ $\left(\ce{Ca=40, O=16, H=1, log 5=0.7, log 1=0}\right)$
	\end{enumerate}
	\item {\color{khtug}(\kml{១៥ ពិន្ទុ})} គេមានសូលុយស្យុងសូល្យូមភ្លុយអរួ $\ce{NaF}$ មានកំហាប់ $0.3M$ មាន $\ce{pH=8.3}$។
	\begin{enumerate}[k]
		\item គណនាកំហាប់អុីយ៉ុង $\ce{OH-}$ ទទួលបាន។
		\item គណនាថេរលំនឹងបាស $\ce{Kb}$ នៃគូរ $\ce{HF/F^{-}}$។ គេឲ្យ $\ce{10^{0.3}=2, 10^{0.7}=5}$ និង $K_w=1 \times10^{-14}$
	\end{enumerate}
	\item {\color{khtug}(\kml{២០ ពិន្ទុ})} គេធ្វើប្រតិកម្មរវាង $\left(\ce{CH3-CH2-COOH}\right)$ ចំនួន $25mL$ កំហាប់ $0.3M$ ជាមួយ $\left(\ce{CH3-CHOH-C2H5}\right)$។
	\begin{enumerate}[m]
		\item ចូរសរសេរសមីការតាងប្រតិកម្ម​ ប្រាប់ឈ្មោះប្រតិកម្មនេះ និង ហៅឈ្មោះសមាសធាតុដែលទទួលបាន។
		\item ក្រោយប្រតិកម្មចប់ គេយកអាសុីតនៅសល់ទៅធ្វើប្រតិកម្មជាមួយ $\ce{KOH}$ ម៉ាស $5.6g$ រលាយក្នុងមាឌ $500mL$។ នៅចំណុចសមមូលអាសុីតបាស គេប្រើសូលុយស្យុង $KOH$ អស់ $12mL$។
		\begin{enumerate}[k]
			\item សរសេរសមីការតាងប្រតិកម្មកើតមានក្នុងអត្រាកម្មខាងលើ។
			\item គណនាចំនួនម៉ូលនៃអាសុីតដើម ម៉ូលអាសុីតចូរប្រតិកម្ម និងម៉ូលអាសុីតនៅសល់។
			\item គណនាម៉ាសអេស្ទែទទួលបាន។
			\item គណនាភាគរយអាសុីតដែលចូរប្រតិកម្ម។ គេឲ្យ $\ce{H=1, C=12, K=39, O=16}$
		\end{enumerate}
	\end{enumerate}
\end{enumerate}\newpage
\borderline{ចម្លើយ}\\
{\color{white}.}\dotfill\\
{\color{white}.}\dotfill\\
{\color{white}.}\dotfill
\\
{\color{white}.}\dotfill\\
{\color{white}.}\dotfill\\
{\color{white}.}\dotfill
\\
{\color{white}.}\dotfill\\
{\color{white}.}\dotfill\\
{\color{white}.}\dotfill
\\
{\color{white}.}\dotfill\\
{\color{white}.}\dotfill\\
{\color{white}.}\dotfill
\\
{\color{white}.}\dotfill\\
{\color{white}.}\dotfill\\
{\color{white}.}\dotfill
\\
{\color{white}.}\dotfill\\
{\color{white}.}\dotfill\\
{\color{white}.}\dotfill
\\
{\color{white}.}\dotfill\\
{\color{white}.}\dotfill\\
{\color{white}.}\dotfill
\\
{\color{white}.}\dotfill\\
{\color{white}.}\dotfill\\
\begin{center}
	\sffamily\color{blue}
	សូមសំណាងល្អ!
\end{center}\newpage
{\maketitle}\\
\borderline{ប្រធាន ០៧}
\begin{enumerate}[I]
	\item {\color{khtug}(\kml{១០ ពិន្ទុ})} តើសូលុយស្យុងណាខាងក្រោមនេះណាខ្លះជាសូលុយស្យុងតំប៉ុង និងណាខ្លះមិនមែនជាសូលុយស្យុងតំប៉ុង?
	\begin{enumerate}[k]
		\item សូលុយស្យុងមាន $\ce{NH3}~0.2mol$ និង $\ce{NH4Cl}~0.2mol$
		\item សូលុយស្យុងមាន $\ce{H2SO4}~0.2mol$ និង $\ce{CH3COONa}~0.8mol$
		\item សូលុយស្យុងមាន $\ce{HF}~0.2mol$ និង $\ce{KOH}~0.1mol$
		\item សូលុយស្យុងមាន $\ce{CH3COOH}~0.2mol$ និង $\ce{Ca(OH)2}~0.1mol$
		\item សូលុយស្យុងមាន $\ce{HCl}~0.2mol$ និង $\ce{NaOH}~0.2mol$
	\end{enumerate}
	\item {\color{khtug}(\kml{១០ ពិន្ទុ})} ថ្មកំបោរមានអំពើជាមួយអាសុីតក្លរីឌ្រិចតាមសមីការតុល្យការៈ\\ $\ce{CaCO3(s) + 2H+(aq)} \rightarrow \ce{Ca^{2+}(aq) + CO2(g) + H2O(l)}$។ នៅខណៈ $t_0=0$ កំហាប់អុីយ៉ុង $\ce{Ca^{2+}}$ មានតម្លៃស្មើសូន្យ។ នៅខណៈ $t_1=15min$ កំហាប់អុីយ៉ុង $\ce{Ca^{2+}}$ កើតឡើងស្មើនឹង $1.8\times10^{-3}M$។ នៅខណៈ $t_2=30min$ កំហាប់អុីយ៉ុង $\ce{Ca^{2+}}$ កើតឡើងស្មើនឹង $3.13\times10^{-3}M$។ នៅខណៈ $t_3=45min$ កំហាប់អុីយ៉ុង $\ce{Ca^{2+}}$ កើតឡើងស្មើនឹង $3.43\times10^{-3}M$។ ចូរគណនាល្បឿនមធ្យមបំបាត់អុីយ៉ុង $\ce{H+}$ នៅចន្លោះ $t_2$ និង $t_3$។ 
	\item \begin{enumerate}[m]
		\item {\color{khtug}(\kml{៥ ពិន្ទុ})} ចូរព្យាករណ៍ទិសដៅលំនឹងនៃប្រតិកម្មនេះ $\ce{2C(s) + O2(g)}\rightleftharpoons\ce{2CO(g)}+\text{កម្តៅ}$
		\begin{enumerate}[k,5]
			\item បង្កើនសម្ពាធ
			\item បន្ថយមាឌ
			\item បង្កើនកម្តៅ
			\item បន្ថែម​ $\ce{CO}$
			\item បន្ថែម $\ce{C}$
		\end{enumerate}
		\item {\color{khtug}(\kml{៥ ពិន្ទុ})} នៅសីតុណ្ហភាព $525^\circ C$ ប្រតិកម្មបំបែកក្រាមកាល់ស្យូមកាបូណាតឲ្យផលជាក្រាមកាលស្យូមអុកសុីត និងឧស្ម័នកាបូនិច មានថេរលំនឹង $K$ ស្មើនឹង $3.35\times10^{-3}$។ ចូររកកំហាប់កាបូនឌីអុកសុីតនៅពេលប្រតិកម្មមានលំនឹង។
	\end{enumerate}
	\item  {\color{khtug}(\kml{១៥ ពិន្ទុ})} គេយកសូលុយស្យុងអាសុីតក្លរីឌ្រិចចំនួន $20mL$ កំហាប់​ $0.01M$ ឲ្យមានប្រតិកម្មជាមួយសូលុយស្យុងបា​​រ្យូ​​មអុីដ្រុកសុីតចំនួន $20mL$។
	\begin{enumerate}[k]
		\item ចូរគណនាកំហាប់អុីយ៉ុងអុីដ្រុកសុីតនៃសូលុយស្យុងបា​រ្យូមអុីដ្រុកសុីតដែលបានយកមកប្រើដើម្បីធ្វើឲ្យល្បាយប្រតិកម្មដល់ចំណុចសមមួលអាសុីតបាស។
		\item ចូរគណនា $\ce{pH}$ នៃសូលុយស្យុងបា​រ្យូមអុីដ្រុកសុីតដែលបានយកមកប្រើ។
	\end{enumerate}
	\item  {\color{khtug}(\kml{១៥ ពិន្ទុ})} គ្រូម្នា់បានធ្វើប្រតិកម្មរវាងអាសុីតអាសេទិចចំនួន $200mL$ និងអុីសូប្រូពីលអាល់កុល គេទទួលបានអេស្ទែ $2.04g$។ ដោយដឹងថាទិន្នផលនៃប្រតិកម្មមាន $60\%$។\\
	ចូរគណនាកំហាប់អាសុីតអេតាណូអុិចដែលត្រូវយកមកប្រើ។ រួចចូរឲ្យឈ្មោះអេស្ទែកកើត។ គេឲ្យៈ $\ce{H=1, C=12}$ និង $\ce{O=16}$។
	\item {\color{khtug}(\kml{១៥ ពិន្ទុ})} គ្រូម្នាក់បានវិភាគសមាសធាតុសរីរាង្គឆ្អែត $A$ (រូបមន្តទូទៅ $\ce{CxHyO2}$) មាន $\%C=40\%$ និង $\%H=6.67\%$។
	\begin{enumerate}[k]
		\item ចូរកំណត់រូបមន្តដុលរបស់ $A$។ គេឲ្យៈ $\ce{H=1, C=12}$ និង $\ce{O=16}$
		\item ចូរសរសេររូបមន្តស្ទើរលាត $A$ ដែលអាចមាន និងព្រមទាំងឲ្យឈ្មោះធ្លាប់ប្រើ។
		\item ដោយដឹងថា $A$ អាចធ្វើប្រតិកម្មអេស្ទែកម្ម។ តើ $A$ មានរូបមន្តដូចម្តេច? រួចចូរឲ្យឈ្មោះផ្លូវការរបស់វា។
	\end{enumerate}
\end{enumerate}\newpage
\borderline{ចម្លើយ}\\
{\color{white}.}\dotfill\\
{\color{white}.}\dotfill\\
{\color{white}.}\dotfill
\\
{\color{white}.}\dotfill\\
{\color{white}.}\dotfill\\
{\color{white}.}\dotfill
\\
{\color{white}.}\dotfill\\
{\color{white}.}\dotfill\\
{\color{white}.}\dotfill
\\
{\color{white}.}\dotfill\\
{\color{white}.}\dotfill\\
{\color{white}.}\dotfill
\\
{\color{white}.}\dotfill\\
{\color{white}.}\dotfill\\
{\color{white}.}\dotfill
\\
{\color{white}.}\dotfill\\
{\color{white}.}\dotfill\\
{\color{white}.}\dotfill
\\
{\color{white}.}\dotfill\\
{\color{white}.}\dotfill\\
{\color{white}.}\dotfill
\\
{\color{white}.}\dotfill\\
{\color{white}.}\dotfill\\
\begin{center}
	\sffamily\color{blue}
	សូមសំណាងល្អ!
\end{center}\newpage
{\maketitle}\\
\borderline{ប្រធាន ០៨}
\begin{enumerate}[I]
	\item ចូរសរសេរទម្រង់អាមីនថ្នាក់ទី $I$ ថ្នាក់ទី $II$ និងថ្នាក់ទី $III$ ព្រមទាំងលើកឧទាហរណ៍ទម្រង់អាមីនមួយៗមកបញ្ជាក់ផង។
	\item ចូរសរសេររូបមន្តនៃសមាសធាតុខាងក្រោម៖
	\begin{enumerate}[k,2]
		\item មេទីលអេទីលប្រូប៉ាណូអាត
		\item ផេនីលអេតាណូអាត
		\item ទែត្យូប៊ុយទីលផរម្ញ៉ាត
		\item អានីឌ្រីតបង់សូអុិច
	\end{enumerate}
	\item ហេតុអ្វីបានជាឧស្ម័នធ្វើប្រតិកម្មលឿនកាលណាគេបង្កើនសម្ពាធទៅលើវា? ចូរពន្យល់។
	\item គេលាយសូលុយស្យុងបា​រ្យូមក្លរួ និងសូ​ដ្យូមស៊ុលផាតចូលគ្នា។ ចូរសរសេរសមីការគីមី សមីការអុីយ៉ុងសព្វ សមីការអុីយ៉ុងសម្រួល ព្រមទាំងប្រាប់អុីយ៉ុងទស្សនិក។
	\item គេប្រើសូលុយស្យុង $\ce{HCl}$ ចំនួន $40mL$ នៅកំហាប់ $0.3388M$ ដើម្បីធ្វើអត្រាកម្មសូលុយស្យុង $\ce{NaOH} ~24.64mL$ ។​\\រកកំហាប់របស់សូលុយស្យុង $\ce{NaOH}$។
	\item គេយអាសុីតភ្លុយអរីឌ្រិច $\ce{HF}$ ចំនួន $0.015mol$ និងប៉ូតាស្យូមភ្លុយអរួ $\ce{KF}$ ចំនួន $0.045mol$ ដាក់ក្នុងកែវពិសោធន៍រួចបន្ថែមទឹកចូលឲ្យបានសូលុយស្យុងមួយមានមាឌ $300mL$។\\
	គណនាកំហាប់អុីយ៉ុងអុីដ្រូញ៉ូម និង $pH$ នៃសូលុយស្យុង។ គេឲ្យៈ $K_a=6.7\times10^{-4}$
	\item សូលុយស្យុងអាសុីតក្លរីឌ្រិច $\ce{HCl}$ មួយមានកំហាប់ $0.001M$។ ចូរគណនាៈ
	\begin{enumerate}[k]
		\item កំហាប់អុីយ៉ុងអុីដ្រូញ៉ូម $\ce{[H3O+]}$។
		\item កំហាប់អុីយ៉ុងអុីដ្រូញ៉ូម $\ce{[OH-]}$។
		\item $\ce{pH}$ របស់សូលុយស្យុង។
	\end{enumerate}
	\item ទិន្នន័យខាងក្រោមប្រមូលបានអំឡុងពេលសិក្សាប្រតិកម្មៈ $\ce{H2O2(aq) + 2H+(aq) -> I2(aq) + 2H2O(l)}$
	\begin{center}
		\begin{tabular}{ | c | c | c |}
			\hline
			\text{រយៈពេល $t(s)$} & $\left[\ce{H+}\right]M$ \text{ឬ}~$mol\cdot L^{-1}$ & $\left[\ce{I_2}\right]M$ \text{ឬ}~$mol\cdot L^{-1}$ \\ \hline
			$0$ & $0.0500$ & $0$ \\ \hline
			$85$ & $0.0298$ & $0.0101$ \\ \hline
			$95$ & $0.0280$ & $0.0110$ \\ \hline
			$105$ & $0.0254$ & $0.0118$ \\ \hline
		\end{tabular}
	\end{center}
	\begin{enumerate}[k]
		\item តើប្រភេទគីមីណាខ្លះជាអង្គធាតុប្រតិករ និងប្រភេទគីមីណាខ្លះជាអង្គធាតុកកើត?
		\item គណនាល្បឿនមធ្យមបំបាត់អុីយ៉ុង $\ce{H+}$ និងល្បឿនមធ្យមកំណ $I_2$ នៅចន្លោះពេល $t=85s$ និង $t=105s$។
	\end{enumerate}
\end{enumerate}
\borderline{ចម្លើយ}\\
{\color{white}.}\dotfill\\
{\color{white}.}\dotfill\\
{\color{white}.}\dotfill
\\
{\color{white}.}\dotfill\\
{\color{white}.}\dotfill\\
{\color{white}.}\dotfill
\\
{\color{white}.}\dotfill\\
{\color{white}.}\dotfill\\
{\color{white}.}\dotfill
\\
{\color{white}.}\dotfill\\
{\color{white}.}\dotfill\\
{\color{white}.}\dotfill
\\
{\color{white}.}\dotfill\\
{\color{white}.}\dotfill\\
{\color{white}.}\dotfill
\\
{\color{white}.}\dotfill\\
{\color{white}.}\dotfill\\
{\color{white}.}\dotfill
\\
{\color{white}.}\dotfill\\
{\color{white}.}\dotfill\\
{\color{white}.}\dotfill
\\
{\color{white}.}\dotfill\\
{\color{white}.}\dotfill\\
\begin{center}
	\sffamily\color{blue}
	សូមសំណាងល្អ!
\end{center}\newpage
{\maketitle}\\
\borderline{ប្រធាន ០៩}
\begin{enumerate}[I]
	\item ចូរសរសេរទម្រង់អាមីតថ្នាក់ទី $I$ ថ្នាក់ទី $II$ និងថ្នាក់ទី $III$ ព្រមទាំងលើកឧទាហរណ៍ទម្រង់អាមីតមួយៗមកបញ្ជាក់ផង។
	\item នៅសីតុណ្ហភាពជាក់លាក់មួយ អាសុីតក្លរីឌ្រិចមានប្រតិកម្មជាមួយដុំកាល់ស្យូមកាបូណាតដោយល្បឿនយឺតជាងម្សៅកាបូណាត។
	\begin{enumerate}[k]
		\item ចូរសរសេរសមីការតាងប្រតិកម្មគីមី។
		\item ចូរពន្យល់ថាហេតុអ្វីបានជាដុំ $\ce{CaCO3}$ មានល្បឿនប្រតិកម្មយឺតជាងម្សៅ $\ce{CaCO3}$។
		\item សូលុយស្យុងអាសុីតក្លរីឌ្រិចខាងលើ មានកំហាប់ $0.10M$ ចំនួន $200mL$។ រកម៉ាសកាល់ស្យូមកាបូណាតដែលប្រើ។
	\end{enumerate}
	\item \begin{enumerate}[m]
		\item ចូរធ្វើអត្តសញ្ញាណកម្មសមាសធាតុខាងក្រោមនេះជាអេឡិចត្រូលីតខ្លាំង អេឡិចត្រូលីតខ្សោយ និងមិនមែនអេឡិចត្រូលីតៈ\\
		$\ce{HCN};\quad\ce{(NH2)2CO};\quad \ce{HOOC-COOH};\quad \ce{NaNO2};\quad \ce{(NH4)2S};\quad \ce{C2H5OH};\quad \ce{KOH};\quad \ce{AgCl}$
		\item ចូរឲ្យនិយមន័យភាពខុសគ្នារវាងអេឡិចត្រូលីតខ្លាំង និងអេឡិចត្រូលីតខ្សោយ។
	\end{enumerate}
	\item គេយកគ្រាប់ស័ង្កសីឲ្យមានប្រតិកម្មជាមួយនឹងសូលុយស្យុងអាសុីតស៊ុលផួរិច ចំនួន $50mL$ នៅកំហាប់ $0.05M$។
	\begin{enumerate}[k]
		\item ចូរសរសេរសមីកាគីមី សមីការអុីយ៉ុងសព្វ និងសមីការអុីយ៉ុងសម្រួល។
		\item រកម៉ាសអំបិលដែលទទួលបាន។
		\item រកមាឌឧស្ម័នអុីដ្រូសែនដែលភាយចេញពីប្រតិកម្មនៅសីតុណ្ហភាព $STP$។
		គេឲ្យៈ $\ce{Zn=65, S=32, O=16, H=1, Vm=22.4L/mol}$
	\end{enumerate}
	\item $0.2mol$ ដែលត្រូវនឹង $24.5g$ អាសុីត $\alpha -$ក្លរ៉ូកាបុកសុីលិចឆ្អែតមួយ បង្កើតបាន $20.6g$ អាសុីត $\alpha -$អាមីណូកាបុកសុីលិចឆ្អែត។
	\begin{enumerate}[k]
		\item តើអាសុីត $\alpha -$អាមីណូកាបុកសុីលិចឆ្អែតនោះមានរូបមន្តដូចម្តេច? ឈ្មោះអ្វី?
		\item តើនៅក្នុងម៉ូលេគុលរបស់វាមានបង្គុំនាទីអ្វីខ្លះ?
		\item ចូរសរសេរសមីការបង្កើតឌីបុិបទីតពីអាសុីត $\alpha -$អាមីណូកាបុកសុីលិចឆ្អែត។
	\end{enumerate}
	\item គេយកសូលុយស្យុងអាសុីតស៊ុលផួរិច $50mL$ ទៅធ្វើអត្រាកម្មដោយសូលុយស្យុងស៊ូតដែលមានកំហាប់ $0.2mol\cdot L^{-1}$។ កាលណាគេបន្តក់សូលុយស្យុងស៊ូត $25mL$ ចូលគេសង្កេតឃើញអង្គធាតុចង្អុលពណ៌ប្រែពណ៌។
	\begin{enumerate}[k]
		\item ចូរគូសគំនូសបំព្រួញនៃការធ្វើអត្រាកម្មនេះ។ ហេតុអ្វីបានគេចាំបាច់ធ្វើអត្រាកម្ម?
		\item តើអង្គធាតុចង្អុលពណ៌ណាមួយដែលសមស្របជាងគេសម្រាប់អត្រាកម្មនេះ?
		\item សរសេរមីការតុល្យការប្រតិកម្មនៃអត្រាកម្ម និងគណនាកំហាប់ជាម៉ូលអាសុីតស៊ុលផួរិចដែលប្រើ។
		\item តើអង្គធាតុចង្អុលពណ៌មាននាទីដូចម្តេចនៅក្នុងអត្រាកម្ម?
	\end{enumerate}
\end{enumerate}
\borderline{ចម្លើយ}\\
{\color{white}.}\dotfill\\
{\color{white}.}\dotfill\\
{\color{white}.}\dotfill
\\
{\color{white}.}\dotfill\\
{\color{white}.}\dotfill\\
{\color{white}.}\dotfill
\\
{\color{white}.}\dotfill\\
{\color{white}.}\dotfill\\
{\color{white}.}\dotfill
\\
{\color{white}.}\dotfill\\
{\color{white}.}\dotfill\\
{\color{white}.}\dotfill
\\
{\color{white}.}\dotfill\\
{\color{white}.}\dotfill\\
{\color{white}.}\dotfill
\\
{\color{white}.}\dotfill\\
{\color{white}.}\dotfill\\
{\color{white}.}\dotfill
\\
{\color{white}.}\dotfill\\
{\color{white}.}\dotfill\\
{\color{white}.}\dotfill
\\
{\color{white}.}\dotfill\\
{\color{white}.}\dotfill\\
\begin{center}
	\sffamily\color{blue}
	សូមសំណាងល្អ!
\end{center}\newpage
{\maketitle}\\
\begin{enumerate}[m]
	\item {\color{khtug}\kml (១០ ពិន្ទុ)} ល្បាយមួយរួមមាន $\ce{NO}=3.9mol$ និង $\ce{CO2}=0.88mol$ មានប្រតិកម្មនៅក្នុងកែវមួយចំណុះ $1L$ នៅសីតុណ្ហភាពជាក់លាក់តាមប្រតិកម្ម $\ce{NO(g) + CO2 (g)}\rightleftharpoons \ce{NO2 (g) + CO (g)}$។ នៅពេលមានលំនឹងគេឃើញមាន $\ce{CO2}$ ចំនួន $0.11mol$។
	\begin{enumerate}[k]
		\item គណនាតម្លៃថេរលំនឹង $K$ នៅសីតុណ្ហភាពនោះ។
		\item តើកត្តាអ្វីខ្លះដែលធ្វើឲ្យលំនឹងនៃប្រតិកម្មនេះមានការប្រែប្រួល?
	\end{enumerate}
	\item {\color{khtug}\kml (១០ ពិន្ទុ)} ម៉ូលេគុលអាម៉ូញាក់ $\ce{NH3}$ គេចាត់ទុកជាសមាសធាតុអំផូទែ។
	\begin{enumerate}[k]
		\item អ្វីទៅជាសមាសធាតុអំផូទែ?
		\item ចូរសរសេរគូរអាសុីត-បាសទាំងពីរនៃសមាសធាតុនេះ។
		\item សរសេរសមីការអូតូប្រូតូលីសបង្ហាញថាម៉ូលេគុល $\ce{NH3}$ ជាអំផូទែ។
	\end{enumerate}
	\item {\color{khtug}\kml (១០ ពិន្ទុ)} គេមាន $500mL$ នៃសូលុយស្យុង​សូល្យូមស៊ុលផាត $\left(\ce{Na2SO4}\right)$ គេមិនស្គាល់កំហាប់។ គេបានបន្ថែម $50mL$ \\នៃសូលុយស្យុងបារ្យ៉ូមក្លរួ $\left(\ce{BaCl2}\right)$ ទៅក្នុងសូលុយស្យុងនេះ គេទទួលបានកករបារ្យ៉ូមស៊ុលផាត $\left(\ce{BaSO4}\right)$ ចំនួន $2.33g$។
	\begin{enumerate}[k]
		\item សរសេរសមីការជំនួសទ្វេ សមីការអុីយ៉ុងសព្វ និងសមីការអុីយ៉ុងសម្រួល។
		\item គណនាកំហាប់ជាម៉ូលនៃសូលុយស្យុងសូល្យូមស៊ុលផាត $\left(\ce{Na2SO4}\right)$ ដើម។
		\item គណនាកំហាប់ជាម៉ូលនៃសូលុយស្យុងបារ្យ៉ូមក្លរួ $\left(\ce{BaCl2}\right)$ ដើម។\\
		គេឲ្យៈ $\ce{Ba=137}, \ce{ Cl=35.5}, \ce{S=32}, \ce{O=16}, \ce{Na=23}$
	\end{enumerate}
	\item {\color{khtug}\kml (១៥ ពិន្ទុ)} គេយកឧស្ម័នអាម៉ូញ៉ាក់ $\left(\ce{NH3}\right)$ ចំនួន $672mL$ និងឧស្ម័នកាបូនិច $\left(\ce{CO2}\right)$ $448mL$\\ ធ្វើប្រតិកម្មជាមួយគ្នានៅលក្ខខណ្ឌធម្មតា $\left(STP\right)$។
	\begin{enumerate}[k]
		\item សរសេរសមីការតាងប្រតិកម្មខាងលើ។
		\item តើឧស្ម័នណាដែលនៅសល់មិនធ្វើប្រតិកម្ម? មានមាឌសល់ប៉ុន្មាន $mL$?
		\item គណនាម៉ាសអ៊ុយរ៉េ $\left(\ce{H2N-CO-NH2}\right)$ ដែលទទួលបាន។\\
		គេឲ្យៈ $V_m=22.4L/mol, N=14, C=12, O=16, H=1$
	\end{enumerate}
	\item {\color{khtug}\kml (១០ ពិន្ទុ)} ក្នុងកែវមួយគេដាក់គ្រាប់ $\ce{Zn}$ ទៅក្នុងអាសុីតក្លរីឌ្រិចខាប់ល្មម នោះប្រតិកម្មពុះកញ្រោល។
	\begin{enumerate}[k]
		\item ចូរសរសេរមីការតាងប្រតិកម្មគីមីខាងលើ។
		\item តើគេត្រូវធ្វើដូចម្តេចដើម្បីបន្ថយល្បឿនប្រតិកម្មគីមីខាងលើ?
		\item ហេតុអ្វីបានជាទង្គិចប្រសិទ្ធរវាងម៉ូលេគុល និងម៉ូលេគុលត្រូវការចាំបាច់ក្នុងប្រតិកម្មគីមីខាងលើនេះ?
	\end{enumerate}
	\item {\color{khtug}\kml (១០ ពិន្ទុ)} គេលាយល្បាយ $0.28mol$ នៃអាសុីតភ្លុយអរីឌ្រិច $\left(\ce{HF}\right)$ និង $0.14mol$ នៃសូដ្យូមអុីដ្រុកសុីត $\left(\ce{NaOH}\right)$ ចំនួន $1L$។
	\begin{enumerate}[k]
		\item គណនាកំហាប់ប្រភេទគីមីនៅពេលលំនឹង។
		\item គណនា $\left[\ce{H3O^{+}}\right]$ និង $pH$ នៃល្បាយសូលុយស្យុងទទួលបាន។
		\item តើសូលុយស្យុងខាងលើមានលក្ខណៈតំប៉ុងដែរឬទេ? ព្រោះអ្វី?
		គេឲ្យៈ $K_{a}=6.7\times10^{-4}, \log6.7=0.82$ 
	\end{enumerate}
	\item {\color{khtug}\kml (១០ ពិន្ទុ)} ម៉ូណូអាមីន $B$ មួយមានរូបមន្ត $\ce{C_{n}H_{2n+3} N}$ បង្ករដោយ $C=2.25mol, H=6.75mol$ និង $N=0.75mol$។
	\begin{enumerate}[k]
		\item ចូរកំណត់រូបមន្តដុលនៃម៉ូណូអាមីន $B$ ខាងលើ។
		\item សរសេររូបមន្តស្ទើលាតនៃម៉ូណូអាមីន $B$ ខាងលើដែលអាចមាន ព្រមទាំងឲ្យឈ្មោះផង។ គេឲ្យៈ $C=12, H=1, N=14$
	\end{enumerate}
\end{enumerate}
\end{document}