\documentclass[12pt, a4paper]{article}
\usepackage[landscape, top=0.5cm, left=1cm, bottom=1.5cm, right=1.5cm]{geometry}
%%import package named hightest
\usepackage{hightest}
\usepackage{comment}
\usepackage[version=3]{mhchem} % Package for chemical equation typesetting
\setlength{\columnseprule}{1pt}
\def\columnseprulecolor{\color{magenta}}
%\usepackage{mathpazo}% change math font
%\usepackage[no-math]{fontspec}% font specfication
\header{រៀនគណិតវិទ្យាទាំងអស់គ្នា}{ គីមីវិទ្យា}{១០/០៣/២០១៨}
\footer{រៀបរៀង និងបង្រៀនដោយ ស៊ុំ សំអុន}{ទំព័រ \thepage}{០៩៦ ៩៤០ ៥៨៤០}
\everymath{\protect\displaystyle\protect\color{black}}
\begin{document}
	\begin{center}
		\sffamily\color{black}
		សូលុយស្យុងទឹក និង $pH$(រូបមន្តសំខាន់ៗ)
	\end{center}
\maketitle
\begin{enumerate}[m, 2]
	\item សមីការស្វ័យអ៊ីយ៉ុងកម្មនៃទឹក
	\begin{equation}
%	\ce{H2O(aq)<=>H3O^+(aq)+OH^-(aq)}
	\end{equation}
	\begin{center}
		To be continued\\
		\sffamily\color{black}
		សូមសំណាងល្អ!
	\end{center}\newpage
\end{enumerate}
	
\end{document}