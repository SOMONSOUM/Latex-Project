\documentclass{officialexam} 
\begin{document}
	{\maketitle}
	\borderline{ផ្នែក៖ សំនួរ}
		\begin{enumerate}[k,2]
			\item តើចរន្តឆ្លាស់ផ្តល់ផលអ្វីខ្លះ?
			\item ចូររៀបរាប់អំពីវគ្គទាំងបួននៃដំណើរការរបស់ម៉ាស៊ីន?
		\end{enumerate}
		\borderline{ផ្នែក៖ លំហាត់}
	\begin{enumerate}[I]
		\item ធុងមួយមានមាឌ $V=0.300 m^3$ ។ ក្នុងធុងមានផ្ទុកឧស្ម័នអេល្យូម $2 mol$ នៅសីតុណ្មភាព $27^\circ C$ ។\\ គេសន្មតអេល្យូមជាឧស្ម័នបរិសុទ្ធ ។
		\begin{enumerate}[k]
			\item គណនាតម្លៃមធ្យមនៃថាមពលស៊ីនេទិចរបស់ម៉ូលេគុលនីមួយៗ 
			\item គណនាថាមពលស៊ីនេទិចសរុបនៃម៉ូលេគុលទាំងអស់ ។\\ គេឲ្យ៖ ថេរបុលស្មាន់ $k_B=1.38\times10^{-23} J/K$ និង ថេរសកលនៃឧស្ម័ន $R=8.31 J/mol.K$
		\end{enumerate}
		\item ឧស្ម័នបរិសុទ្ធ $2$ ម៉ូល មានសីតុណ្ហភាពថេរ $27^\circ C$ ក្នុងរយៈពេលបម្រែបម្រួលមាឌពី $3 L$ ទៅ $6 L$ ។
		\begin{enumerate}[k]
			\item គណនាកម្មន្តដែលបំពេញដោយឧស្ម័នក្នុងរយៈពេលបម្រែបម្រួលមាឌនេះ ។
			\item គណនាបម្រែបម្រួលថាមពលក្នុងរបស់ឧស្ម័ន ។\\
			 គេឲ្យ៖ $\ln3=1.1;~\ln2=0.7;~\ln6=1.7$
		\end{enumerate}
		\item គេដាក់ឧស្ម័នបរិសុទ្ធមួយក្នុងស៊ីឡាំងមួយ ដែលមានមុខកាត់ $A=500 cm^2$ និងបិទជិតដោយគំរបខាងលើជាពីស្តុងដែលអាចចល័តបាន ។ គេផ្តល់កម្តៅបន្តិចម្ដងៗ ឲ្យទៅឧស្ម័ននោះដោយរក្សាសម្ពាធ $P=10^5 Pa$ ឲ្យនៅដដែល ហើយពីស្តុងផ្លាស់ទីឡើងបាន $10 cm$ ។
		\begin{enumerate}[k]
			\item តើប្រព័ន្ធនៃឧស្ម័ននេះរងនូវបម្លែងតាមលំនាំអ្វី?
			\item គណនាកម្មន្ត ដែលបានបំពេញក្នុងរយៈពេលនៃបម្លែងនេះ ។
			\item ប្រសិនបើក្នុងរយៈពេលបម្លែងនេះគេប្រើកម្តៅអស់ $600 J$ គណនាបម្រែបម្រួលថាមពលក្នុងរបស់ឧស្ម័ន ។
		\end{enumerate}
		\item ម៉ាស៊ីនម៉ាស៊ូតនៃរថយន្តមួយដែលមានទិន្នផលកម្តៅ $0.35$ ហើយវាស្រូបបរិមាណកម្តៅ $5\times10^6 J$ ។
		\begin{enumerate}[k]
			\item គណនាកម្មន្តមេកានិចដែលបានពីពីស្តុង ។
			\item គណនាបរិមាណកម្តៅដែលភាយចេញទៅបរិយាកាស ។
			\item គណនាកម្មន្តបានការ បើគេដឹងថា ទិន្នផលនៃគ្រឿងបញ្ជួនស្មើនឹង $0.85$ ។
		\end{enumerate}
		\item បម្លាស់ទីនៃរលកស៊ីនុយសូអុីតមួយមានសមីការ $y=0.50\sin(0.50x-0.25t)(m)$ ។\\
		គណនាអំព្លីទុតនៃរលក ចំនួនរលក ជំហានរលក ខួបនៃរលក និងល្បឿនដំណាលនៃរលក ។
		\item ប្រូតុងមួយមានម៉ាស $m=1.67\times10^{-27}Kg$ ធ្វើចលនាក្នុងដែនម៉ាញេទិចឯកសណ្ឋាន $\overleftrightarrow{B}$ ដែលមាន $B=0.250T$ ដោយវ៉ិចទ័រល្បឿនកែងនឹងវ៉ិចទ័រដែនម៉ាញេទិចឯងសណ្ឋាន $\overleftrightarrow{B}$ ហើយមានតម្លៃ $3.50\times10^6 m.s^{-1}$ ។ គណនាកាំកំណោងដែលគូសបាន ។\\ គេឲ្យ៖  $e=1.6\times10^{-19} C$
		\item ទម្រអង្គធាតុចម្លងពីររាងជាស៊ីឡាំងបានដាក់ឲ្យស្របគ្នាក្នុងប្លង់ដេកដែលចុងទាំងពីររបស់វាភ្ជាប់គ្នាដោយរេស៊ីស្តង់ $R=14.1 \Omega$ ទម្រទាំងពីរនៅឃ្លាតគ្នាចម្ងាយ $0.5 m$ ។ របារលោហៈ $AB$ មួយដាក់ឲ្យកែងលើទម្រទាំងពីរ។ ប្លង់ទម្រកែងនឹងដែនម៉ាញេទិចឯកសណ្ឋានមានអាំងឌុចស្យុង $B=0.80 T$ ។ គេទាញរបារលោហៈ $AB$ ឲ្យផ្លាស់ទីលើទម្រទាំងពីរដោយល្បឿន $v$ គេទទួលបានអាំងតង់ស៊ីតេចរន្តអគ្គីសនី $I=0.141A$ ។ 
		\begin{enumerate}[k]
			\item ចូរគូសរូបញ្ជាក់ ។
			\item គណនាកម្លាំងអគ្គិសនីចលករអាំងឌ្វី​ និងល្បឿនរបស់របារលោហៈ $AB$ ។​\\​(ដោយគេមិនគិតកកិតរវាងទម្រ និងរបារ​ ហើយរបារ និងទម្រមានរេសីុស្តង់អាចចោលបាន)
		\end{enumerate}
		\item បូប៊ីនមួយមានរេស៊ីស្តង់ $R=6\Omega$ និងអាំងឌុចតង់ $L$ ។
		\begin{enumerate}[k]
			\item គណនាអាំងឌុចតង់ $L$ បើថេរពេលនៃសៀគ្វីមានតម្លៃ $\tau=2 ms$
			\item បូប៊ីនមានប្រវែង $30 cm$ មានចំនួនស្ពៀ $1000$។ គណនាអង្គត់ផ្ចិតនៃបូប៊ីននេះ ។
			\item គេធ្វើឲ្យចរន្តប្រែប្រួល $i=3t-2 (A)$ ឆ្លងកាត់បូប៊ីន។ កំណត់កន្សោមតង់ស្យុងរវាងគោលទាំងពីរនៃបូប៊ីននេះ ។\\
			គេឲ្យ៖  $\mu_0=4\pi\times10^{-7}T.m.A^{-1}$ និង $\pi^2=10$ ។
		\end{enumerate}
		\item នៅមជ្ឈមណ្ឌលចែកចាយថាមពលអគ្គិសនីមួយទទួលតង់ស្យុងប្រសិទ្ធ $2400V$ និងអនុភាពអគ្គិសនី $360kW$ ។\\ គេបានប្រើត្រង់ស្វូរម៉ាទ័រមួយដោយយកតង់ស្យុងចេញ $220V$ ដើម្បីប្រើប្រាស់ ។ ត្រង់ស្វូរម៉ាទ័រមានទិន្នផល $90\%$ និងមានចំនួនស្ពៀនៅរបុំបឋម $2400$ស្ពៀរ ។
		\begin{enumerate}[k]
			\item កំណត់ប្រភេទនៃត្រង់ស្វូ និងគូសគំនូសតាងនិមិត្តសញ្ញាត្រង់ស្វូ ។
			\item គណនាអាំងតង់ស៊ីតេចរន្តរបុំបឋម និងក្នុងរបុំមធ្យម យកកត្តាអនុភាព $k=1$ ។
			\item គណនាចំនួនស្ពៀរបុំមធ្យមនៃត្រង់ស្វូ ។ 
		\end{enumerate}
		\item សូលេណូអ៊ីតមួយមានប្រវែង $l=0.5m$ មានចំនួនស្ពៀរ $N=1000$ មានកំា $R=10cm$ ។
		\begin{enumerate}[k]
			\item គណនាអាំងឌុចតង់នៃបួប៊ីន ។
			\item គណនារេស៊ីស្តង់បូប៊ីន បើខ្សែចម្លងមានរេស៊ីស្ទីវីតេ $\rho=1.6\times10^{-8}\Omega m$ និងមានមុខកាត់ $A_w=1mm^2$ ។
			\item គណនាអាំងឌុចស្យុងម៉ាញេទិចក្នុងបូប៊ីនពេលមានចរន្តឆ្លងកាត់ដោយចរន្ត $I=1A$ ។
			\item គណនាផលសងប៉ូតងស្យែល និងថាមពលម៉ាញេទិចនៃបូប៊ីន ។ គេឲ្យ៖ $\mu_0=4\pi\times10^{-7}T.m.A^{-1}$ និង $\pi^2=10$ ។ 
		\end{enumerate}
		\item បូប៊ីនមួយមានអាំងឌុចតង់ $L=0.02H$ បានស្តុកទុកថាមពលអេឡិចត្រូម៉ាញេទិច $E_L$ ហើយឆ្លងកាត់ដោយអាំងតង់ស៊ីតេចរន្ត $i=0.224A$ ។ គណនាថាមពលអេឡិចត្រូម៉ាញេទិចនៃបូប៊ីននេះ ។
	\end{enumerate}
\borderline{\bigg[សូមសំណាងល្អគ្រប់ៗគ្នា!\bigg]}\\
{\color{white}.}\dotfill\\
{\color{white}.}\dotfill\\
{\color{white}.}\dotfill
\newpage
{\maketitle}
\borderline{ផ្នែក៖ សំនួរ}
\begin{enumerate}[k]
	\item ចូរកំណត់ទិសដៅដែនម៉ាញេទិចក្នុងករណីដូចខាងក្រោម៖
	\begin{enumerate}[a,3]
		\item ករណីចរន្តត្រង់
		\item ករណីចរន្តវង់
		\item ករណីចរន្តឆ្លងកាត់បូប៊ីន​។
	\end{enumerate}
	\item ចូរសរសេររូបមន្តនៃតម្លៃអាំងឌុចស្យុងម៉ាញេទិចដែលកើតមានក្នុងករណីដូចខាងក្រោម៖
	\begin{enumerate}[a,4]
		\item ចរន្តត្រង់
		\item ចរន្តវង់
		\item បូប៊ីនសំប៉ែត
		\item សូលេណូអ៊ីត
	\end{enumerate}
	\item តើអ្វីខ្លះជាប្រភពនៃដែនម៉ាញេទិច? ហើយវាត្រូវបានគិតជាអ្វី? 
\end{enumerate}
\borderline{ផ្នែក៖ លំហាត់}
\begin{enumerate}[I]
	\item នៅក្នុងធុងមួយមានមាឌ $2.00mL$ មានឧស្ម័នដែលមានម៉ាស $50mg$ និងសម្ពាធ $100kPa$ ។ \\ម៉ាសម៉ូលេគុលឧស្ម័ននីមួយៗគឺ $8.0\times10^{-26}kg$ ។ គេឲ្យ៖  $k_B=1.38\times10^{-23}J/K$ ។
	\begin{enumerate}[k]
		\item គណនាចំនួនម៉ូលេគុលនៃឧស្ម័ន ។
		\item គណនាតម្លៃថាមពលស៊ីនេទិចមធ្យមនៃម៉ូលេគុលនីមួយៗ ។
		\item គណនាថាមពលស៊ីនេទិចសរុបនៃម៉ូលេគុលឧស្ម័ននៅក្នុងធុង ។
	\end{enumerate}
	\item គណនាតម្លៃល្បឿនប្ញសការេនៃការេល្បឿនមធ្យមនៃម៉ូលេគុលឧស្ម័នអុកស៊ីសែននៅសីតុណ្ហភាព $200^\circ C$ ។ \\គេឧ្យម៉ាសម៉ូលអុកស៊ីសែន $32g/mol$ និង $R=8.31J/mol.K$ ។
	\item ពីស្តុងក្នុងស៊ីឡាំងមួយមានមុខកាត់ $0.010 m^2$ ក្រោមសម្ពាធថេរ $7.5\times10^5 Pa$ ។ ពីស្តុងផ្លាស់ទីបានប្រវែង $0.040 m$ ។\\ គណនាកម្មន្តដែលបំពេញដោយពីស្តុង ។
	\item គណនាបម្រែបម្រួលថាមពលក្នុងនៃប្រព័ន្ធ ក្នុងករណីនីមួយខាងក្រោម៖ 
	\begin{enumerate}[k]
		\item ប្រព័ន្ធស្រូបកម្តៅ $500 cal$ និងបញ្ចេញកម្មន្ត $400 J$ ។
		\item ប្រព័ន្ធស្រូបកម្តៅ $300 cal$ និងរងនូវកម្មន្ត $420 J$ ។
		\item ប្រព័ន្ធឧស្ម័នមានមាឌថេរ និងបំភាយកម្តៅអស់ $1200 cal$ ។ គេឲ្យ៖ $1 cal =4.2 J$ ។
		\item ប្រព័ន្ធឧស្ម័នរងនូវលំនាំអាដ្យាបាទិចរហូតដល់មាឌនៅត្រឹម $\frac{1}{3}$ នៃមាឌដើម $V_0$ ហើយប្រើកម្មន្តអស់ $450J$ ទៅលើឧស្ម័ន ។
	\end{enumerate}
		\item ម៉ូទ័រសាំងនៃរថយន្តរេណូលមួយបានទទួលកម្តៅ $2.1\times10^5J/s$ ដើម្បីឲ្យមានបន្ទុះស៊ីឡាំងឥន្ធនៈ ។\\ ម៉ូទ័រនេះបានបញ្ចេញកម្តៅ $1.3\times10^5J/s$ ទៅមជ្ឈដ្ឋានក្រៅ ។
	\begin{enumerate}[k]
		\item គណនាកម្មន្តដែលធ្វើដោយពីស្តុងក្នុងរយៈពេលមួយវិនាទី ។
		\item គណនាទិន្នផលកម្តៅនៃម៉ូទ័រ ។
		\item គេដឹងថាម៉ូទ័រមានទិន្នផលមេកានិច $0.85$ ។ គណនាកម្មន្តដែលភ្លៅម៉ូទ័របានទទួលក្នុងរយៈពេលមួយវិនាទី ។
	\end{enumerate}
	\item \begin{enumerate}[k]
		\item ជារៀងរាល់ព្រឹកម៉ាន់ដេតែងតែមករត់ហាត់ប្រាណតាមបណ្តោយសួនច្បារមួយ ដោយបានបំពេញនូវកម្មន្ត $4.3\times10^5 J$ និងបានបញ្ចេញកម្តៅ $3.8\times10^5 J$ ។ គណនាបម្រែបម្រួលថាមពលក្នុងរបស់ម៉ាន់ដេ ។
		\item បើគាត់ប្តូរពីរត់មកដើរវិញ នោះគាត់បានបញ្ចេញកម្តៅបាន $1.2\times10^5 J$ និងថាមពលក្នុងបានថយចុះអស់ $2.6\times10^5 J$ ។ \\តើក្នុងពេលដើរម៉ាន់ដេធ្វើបានកម្មន្តប៉ុន្មានស៊ូល?
	\end{enumerate}
	\item ម៉ូទ័រសាំងនៃរថយន្តរេណូលមួយបានទទួលកម្តៅ $2.1\times10^5J/s$ ដើម្បីឲ្យមានបន្ទុះស៊ីឡាំងឥន្ធនៈ ។\\ម៉ូទ័រនេះបានបញ្ចេញកម្តៅ $1.3\times10^5J/s$ ទៅមជ្ឈដ្ឋានក្រៅ ។
	\begin{enumerate}[k]
		\item គណនាកម្មន្តដែលធ្វើដោយពីស្តុងក្នុងរយៈពេលមួយវិនាទី ។
		\item គណនាទិន្នផលកម្តៅនៃម៉ូទ័រ ។
		\item គេដឹងថាម៉ូទ័រមានទិន្នផលមេកានិច $0.85$ ។ គណនាកម្មន្តដែលភ្លៅម៉ូទ័របានទទួលក្នុងរយៈពេលមួយវិនាទី ។
	\end{enumerate}
	\item លំញ័រមួយចាប់ផ្តើមដាលពីចំណុច $O$ ដោយខួប $2s$ និងអំព្លីទុត $4cm$ ។
	\begin{enumerate}[k]
		\item សរសេរសមីការរលកត្រង់ $O$ ។
		\item គណនាជំហានរលក ដោយគេដឹងថារលកដាលដោយល្បឿន $5m/s$ ។
		\item សរសេរមីការរលកត្រង់ចំណុចមួយដែលស្ថិតចម្ងាយ $6m$ ពីចំណុច $O$ ។
	\end{enumerate}
	\item លំយោលនៃសៀគ្វីអគ្គីសនីមួយមានប្រកង់ $f=10^5 Hz$ ដែលកើតឡើងដោយកុងដង់សាទ័រដែលមានកាប៉ាស៊ីតេ $C=40pF$ និងអាំងឌុចតង់នៃបូប៊ីន $L$ ។
	\begin{enumerate}[k]
		\item គណនាខួបនៃលំយោលសៀគ្វីនេះ​
		\item គណនាតម្លៃអាំងឌុចតង់នៃបួប៊ីន ។
	\end{enumerate}
	\item \begin{enumerate}[k]
		\item កុងដង់សាទ័រមួយមានកាប៉ាស៊ីតេ $C=1\mu F$ ត្រូវបានផ្ទុកក្រោមតង់ស្យុង $V=E=2V$ ។\\ គណនាថាមពលសន្សំទុកនៃកុងដង់សាទ័រ ។
		\item គេយកកុងដង់សាទ័រដែលផ្ទុករួចនេះ ទៅតភ្ជាប់នឹងបួប៊ីនមួយដែលមានអាំងឌុចតង់ $L=0.1H$ និងមានរេស៊ីស្តង់អាចចោលបាន ។ គណនាអាំងតង់ស៊ីតេនៃចរន្តអតិបរមា $i_m$ ។ 
	\end{enumerate}
	\item គេឲ្យសៀគ្វីដូចរូបដែលមានៈ ជនិតាអ៊ីដេអាល់ដែលមានតង់ស្យុងថេរ $V=12V$ និងបូប៊ីនដែលមានអាំងឌុចតង់ $L=0.4H$ និង\\រេស៊ីស្តង់ $R=16\Omega$ ។ គេបិទកុងតាក់ $K$ ចូរគណនា៖ 
	\begin{enumerate}[k]
		\item អាំងតង់ស៊ីតេចរន្តឆ្លងកាត់សៀគ្វីក្នុងរបបអចិន្ត្រៃយ៍ ។
		\item ថាមពលអេឡិចត្រូម៉ាញេទិចនៃបូប៊ីនក្នុងរបបអចិន្ត្រៃយ៍ ។
		\item ថេរពេលនៃសៀគ្វី $RL$ ។
		\item អាំងតង់ស៊ីតេចរន្តខណៈ $t_1=\tau$ និង $t_2=5\tau$ រួចសង់ខ្សែកោងតាងបម្រែបម្រួល $i=f(i)$ ។
	\end{enumerate}
\end{enumerate}
\end{document}