\documentclass[11pt, a5paper]{article}
%%import package named hightest
\usepackage{hightest}
\usepackage[export]{adjustbox}
\usepackage{wrapfig}
\usepackage{tkz-tab}
%សរសេរគីមីវិទ្យា
%\usepackage[version=3]{mhchem}
%\usepackage{mathpazo}% change math font
%\usepackage[no-math]{fontspec}% font specfication
\everymath{\protect\displaystyle\protect\color{blue}}
\begin{document}
\maketitle
\borderline{\circled{១}~វិញ្ញាសាទី~\circled{១}}
\begin{enumerate}[m]
	 \item គណនាលីមីតខាងក្រោម៖
	 \begin{enumerate}[k,3]
	 	\item $\lim\limits_{x\to0}\frac{\sin34x}{2x}$
	 	\item $\lim\limits_{x\to0}\frac{\sin34x}{2x}$
	 	\item $\lim\limits_{x\to0}\frac{\sin34x}{2x}$
	 \end{enumerate}
 	\item គណនាលីមីត
\end{enumerate}
	{\color{magenta}.}\dotfill\\
	{\color{magenta}.}\dotfill\\
	{\color{magenta}.}\dotfill\\
	{\color{magenta}.}\dotfill\\
	{\color{magenta}.}\dotfill\\
	{\color{magenta}.}\dotfill\\
 	\borderline{សូមសំណាងល្អគ្រប់ៗគ្នា}
\newpage
\maketitle
\borderline{\circled{២}~វិញ្ញាសាទី~\circled{២}}
\begin{enumerate}[m]
	\item សរសេរលំហាត់ទីនេះ
	\begin{enumerate}[k,3]
		\item ចម្លើយទី១
		\item ចម្លើយទី២
		\item ចម្លើយទី៣
	\end{enumerate}
	\item សរសេរលំហាត់ទីនេះ
	\begin{enumerate}[k,3]
		\item ចម្លើយទី១
		\item ចម្លើយទី២
		\item ចម្លើយទី៣
	\end{enumerate}
	\item សរសេរលំហាត់ទីនេះ
	\begin{enumerate}[k,3]
		\item ចម្លើយទី១
		\item ចម្លើយទី២
		\item ចម្លើយទី៣
	\end{enumerate}
\end{enumerate}
	{\color{magenta}.}\dotfill\\
	{\color{magenta}.}\dotfill\\
	{\color{magenta}.}\dotfill\\
	{\color{magenta}.}\dotfill\\
	{\color{magenta}.}\dotfill\\
	{\color{magenta}.}\dotfill\\
	\borderline{សូមសំណាងល្អគ្រប់ៗគ្នា}
\end{document}