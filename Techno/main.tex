\documentclass[12pt, a4paper]{article}
%%import package named techno
\usepackage{techno}
\usepackage[export]{adjustbox}
\usepackage{wrapfig}
\usepackage{tkz-tab}
%សរសេរគីមីវិទ្យា
%\usepackage[version=3]{mhchem}
%\usepackage{mathpazo}% change math font
%\usepackage[no-math]{fontspec}% font specfication
\everymath{\protect\displaystyle\protect\color{blue}}
\begin{document}
\maketitle\koc
{\color{blue}\hrulefill}
\begin{enumerate}[m]
	 \item កន្សោម $\mathbf{D_n}=1+2+2^2+2^3+\dots+2^n$ ស្មើនឹង
	 \begin{enumerate}[k,5]
	 	\item $\mathbf{D_n}=2^n-1$
	 	\item $\mathbf{D_n}=2^{n+1}-1$
	 	\item $\mathbf{D_n}=2^n+1$
	 	\item $\mathbf{D_n}=2^{n+1}+1$
	 	\item $\mathbf{D_n}=2^{n+1}$
	 \end{enumerate}
	 {\color{blue}\hrulefill}
 	\item គេឲ្យវ៉ិចទ័របី $\vec{a}=(1,1,1), \vec{b}=(1,-2,-1), \vec{c}=(-1,-2,1)$ ។ ចូរកំណត់រក $\mathbf{E}=\left(\vec{a}\times\vec{b}\right)\cdot\vec{c}$
 	\begin{enumerate}[k,5]
 		\item $\mathbf{E}=-6$
 		\item $\mathbf{E}=8$
 		\item $\mathbf{E}=-8$
 		\item $\mathbf{E}=6$
 		\item ចម្លើយផ្សេង
 	\end{enumerate}
 	{\color{blue}\hrulefill}
 	\item យកកន្សោម $\mathbf{E}=\frac{\sin^6x+\cos^6x-1}{\sin^4x+\cos^4x-1}$ ។ នោះ $\mathbf{E}$ ស្មើនឹង 
 	\begin{enumerate}[k,5]
 		\item $-\frac{2}{3}$
 		\item $-\frac{3}{2}$
 		\item $\frac{2}{3}$
 		\item $\frac{3}{2}$
 		\item ចម្លើយផ្សេង
 	\end{enumerate}
 	{\color{blue}\hrulefill}
 	\item គេយក $E$ ជាសំណុំចម្លើយទាំងអស់របស់សមីការឌីផេរ៉ង់ស្យែល $y''+4y'+13y=0$ ។ ក្នុងចំណោមអនុគមន៍ខាងក្រោមនេះ តើមួយណាជាធាតុរបស់ $E$ ?
 	\begin{enumerate}[k,3]
 		\item $y=e^{2t}\left(\cos3t+4\sin3t\right)$
 		\item $y=e^{-2t}\cos4t$
 		\item $y=e^{-2t}\left(\cos3t+4\sin3t\right)$
 		\item $y=e^{2t}\cos4t$
 		\item $y=e^{-3t}\left(\cos3t+4\sin3t\right)$
 	\end{enumerate}
 	{\color{blue}\hrulefill}
 	\item ដេរីវេនៃអនុគមន៍ $f(x)=\ln\left(x+\sqrt{1+x^2}\right)$ គឺ
 	\begin{enumerate}[k,5]
 		\item $\frac{x}{\sqrt{1-x^2}}$
 		\item $\frac{1}{\sqrt{1+x^2}}$
 		\item $\frac{x}{\sqrt{1+x^2}}$
 		\item $\frac{x^2}{\sqrt{1+x^2}}$
 		\item ចម្លើយផ្សេង
 	\end{enumerate}
 	{\color{blue}\hrulefill}
 	\item កន្សោម $\mathbf{E}=\left(\frac{\sqrt{2}}{2}+\frac{\sqrt{2}}{2}i\right)^{2015}$ ស្មើនឹង
 	\begin{enumerate}[k,3]
 		\item $\mathbf{E}=-\frac{\sqrt{2}}{2}+\frac{\sqrt{2}}{2}i$
 		\item $\mathbf{E}= \frac{\sqrt{2}}{2}+\frac{\sqrt{2}}{2}i$
 		\item $\mathbf{E}=\frac{\sqrt{2}}{2}-\frac{\sqrt{2}}{2}i$
 		\item $\mathbf{E}=-\frac{\sqrt{2}}{2}-\frac{\sqrt{2}}{2}i$
 		\item ចម្លើយផ្សេង
 	\end{enumerate}
 	{\color{blue}\hrulefill}
 	\item ចូរគណនា $\lim\limits_{x\to\frac{\pi}{6}}\frac{2\sin^2x+\sin x-1}{2\sin^2x-3\sin x+1}$ ។
 	\begin{enumerate}[k,5]
 		\item $-1$
 		\item $3$
 		\item $1$
 		\item $-3$
 		\item ចម្លើយផ្សេង
 	\end{enumerate}
 	{\color{blue}\hrulefill}\\
\end{enumerate}
\makeads
\newpage
\maketitle\koc
{\color{blue}\hrulefill}
\begin{enumerate}[m]
	\item ដោយដឹងថា $2$ និង $3+i$ ជាប្ញសនៃសមីការ $az^3+bz^2+cz+d=0 $ ដែល $a,b,c,d\in\mathbb{R}$ នោះប្ញសមួយទៀតនៃសមីការនេះគឺ
	\begin{enumerate}[k,5]
		\item $-2$
		\item $-3+i$
		\item $-3-i$
		\item $3-i$
		\item ចម្លើយផ្សេង
	\end{enumerate}
	{\color{blue}\hrulefill}
	\item គណនាលីមីត $\lim\limits_{x\to0}\frac{x^2+2\sin x+e^x-1}{x^9+x+1-\cos x}$
	\begin{enumerate}[k,5]
		\item $0$
		\item $1$
		\item $2$
		\item $3$
		\item ចម្លើយផ្សេង
	\end{enumerate}
	{\color{blue}\hrulefill}
	\item គណនាតម្លៃនៃកន្សោម $\frac{1}{\alpha}+\frac{1}{\beta}$ ដោយដឹងថា $\alpha$ និង $\beta$ ជាប្ញសនៃសមីការ $3x^2-14x+17=0$ 
	\begin{enumerate}[k,5]
		\item $-14$
		\item $-7$
		\item $-2$
		\item $2$
		\item $7$
	\end{enumerate}
	{\color{blue}\hrulefill}
	\item យក $x$ ជាមេគុណនៃឯកធា $a^3bd^7$ និង $y$ ជាចំនួននៃឯកធាទាំងអស់នៅក្នុងពហុធាដឺក្រេទី១១ $\left(a+b+c+d\right)^{11}$ ។ គេបាន
	\begin{enumerate}[k,3]
		\item $\left(x=1333,y=365\right)$
		\item $\left(x=1234,y=363\right)$
		\item $\left(x=1365,y=366\right)$
		\item $\left(x=1236,y=367\right)$
		\item $\left(x=1320,y=364\right)$
	\end{enumerate}
	{\color{blue}\hrulefill}
	\item សំណុំនៃប្ញសទាំងអស់របស់វិសមីការ $\ln x\leq\frac{3\ln x-2}{\ln x}$ គឺ
	\begin{enumerate}[k,3]
		\item $\left(-\infty,1\right)\cup\left[e,e^2\right]$
		\item $\left(0,1\right)\cup\left[e,e^2\right]$
		\item $\left(0,1\right)\cup\left(e,e^2\right)$
		\item $\left[e,e^2\right]$
		\item ចម្លើយផ្សេង
	\end{enumerate}
	{\color{blue}\hrulefill}
	\item ចូរគណនា $\lim\limits_{x\to0}\frac{e^{1-\cos^2x}-\cos x}{\sin^2x}$ ។
	\begin{enumerate}[k,5]
		\item $\frac{3}{2}$
		\item $\frac{2}{3}$
		\item $-\frac{2}{3}$
		\item $-\frac{3}{2}$
		\item ចម្លើយផ្សេង
	\end{enumerate}
	{\color{blue}\hrulefill}
	\item យក $f(x)=e^{-3x}\left(9\sin9x-3\cos9x\right)$ ជាអនុគមន៍ និង $f'(x)$ ជាដេរីវេនៃ $f(x)$ ។ គេបាន
	\begin{enumerate}[k,3]
		\item $f'(x)=90e^{-3x}\cos8x$
		\item $f'(x)=90e^{3x}\cos9x$
		\item $f'(x)=90e^{-3x}\cos9x$
		\item $f'(x)=90e^{-3x}\cos8x$
		\item ចម្លើយផ្សេង
	\end{enumerate}
	{\color{blue}\hrulefill}
\end{enumerate}
\makeads
\newpage
\maketitle\koc
{\color{blue}\hrulefill}
\begin{enumerate}[m]
	\item រកមេគុណនៃ $x^2$ ក្នុងការពន្លាតកន្សោម $\left(x^3+\frac{1}{x^2}\right)^9$ គឺ
	\begin{enumerate}[k,5]
		\item $0$
		\item $1$
		\item $124$
		\item $126$
		\item ចម្លើយផ្សេង
	\end{enumerate}
	{\color{blue}\hrulefill}
	\item គណនាតម្លៃនៃកន្សោម $8\sin^4\theta+4\cos\left(2\theta\right)-\cos\left(4\theta\right),~~\theta\in\mathbb{R}$
	\begin{enumerate}[k,5]
		\item $-1$
		\item $0$
		\item $1$
		\item $2$
		\item $3$
	\end{enumerate}
	{\color{blue}\hrulefill}
	\item គេដឹងថា $\frac{2x+1}{\left(x+2\right)\left(x+1\right)^2}=\frac{a}{x+2}+\frac{b}{x+1}+\frac{c}{\left(x+1\right)^2}$ ។ នោះគេបាន
	\begin{enumerate}[k,3]
		\item $a=3,b=-3,c=-1$
		\item $a=-3,b=3,c=-1$
		\item $a=-1,b=3,c=-3$
		\item $a=-3,b=-1,c=3$
		\item ចម្លើយផ្សេង
	\end{enumerate}
	{\color{blue}\hrulefill}
	\item ក្រឡាផ្ទៃនៃដែនប្លង់ដែលខ័ណ្ឌដោយខ្សែកោងតាង $y=x^2$ និង $y=4$ ស្មើនឹង 
	\begin{enumerate}[k,5]
		\item $\frac{32}{3}$
		\item $\frac{31}{3}$
		\item $\frac{37}{3}$
		\item $\frac{35}{3}$
		\item ចម្លើយផ្សេង
	\end{enumerate}
	{\color{blue}\hrulefill}
	\item ចូររកតម្លៃនៃ $\lim\limits_{x\to\infty}\left(\frac{x^2+1}{x^2-2}\right)^{x^{2}}$ ។
	\begin{enumerate}[k,5]
		\item $e^{-2}$
		\item $e^{-3}$
		\item $e^{3}$
		\item $e^{2}$
		\item ចម្លើយផ្សេង
	\end{enumerate}
	{\color{blue}\hrulefill}
	\item យក $f(x)=\int_{0}^{x^2}\frac{\sin t}{t}dt$ ។ ចូរគណនាដេរីវេ $f'(x)$ នៃ $f(x)$ ។
	\begin{enumerate}[k,3]
		\item $f'(x)=\frac{\sin\left(x^2\right)}{x^2}$
		\item $f'(x)=\frac{\sin\left(x\right)}{x}$
		\item $f'(x)=\frac{2\sin\left(x^2\right)}{x}$
		\item $f'(x)=\frac{2\sin\left(x\right)}{x}$
		\item ចម្លើយផ្សេង
	\end{enumerate}
	{\color{blue}\hrulefill}
	\item ចូរគណនាអាំងតេក្រាល $\mathbf{I}=\int_{0}^{2}\sqrt{4-x^2}dx$ ។
	\begin{enumerate}[k,5]
		\item $\mathbf{I}=4\pi$
		\item $\mathbf{I}=3\pi$
		\item $\mathbf{I}=2\pi$
		\item $\mathbf{I}=\pi$
		\item ចម្លើយផ្សេង
	\end{enumerate}
	{\color{blue}\hrulefill}
\end{enumerate}
\makeads
\newpage
\maketitle\koc
{\color{blue}\hrulefill}
\begin{enumerate}[m]
	\item ចូររកតម្លៃអប្សរមានៃ $y=x^4+x^2+2+\frac{4}{x^4+x^2+2}$ ។
	\begin{enumerate}[k,5]
		\item $2$
		\item $3$
		\item $5$
		\item $6$
		\item $4$
	\end{enumerate}
	{\color{blue}\hrulefill}
	\item សំណុំនៃប្ញសទាំងអស់របស់សមីការ $\left(x-7\right)\left(x-5\right)\left(x+4\right)\left(x+6\right)=608$ គឺ
	\begin{enumerate}[k,2]
		\item $\mathbf{S}=\left\lbrace\left(1\pm\sqrt{19}\right)/2,\left(1\pm\sqrt{234}\right)/2\right\rbrace$
		\item $\mathbf{S}=\left\lbrace1\pm\sqrt{17},1\pm\sqrt{233}\right\rbrace$
		\item $\mathbf{S}=\left\lbrace\left(1\pm\sqrt{17}\right)/2,\left(1\pm\sqrt{233}\right)/2\right\rbrace$
		\item $\mathbf{S}=\left\lbrace1\pm\sqrt{19},1\pm\sqrt{234}\right\rbrace$
		\item ចម្លើយផ្សេង
	\end{enumerate}
	{\color{blue}\hrulefill}
	\item បើ $x_0>0,x_n=\frac{2014}{2015}x_{n-1}+\frac{1}{x^{2014}_{n-1}}, n=1,2,3,\dots$ នោះ លីមីតនៃស្វ៊ីត $x_n$ ស្មើនឹង\\
			\begin{enumerate}[k,5]
				\item $\sqrt[2014]{2014}$
				\item $\sqrt[2014]{2015}$
				\item $\sqrt[2015]{2015}$
				\item $\sqrt[2015]{2014}$
				\item ចម្លើយផ្សេង
			\end{enumerate}
	{\color{blue}\hrulefill}
	\item យក $\mathbf{S_n}=\frac{81}{10^n}\left(8+88+\dots+ 88\dots88\right)$ និង $\mathbf{S}=\lim\limits_{x\to+\infty}\mathbf{S_n}$ ។ គេបាន
	\begin{enumerate}[k,5]
		\item $\mathbf{S}=72$
		\item $\mathbf{S}=80$
		\item $\mathbf{S}=81$
		\item $\mathbf{S}=90$
		\item ចម្លើយផ្សេង
	\end{enumerate}
	{\color{blue}\hrulefill}
	\item នៅក្នុងសំណុំនៃចំនួនគត់ធំជាង $1$ ចូររកចំនួននៃប្ញសទាំងអស់របស់សមីការ $a+b+c+d=16$ ។
	\begin{enumerate}[k,5]
		\item $152$
		\item $165$
		\item $173$
		\item $184$
		\item ចម្លើយផ្សេង
	\end{enumerate}
	{\color{blue}\hrulefill}
	\item តម្លៃនៃកន្សោម $\sqrt[3]{6+\sqrt[3]{6+\sqrt[3]{6+\sqrt[3]{6+\dots}}}}$ ស្មើនឹង
	\begin{enumerate}[k,5]
		\item $3$
		\item $2$
		\item $1$
		\item $-2$
		\item ចម្លើយផ្សេង
	\end{enumerate}
	{\color{blue}\hrulefill}
	\item យក $f(x)=\frac{x+\sqrt{3}}{1-x\sqrt{3}}$ និង $f_n(x)=f\left(\dots f\left(f\left(x\right)\right)\dots\right)$ ដែល $f$ មានចំនួន $n$ ដង។ គេបាន
	\begin{enumerate}[k,3]
		\item $f_{2015}\left(x\right)=x$
		\item $f_{2015}\left(x\right)=\frac{x+\sqrt{3}}{1-x\sqrt{3}}$
		\item $f_{2015}\left(x\right)=\frac{x-\sqrt{3}}{x\sqrt{3}+1}$
		\item $f_{2015}\left(x\right)=\frac{x+\sqrt{3}}{1+\sqrt{3}x}$
		\item ចម្លើយផ្សេង
	\end{enumerate}
	{\color{blue}\hrulefill}
\end{enumerate}
\makeads
\newpage
\maketitle\koc
{\color{blue}\hrulefill}
\begin{enumerate}[m]
	\item បើ $f(x)=5^x$ នោះគេបានដេរីវេនៃ $f$ គឺ $f'(x)$ ស្មើនឹង
	\begin{enumerate}[k,5]
		\item $5^x$
		\item $x5^x$
		\item $5^x\ln5$
		\item $5e^x$
		\item ចម្លើយផ្សេង
	\end{enumerate}
	{\color{blue}\hrulefill}
	\item តាង $0\leq\alpha,\beta\leq\frac{\pi}{4}$ ។ បើ $\cos\left(\alpha+\beta\right)=\frac{4}{5}$ និង $\sin\left(\alpha-\beta\right)=\frac{5}{13}$ ចូរគណនាតម្លៃនៃ $\tan\left(2\alpha\right)$ ។
	\begin{enumerate}[k,5]
		\item $0$
		\item $\frac{56}{33}$
		\item $\frac{33}{56}$
		\item $-\frac{33}{56}$
		\item $-\frac{56}{33}$
	\end{enumerate}
	{\color{blue}\hrulefill}
	\item រកប្ញសមួយនៃសមីការ $3x^4+4x^3-x^2-5x+2=0$
	\begin{enumerate}[k,5]
		\item $-1$
		\item $\frac{1}{2}$
		\item $\frac{2}{3}$
		\item $\frac{3}{4}$
		\item $\frac{4}{5}$
	\end{enumerate}
	{\color{blue}\hrulefill}
	\item យក $u_1>0, u_{n+1}=\sqrt{u_n+u_{n-1}+\dots+u_2+u_1},n=1,2,3,\dots$ ។ នោះលីមីតនៃស្វ៊ីត $\frac{u_n}{n}$ ស្មើនឹង
	\begin{enumerate}[k,5]
		\item $4$
		\item $\frac{1}{2}$
		\item $2$
		\item $\frac{1}{4}$
		\item ចម្លើយផ្សេង
	\end{enumerate}
		{\color{blue}\hrulefill}
	\item $f(x)$ ជាអនុគមន៍ពិតផ្ទៀងផ្ទាត់ $f(x)+f\left(\frac{x-1}{x}\right)=1+x$ ។ ចូរកំណត់រក $f(x)$ ។
	\begin{enumerate}[k,3]
		\item $f(x)=\frac{x^3+x^2-1}{2x(x+1)}$
		\item $f(x)=\frac{x^3-x^2-1}{2x(x+1)}$
		\item $f(x)=\frac{x^3-x^2-1}{2x(x-1)}$
		\item $f(x)=\frac{x^3+x^2+1}{2x(x+1)}$
		\item $f(x)=\frac{x^3+x^2+1}{x(x+1)}$
	\end{enumerate}
	{\color{blue}\hrulefill}
	\item តម្លៃនៃកន្សោម $\sin\left(\frac{\pi}{722}\right)\sin\left(\frac{2\pi}{722}\right)\dots\sin\left(\frac{360\pi}{722}\right)$ ស្មើនឹង
	\begin{enumerate}[k,5]
		\item $\frac{17}{2^{361}}$
		\item $\frac{17\sqrt{3}}{2^{361}}$
		\item $\frac{19\sqrt{3}}{2^{360}}$
		\item $\frac{19}{2^{360}}$
		\item ចម្លើយផ្សេង
	\end{enumerate}
	{\color{blue}\hrulefill}
	\item យក $\mathbf{S}=\lim\limits_{n\to+\infty}\left(\frac{n}{n^4+n^2+1}+\frac{4n}{n^4+4n^2+16}+\dots+\frac{n^3}{n^4+n^4+n^4}\right)$ ។ គេបាន
	\begin{enumerate}[k,3]
		\item $12\mathbf{S}=\pi\sqrt{3}-3\ln3$
		\item $12\mathbf{S}=\pi\sqrt{3}-3\ln2$
		\item $12\mathbf{S}=\pi\sqrt{3}+3\ln3$
		\item $12\mathbf{S}=\pi\sqrt{3}+3\ln2$
		\item ចម្លើយផ្សេង
	\end{enumerate}
	{\color{blue}\hrulefill}
\end{enumerate}
\makeads
\newpage
\maketitle\koc
{\color{blue}\hrulefill}
\begin{enumerate}[m]
	\item ដោយដឹងថា $\left(a_n\right)$ ជាស្វ៊ីតនព្វន្តដែលមានតួ $a_7=6$ និង $a_{10}=10$ ចូរកំណត់តម្លៃនៃ $a_{15}$ ។
	\begin{enumerate}[k,5]
		\item $14$
		\item $15$
		\item $\frac{55}{4}$
		\item $\frac{50}{3}$
		\item ចម្លើយផ្សេង
	\end{enumerate}
	{\color{blue}\hrulefill}
	\item តាង $n\in \mathbb{N}$ ។ រកចំនួនធំបំផុត $n$ ដែលផ្ទៀងផ្ទាត់វិសមាភាព $n^{2000}<5^{3000}$
	\begin{enumerate}[k,5]
		\item $10$
		\item $11$
		\item $12$
		\item $13$
		\item ចម្លើយផ្សេង
	\end{enumerate}
	{\color{blue}\hrulefill}
	\item គេមានកន្សោម $E=\frac{\sin^8x-\cos^8x}{\left(\sin^2x-\cos^2x\right)\left(1-2\sin^2x\cos^2x\right)}$ នោះ $E$ ស្មើនឹង
	\begin{enumerate}[k,5]
		\item $2$
		\item $-2$
		\item $-1$
		\item $1$
		\item ចម្លើយផ្សេង
	\end{enumerate}
	{\color{blue}\hrulefill}
	\item យក $f(x)=\int_{-x^2}^{x^2}e^{t^2}dt$ ។ ចូរគណនាដេរីវេ $f'(x)$ នៃ $f(x)$ ។
	\begin{enumerate}[k,5]
		\item $f'(x)=4xe^{x^2}$
		\item $f'(x)=2xe^{x^4}$
		\item $f'(x)=4xe^{x^4}$
		\item $f'(x)=2xe^{x^2}$
		\item ចម្លើយផ្សេង
	\end{enumerate}
	{\color{blue}\hrulefill}
	\item ក្រឡាផ្ទៃនៃដែនខ័ណ្ឌដោយខ្សែកោងតាង $y=-x^2$ និង $y=-x-2$ ស្មើនឹង 
	\begin{enumerate}[k,5]
		\item $\frac{11}{2}$
		\item $\frac{9}{2}$
		\item $\frac{10}{3}$
		\item $\frac{13}{2}$
		\item ចម្លើយផ្សេង
	\end{enumerate}
	{\color{blue}\hrulefill}
	\item ចូរគណនាអាំងតេក្រាល $I=\int_{0}^{2}x^2\sqrt{4-x^2}dx$
	\begin{enumerate}[k,5]
		\item $3\pi$
		\item $2\pi$
		\item $4\pi$
		\item $\frac{\pi}{2}$
		\item $\pi$
	\end{enumerate}
	{\color{blue}\hrulefill}
	\item កន្សោម $\sqrt{1+\sqrt{7+\sqrt{1+\sqrt{7+\sqrt{1+\sqrt{7+\dots}}}}}}$ ស្មើនឹង 
	\begin{enumerate}[k,5]
		\item $2$
		\item $-2$
		\item $-3$
		\item $5$
		\item ចម្លើយផ្សេង
	\end{enumerate}
	{\color{blue}\hrulefill}
	\item បើ $x_0=0,x_{n+1}=2+\frac{1}{2+x_n}, n=1,2,3,\dots$ នោះលីមីតនៃស្វ៊ីត $x_n$ ស្មើនឹង
	\begin{enumerate}[k,5]
		\item $\sqrt{6}$
		\item $-\sqrt{5}$
		\item $\sqrt{7}$
		\item $\sqrt{5}$
		\item ចម្លើយផ្សេង
	\end{enumerate} 
	{\color{blue}\hrulefill}
\end{enumerate}
\makeads
\newpage
\maketitle\koc
{\color{blue}\hrulefill}
\begin{enumerate}[m]
	\item បើ $f'(x)$ ជាដេរីវេនៃអនុគមន៍ $f(x)=\frac{-1}{x^2+4}$ នោះ
	\begin{enumerate}[k,5]
		\item $-\frac{1}{\left(x^2+4\right)^2}$
		\item $\frac{2x}{\left(x^2+4\right)^2}$
		\item $-\frac{2x}{\left(x^2+4\right)^2}$
		\item $\frac{1}{\left(x^2+4\right)^2}$
		\item $\frac{2x}{x^2+4}$
	\end{enumerate}
	{\color{blue}\hrulefill}
	\item គេឲ្យវ៉ិចទ័របី $\vec{a}=(1,1,1), \vec{b}=(-1,-2,1), \vec{c}=(-1,-2,1)$ ។ ចូរគណនាមាឌ $\mathbf{V}$ នៃប្រឡេពីប៉ែត\\ ដែលកំណត់ដោយវ៉ិចទ័រទាំងបីនេះ ។
	\begin{enumerate}[k,5]
		\item $\mathbf{V}=6$
		\item $\mathbf{V}=7$
		\item $\mathbf{V}=8$
		\item $\mathbf{V}=9$
		\item ចម្លើយផ្សេង
	\end{enumerate}
	{\color{blue}\hrulefill}
	\item លេខខ្ទង់រាយនៃ $2017^{2018}$ គឺ
	\begin{enumerate}[k,5]
		\item $1$
		\item $3$
		\item $5$
		\item $7$
		\item $9$
	\end{enumerate}
	{\color{blue}\hrulefill}
	\item តើ $8^{16}5^{42}$ មានលេខចំនួនប៉ុន្មានខ្ទង់ 
	\begin{enumerate}[k,5]
		\item $42$
		\item $43$
		\item $44$
		\item $45$
		\item ចម្លើយផ្សេង
	\end{enumerate}
	{\color{blue}\hrulefill}
	\item តាង $x=\frac{-1+i\sqrt{3}}{2}$ និង $x=\frac{-1-i\sqrt{3}}{2}$ តើសមីការមួយណាខាងក្រោមដែលមិនត្រឹមត្រូវ?
	\begin{enumerate}[k,3]
		\item $x^5+y^5=-1$
		\item $x^7+y^7=-1$
		\item $x^9+y^9=-1$
		\item $x^{11}+y^{11}=-1$
		\item $x^{13}+y^{13}=-1$
	\end{enumerate}
	{\color{blue}\hrulefill}
	\item បើ $\left(3x-1\right)^7=a_7x^7+a_6x^6+\dots+a_1x+a_0$ នោះគេបាន $a_1+2a_2+\dots+7a_7$ ស្មើនឹង
	\begin{enumerate}[k,5]
		\item $2$
		\item $14$
		\item $21\times2^4$
		\item $21\times2^5$
		\item $21\times2^6$
	\end{enumerate}
	{\color{blue}\hrulefill}
	\item តាង $p$ ជាចំនួនគត់វិជ្ជមាន ។ គណនា $\lim\limits_{n\to\infty}\frac{1^{p}+2^{p}+\dots+n^{p}}{n^{p+1}}$
	\begin{enumerate}[k,5]
		\item $1$
		\item $\frac{1}{p+1}$
		\item $\frac{1}{p-1}$
		\item $\frac{1}{p}-\frac{1}{p-1}$
		\item $\frac{1}{p+2}$
	\end{enumerate}
	{\color{blue}\hrulefill}
	\item តើ $7^{2017}+7^{2018}+7^{2019}$ ចែកដាក់នឹងចំនួនមួយណាខាងក្រោម៖
	\begin{enumerate}[k,5]
		\item $41$
		\item $47$
		\item $57$
		\item $75$
		\item $141$ 
	\end{enumerate}
	{\color{blue}\hrulefill}
\end{enumerate}
%\makeads
\newpage
\maketitle\koc
{\color{blue}\hrulefill}
\begin{enumerate}[m]
	\item បើ $\alpha$ និង $\beta$ ជាប្ញសនៃសមីការដឺក្រេទី២ $x^2-4x+9=0$ ។ ចូរកំណត់រកតម្លៃនៃ $\alpha^{2}+\beta^{2}$ 
	\begin{enumerate}[k,5]
		\item $-1$
		\item $-2$
		\item $-3$
		\item $-4$
		\item ចម្លើយផ្សេង
	\end{enumerate}
	{\color{blue}\hrulefill}
	\item រកសមីការដឺក្រេទី២ ដែលមានប្ញសពីរគឺ $2$ និង $5$ ។ 
	\begin{enumerate}[k,3]
		\item $2x^2+7x^2+10=0$
		\item $x^2-7x+3=0$
		\item $x^2-7x+10=0$
		\item $x^2-4x+1=0$
		\item ចម្លើយផ្សេង
	\end{enumerate}
	{\color{blue}\hrulefill}
	\item កំណត់ផ្ចិត $(x_0, y_0)$ នៃរង្វង់ $\left(C\right)$ ដែលមានសមីការទូទៅ $\left(C\right):~~x^2+4x+y^2-2y-4=0$
	\begin{enumerate}[k,5]
		\item $\left(4,-2\right)$
		\item $\left(2,4\right)$
		\item $\left(1,-1\right)$
		\item $\left(2,1\right)$
		\item $\left(-2,1\right)$
	\end{enumerate}
	{\color{blue}\hrulefill}
	\item តាង $f(x)=\left(x-1\right)^{5}+5\left(x-1\right)^{4}+10\left(x-1\right)^{3}+10x^2-15x+4$ ។ នោះ $f(x)$ ស្មើនឹង
	\begin{enumerate}[k,5]
		\item $x^5$
		\item $\left(x-1\right)^4$
		\item $\left(x-1\right)^5$
		\item $x^5-2$
		\item $x^5-1$
	\end{enumerate}
	{\color{blue}\hrulefill}
	\item ចូររកសំណល់នៃការចែកពហុធា $f(x)$ ដោយ $g(x)$ ដែល $f(x)=2x^{15}+x^{12}-2x^2+3x+1$ និង $g(x)=x^2-1$ 
	\begin{enumerate}[k,5]
			\item $x+5$
			\item $x-5$
			\item $5x$
			\item $5$
			\item ចម្លើយផ្សេង
	\end{enumerate}
	{\color{blue}\hrulefill}
	\item គេឲ្យ $f$ ជាអនុគមន៍ដែលផ្ទៀងផ្ទាត់ $3f(x)+2f\left(\frac{1}{x}\right)=\frac{5}{4x^2}$ ។ ចូរកំណត់រកអនុគមន៍ $f(x)$ ។ 
	\begin{enumerate}[k,3]
		\item $f(x)=\frac{3}{2x^4}$
		\item $f(x)=\frac{4x^2}{3-2x^4}$
		\item $f(x)=\frac{3+2x^4}{4x^2}$
		\item $f(x)=\frac{3-2x^4}{4x^2}$
		\item $f(x)=\frac{2x^4-3}{4x^2}$
	\end{enumerate}
	{\color{blue}\hrulefill}
	\item គណនា $\mathbf{I}=\int_{-\pi}^{\pi}\frac{x^3-\sin x}{\cos x+x^2+4}dx$
	\begin{enumerate}[k,5]
		\item $-1$
		\item $-\pi$
		\item $0$
		\item $\pi$
		\item ចម្លើយផ្សេង
	\end{enumerate}
	{\color{blue}\hrulefill}
	\item គណនា $\mathbf{L}=\lim\limits_{x\to+\infty}\left(\frac{x^2+5x+3}{x^2+x+3}\right)^x$។
	\begin{enumerate}[k,5]
		\item $1$
		\item $e$
		\item $e^{-1/4}$
		\item $e^{1/4}$
		\item $e^{4}$
	\end{enumerate}
	{\color{blue}\hrulefill}
\end{enumerate}
%\makeads
\newpage
\maketitle\koc
{\color{blue}\hrulefill}
\begin{enumerate}[m]
	\item ចូររកដែនកំណត់នៃអនុគមន៍ $f(x)=\frac{1}{\sqrt{|x|-x}}$ ។
	\begin{enumerate}[k,5] 
		\item $\left(-\infty,\infty\right)$
		\item $\left(-\infty,0\right)$
		\item $\left(-\infty,0\right]$
		\item $\left(0,\infty\right)$
		\item $\left[0,\infty\right)$
	\end{enumerate}
	{\color{blue}\hrulefill}
	\item តាង $\left(u_n\right)$ ជាស្វ៊ីតនៃចំនួនពិតកំណត់ដោយទំនាក់ទំនង $u_1=2$ និង $u_{n+1}=\frac{2}{1+u_n}, \forall n\ge1$ គណនា $\lim\limits_{n\to\infty}u_n$ ។
	\begin{enumerate}[k,5]
		\item $-1$
		\item $0$
		\item $1$
		\item $2$
		\item $+\infty$
	\end{enumerate}
	{\color{blue}\hrulefill}
	\item ចូរគណនា $\lim\limits_{x\to0}\left(\frac{1+x\cdot8^x}{1+x\cdot2^x}\right)^{\frac{1}{x^2}}$ ។
	\begin{enumerate}[k,5]
		\item $2$
		\item $3$
		\item $4$
		\item $5$
		\item ចម្លើយផ្សេង
	\end{enumerate}
	{\color{blue}\hrulefill}
	\item យក $f(x)=\frac{x^2\left(2\ln x-1\right)}{4}$ ជាអនុគមន៍ និង $f'(x)$ ជាដេរីវេនៃ $f(x)$ ។ គេបាន
	\begin{enumerate}[k,5]
		\item $f'(x)=3x\ln x$
		\item $f'(x)=x^2\ln x$
		\item $f'(x)=x\ln x$
		\item $f'(x)=\frac{\ln x}{x}$
		\item ចម្លើយផ្សេង
	\end{enumerate}
	{\color{blue}\hrulefill}
	\item តាង $\alpha\in\left]\frac{\pi}{2},\pi\right[$ ។ ចូររកទម្រង់ធរណីមាត្រនៃចំនួនកុំផ្លិច $z=-1+i\tan\alpha$ ។
	\begin{enumerate}[k,2]
		\item $\frac{1}{\cos\alpha}\left[ \cos\left(\pi-\alpha\right)+i\sin\left(\pi-\alpha\right) \right]$
		\item $\frac{1}{\cos\alpha}\left[ \cos\left(-\alpha\right)+i\sin\left(-\alpha\right)\right]$
		\item $-\frac{1}{\cos\alpha}\left[ \cos\left(\alpha\right)+i\sin\left(\alpha\right)\right]$
		\item $-\frac{1}{\cos\alpha}\left[ \cos\left(-\alpha\right)+i\sin\left(-\alpha\right)\right]$
		\item $\frac{1}{\cos\alpha}\left[ \cos\left(\alpha\right)+i\sin\left(\alpha\right) \right]$
	\end{enumerate}
	{\color{blue}\hrulefill}
	\item ចូរគណនា $\lim\limits_{x\to0}\frac{\sin^3x}{\sqrt[2018]{1+1009x^3}-1}$
	\begin{enumerate}[k,5]
		\item $2$
		\item $-3$
		\item $3$
		\item $-2$
		\item ចម្លើយផ្សេង
	\end{enumerate}
	{\color{blue}\hrulefill}
	\item ចូរគណនា $\lim\limits_{x\to\frac{\pi}{3}}\frac{8\cos^25x+2\cos x-3}{4\cos^25x+8\cos x-5}$
	\begin{enumerate}[k,5]
		\item $-\frac{19}{6}$
		\item $-\frac{19}{7}$
		\item $\frac{19}{7}$
		\item $\frac{19}{6}$
		\item ចម្លើយផ្សេង
	\end{enumerate}
	{\color{blue}\hrulefill}
\end{enumerate}
\makeads
\newpage
\maketitle
{\color{blue}\hrulefill}
\begin{enumerate}[m]
	\item ចូរគណនាតម្លៃលេខនៃ $\cos\left(\frac{\pi}{5}\right)$ ។
	\begin{enumerate}[k,2]
		\item $\cos\left(\frac{\pi}{5}\right)=\frac{\sqrt{\sqrt{5}-1}}{2}$
		\item $\cos\left(\frac{\pi}{5}\right)=\frac{\sqrt{\sqrt{5}+1}}{2}$
		\item $\cos\left(\frac{\pi}{5}\right)=\frac{\sqrt{5}+1}{4}$
		\item $\cos\left(\frac{\pi}{5}\right)=\frac{\sqrt{5}-1}{4}$
		\item ចម្លើយផ្សេង
	\end{enumerate}
	{\color{blue}\hrulefill}
	\item គេយក $f(x)=x^3-3x+m+2$ ដែល $m$ ជាប៉ារ៉ាម៉ែត្រ។ ចូរកំណត់តម្លៃទាំងអស់នៃ $m$ ដើម្បីឲ្យខ្សែកោងតាងអនុគមន៍នេះកាត់តាមអ័ក្សអាប់ស៊ីសបាន៣ ចំណុចខុសគ្នា។
	\begin{enumerate}[k,5]
		\item $m<-8$
		\item $-8\le m <-4$
		\item $-4<m<0$
		\item $-4\le m \le0$
		\item ចម្លើយផ្សេង
	\end{enumerate}
	{\color{blue}\hrulefill}
	\item តម្លៃនៃ $\lim\limits_{x\to0}\left(x^{x^{2017}}\right)$ គឺ
	\begin{enumerate}[k,5]
		\item $1$
		\item $2$
		\item $e$
		\item $e^{-1}$
		\item ចម្លើយផ្សេង
	\end{enumerate}
	{\color{blue}\hrulefill}
	\item ចូរគណនាដេរីវេនៃអនុគមន៍ $f(x)=x^{x^{2017}}$ ។
	\begin{enumerate}[k,3]
		\item $x^{x^{2017}}\left(2017\ln\left(x\right)+1\right)$
		\item $x^{x^{2017}+2016}\left(2016\ln\left(x\right)+1\right)$
		\item $x^{x^{2017}+2016}\left(2017\ln\left(x\right)+1\right)$
		\item $x^{x^{2017}+2016}\left(2017\ln\left(x\right)-1\right)$
		\item ចម្លើយផ្សេង
	\end{enumerate}
	{\color{blue}\hrulefill}
	\item គេឲ្យ $f$ ជាអនុគមន៍កំណត់បាន និងមានអាំងតេក្រាលលើចន្លោះ $\left[0;\frac{\pi}{2}\right]$។ ចូរគណនារកតម្លៃនៃ $\mathbf{I}=\int_{0}^{\frac{\pi}{2}}\frac{f\left(\cos x\right)}{f\left(\cos x\right)+f\left(\sin x\right)}dx$
	\begin{enumerate}[k,5]
		\item $\mathbf{I}=\frac{\pi}{3}$
		\item $\mathbf{I}=\frac{2\pi}{3}$
		\item $\mathbf{I}=\frac{\pi}{2}$
		\item $\mathbf{I}=\frac{\pi}{4}$
		\item ចម្លើយផ្សេង
	\end{enumerate}
	{\color{blue}\hrulefill}
	\item រកសមីការបន្ទាត់កាត់តាមចំណុច $\left(1,0\right)$ ហើយប៉ះនឹងរង្វង់ដែលមានសមីការ $x^2+2x+y^2=0$ ។
	\begin{enumerate}[k,3]
		\item $y=\pm\frac{\sqrt{3}}{3}\left(x-1\right)$
		\item $y=\pm\left(x-1\right)$
		\item $y=\pm\frac{1}{3}\left(x-1\right)$
		\item $y=\pm\sqrt{3}\left(x-1\right)$
		\item $y=\pm\frac{\sqrt{3}}{2}\left(x-1\right)$
	\end{enumerate}
	{\color{blue}\hrulefill}
	\item សំណល់នៃការចែក $3^{2017}$ នឹង $7$ គឺ
	\begin{enumerate}[k,5]
		\item $1$
		\item $2$
		\item $3$
		\item $5$
		\item $6$
	\end{enumerate}
	{\color{blue}\hrulefill}
\end{enumerate}
\newpage
\maketitle
{\color{blue}\hrulefill}
\begin{enumerate}[m]
	\item គេឲ្យ $E$ ជាសំណុំប្ញសទាំងអស់នៃសមីការ $x^2+5x+6=0$ ។
\end{enumerate}
\end{document}