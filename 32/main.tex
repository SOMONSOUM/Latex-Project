\documentclass[a4paper, 11pt]{exam}
\makeatletter
\usepackage[top=0.5cm, left=1cm, bottom=1.8cm, right=1.5cm]{geometry}
\usepackage{amsmath,amssymb}
\usepackage{tcolorbox}
\usepackage[export]{adjustbox}
\usepackage{graphicx}
\usepackage{wrapfig}
\usepackage{pgf}
\usepackage{tikz}% graphic drawing
\usetikzlibrary{arrows}
\pagestyle{empty}
\usepackage{wasysym}
\usepackage{mathpazo}% change math font
\usepackage{enumitem}% change list environment like enumerate, itemize and description
\usepackage{multicol}% multi columns
\usepackage{xcolor}
\newcommand{\teacher}{ស៊ំុ សំអុន}
\newcommand{\tell}{០៩៦ ៩៤០ ៥៨៤០}
\newcommand{\class}{ត្រៀមប្រឡងសញ្ញាបត្រមធ្យមសិក្សាទុតិយភូមិ}
\newcommand{\dateofexam}{សម័យប្រឡង៖~សីហា ២០ ២០១៨}
\newcommand{\subject}{វិញ្ញាសា៖~គណិតវិទ្យា~(វិទ្យាសាស្រ្តពិត)}
\newcommand{\timelimit}{១៥០នាទី}
\newcommand{\score}{ពិន្ទុសរុប៖~១២៥~ពិន្ទុ}
\usepackage[no-math]{fontspec}% font specfication
\setmainfont{Khmer OS Content}% set default font to Khmer OS
\setsansfont[Ligatures=TeX,AutoFakeBold=0,AutoFakeSlant=0.25]{Khmer OS Muol Light}% sans serif font
%
\newcommand{\heart}{\ensuremath\heartsuit}
\newcommand{\butt}{\rotatebox[origin=c]{180}{\heart}}
\newcommand*\circled[1]{\tikz[baseline=(char.base)]{
		\node[shape=circle,draw,inner sep=2pt] (char) {#1};}}
%
\makeatletter %line1-8 make Khmer known
\def\@khmernum#1{\expandafter\@@khmernum\number#1\@nil} 
\def\@@khmernum#1{%
	\ifx#1\@nil 
	\else 
	\char\numexpr#1+"17E0\relax 
	\expandafter\@@khmernum\fi 
} 
\def\knum#1{\expandafter\@khmernum\csname c@#1\endcsname}
\def\khmernumeral#1{\@@khmernum#1\@nil}
\AddEnumerateCounter{\knum}{\@knum}{}
\makeatother
\makeatletter
\newcommand*{\kalph}[1]{%
	\expandafter\@kalph\csname c@#1\endcsname%
}
\newcommand*{\@kalph}[1]{%
	\ifcase#1\or ក\or ខ\or គ\or ឃ\or ង\or ច\or ឆ\or ជ\or ឈ\or ញ\or ដ\or ឋ\or ឌ\or ឍ\or ណ\or ត\or ថ\or ទ\or ធ\or ន\or ប\or ផ\or ព\or ភ\or ម\or យ\or រ\or ល\or វ\or ស\or ហ\or ឡ\or អ%
	\else\@ctrerr\fi%
}
\AddEnumerateCounter{\kalph}{\@kalph}{}
\makeatother

\SetEnumitemKey{I}{%
	leftmargin=*,
	label={\protect\tikz[baseline=-0.9ex]\protect\node[text=blue,fill=white]{\en\Roman*.};},%
	font=\small\sffamily\bfseries,%
	labelsep=1ex,%
	topsep=0pt}
%
\SetEnumitemKey{a}{%
	leftmargin=*,%
	label={\protect\tikz[baseline=-0.9ex]\protect\node[text=black,fill=white]{\kalph*.};},%
	font=\small\kbk\bfseries,%
	labelsep=1ex,%
	topsep=0pt}
%
\SetEnumitemKey{1}{leftmargin=*,%
	label={\protect\tikz[baseline=-0.9ex]\protect\node[text=red,fill=white]{\knum*.};},%
	font=\small\kbk\bfseries,%
	labelsep=1ex,%
	topsep=0pt}
%
\def\hard{\leavevmode\makebox[0pt][r]{\large\ensuremath{\star}\hspace{2em}}}
%
\def\hhard{\leavevmode\makebox[0pt][r]{\large\ensuremath{\star\star}\hspace{2em}}}
%

\everymath{\protect\displaystyle\protect\color{blue}}
%
%\pagecolor{cyan!1!white}
%
\usepackage{amsmath}
\usepackage{amssymb}
\usepackage{wasysym}
\makeatother
\makeatother
\setmainfont{Khmer OS Content}% set default font to Khmer OS
\newcommand{\ko}{\fontspec[Script=Khmer]{Khmer OS}\selectfont}
\newcommand{\kml}{\fontspec[Script=Khmer]{Khmer OS Muol Light}\selectfont}
\newcommand{\kos}{\fontspec[Script=Khmer]{Khmer OS System}\selectfont}
\newcommand{\kb}{\fontspec[Script=Khmer]{Khmer OS Battambang}\selectfont}
\newcommand{\kbk}{\fontspec[Script=Khmer]{Khmer OS Bokor}\selectfont}
\newcommand{\en}{\fontspec{Liberation Serif}\selectfont}

\pagestyle{foot}
\firstpagefooter{}{ ទំព័រ~\thepage\ នៃ \numpages}{}
\runningfooter{រៀបរៀងដោយ~\teacher}{ ទំព័រ \thepage\ នៃ \numpages}{ទូរស័ព្ទ~\tell}
\runningfootrule

\begin{document}
\noindent
%\sffamily\color{black}
\begin{tabular*}{\textwidth \sffamily\color{black}}{l @{\extracolsep{\fill}} r @{\extracolsep{6pt}} l}
\textbf{\class} & \textbf{មណ្ឌលប្រឡង} & \makebox[2in]{\hrulefill}\\
\textbf{\dateofexam} & \textbf{លេខបន្ទប់} & \makebox[2in]{\hrulefill}\\
\textbf{\subject} & \textbf{លេខតុ} & \makebox[2in]{\hrulefill}\\
\textbf{\score} & \textbf{ឈ្មោះបេក្ខជន} & \makebox[2in]{\hrulefill}\\
\textbf{រយៈពេលសរុប៖ \timelimit} & \textbf{ហេត្ថលេខា} & \makebox[2in]{\hrulefill}
\end{tabular*}\\
\noindent
\rule[2ex]{\textwidth\color{magenta}}{2pt}
\begin{center}
	\sffamily\color{black}
	ប្រធានលំហាត់ ០១\\
\end{center}
\vspace{-0.3cm}
\begin{enumerate}[I]
	\item គេមានចំនួនកុំផ្លិច​ $z_1=1+i$ ជាឬសនៃសមីការ​ $(C):$ $z^2-2z+2=0$ ។
	\begin{enumerate}[a]
		\item រកឬសមួយផ្សេងទៀតនៃសមីការ $(C)$ រួចសរសេរ ឬសទាំងពីរនៃសមីការ $(C)$ ជាទម្រង់ត្រីកោណមាត្រ
		\item គណនា $z_1\times z_2$ និង $\frac{z_1}{z_2}$ ។
		\item គណនា $z_1^{2018} + z_2^{2018}$ ។
	\end{enumerate}
	\item គណនាលីមីតនៃអនុគមន៍ខាងក្រោម៖
	\begin{multicols}{3}
		\begin{enumerate}[a]
			\item $\lim\limits_{x\to2} \frac{x^2-5x+6}{x^2-4}$
			\item $\lim\limits_{x\to 0} \frac{\sin6x}{\sqrt{1+x}-\sqrt{1-x}}$
			\item $\lim\limits_{x\to+\infty} (\sqrt{2x^2+1}-\sqrt{x^2-5})$
		\end{enumerate}
	\end{multicols}
	\item \begin{enumerate}[1]
		\item គណនាអាំងតេក្រាលនៃអនុគមន៍ខាងក្រោម៖
		\begin{multicols}{3}
			\begin{enumerate}[a]
				\item $\int_{-1}^{2}(3x^2+x-3)dx$
				\item $\int_{0}^{\frac{\pi}{2}} (1-2\sin^2x)dx$
				\item $\int_{2}^{3}\frac{1}{(x-1)^2}dx$
			\end{enumerate}
		\end{multicols}
		\item គេមានអនុគមន៍ $f(x)=\frac{3x^2}{x^3-1}$ ។
		\begin{enumerate}[a]
			\item កំណត់ចំនួនពិត $a, ~b$ និង $c$ ដើម្បីឲ្យ $f(x)=\frac{a}{x-1}+\frac{bx+c}{x^2+x+1}$ ។
			\item គណនាអាំងតេក្រាល $\int_{-1}^{0} f(x) dx$ 
		\end{enumerate}
	\end{enumerate}
	\item ដោះស្រាយសមីការឌីផេរ៉ង់ស្យែល $(E):$ $y''+4y = 0$ បើគេដឹងថា $y(\pi) =-1$ និង $y'(\pi) =4$ ។ 
	\item គេដាក់ឃ្លីពណ៌ក្រហម $5$ និងឃ្លីពណ៌ខៀវ $15$ ចូរគ្នាក្នុងប្រអប់តែមួយ។ ឃ្លី $3$ ត្រូវបានគេយកចេញពីប្រអប់ដោយចៃដន្យ។ រកប្រូបាបដែលចាប់បាន៖
	\begin{enumerate}[a]
		\item $A:$ ឃ្លីទាំង $3$ សុទ្ធតែពណ៌ក្រហម
		\item $B:$ ឃ្លី $2$ ពីគត់ពណ៌ក្រហម 
		\item $C:$ យ៉ាងតិចឃ្លី $2$ ពណ៌ក្រហម
		\item $D:$ យ៉ាងហោចណាស់មានឃ្លីពណ៌ខៀវមួយ ។ 
	\end{enumerate}
	\item គេមានសមីការប៉ារ៉ាបូល $(P): y^2-6y+2x+1=0$ ។
	\begin{enumerate}[a]
		\item កំណត់សមីការស្តង់ដានៃប៉ារ៉ាបូលនេះ រួចរកកូអរដោនេ កំពូល កំណុំ និងសមីការបន្ទាត់ប្រាប់ទិស ។
		\item រកកូអរដោនេនៃចំណុចប្រសព្វរវាងប៉ារ៉ាបូល និងអក្ស័អរដោនេ រួចសង់ប៉ារ៉ាបូលនេះ​ ។ 
	\end{enumerate}
	\item គេមានអនុគមន៍ $f$ ដោយ $f(x) = 2e^{x-2}-2x+1$ កំណត់ចំពោះគ្រប់ $x\in\mathbb{R}$​។
	\begin{enumerate}[a]
		\item គណនា $\lim\limits_{x\to-\infty} f(x)$ និង $\lim\limits_{x\to+\infty} f(x)$ ។
		\item គណនា $f'(x)$ សិក្សាសញ្ញានៃ $f'(x)$ រួចសង់តារាងអថេរភាពនៃ $f$ ។
		\item បង្ហាញថា $\Delta:​y=-2x+1$ ជាសមីការអាស៊ីមតូតទ្រេតនៃក្រាប $(C)$ ខាង $-\infty$ រួចសិក្សាទីតាំងធៀបនៃ $\Delta$ និង $(C)$
		\item សង់ក្រាប $C$ និង $\Delta$ ក្នុងតម្រុយតែមួយ ។
		\item កំណត់ផ្ទៃក្រឡាផ្នែកប្លង់ដែលខណ្ឌដោយក្រាប $(C)$ និងបន្ទាត់ $\Delta$​លើចន្លោះ $[1,2]$ ។
	\end{enumerate}
\end{enumerate}
\begin{center}
	\sffamily\color{black}
	សូមសំណាងល្អ!
\end{center}\newpage

\end{document}
