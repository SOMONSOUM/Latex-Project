\documentclass[12pt, a4paper]{article}
%%import package named hightest
\usepackage{hightest}
\usepackage{amsmath}
\usepackage{cases}
\usepackage[export]{adjustbox}
\usepackage{wrapfig}
\usepackage{tkz-tab}
%\usepackage{mathpazo}% change math font
%\usepackage[no-math]{fontspec}% font specfication
\header{រៀនគណិតវិទ្យាទាំងអស់គ្នា}{គណិតវិទ្យា}{០៤/០៣/២០១៩}
\footer{រៀបរៀង និងបង្រៀនដោយ ស៊ុំ សំអុន}{ទំព័រ \thepage}{០៩៦ ៩៤០ ៥៨៤០}
\everymath{\protect\displaystyle\protect\color{magenta}}
\begin{document}
\maketitle
\begin{enumerate}[m]
	\item \textbf{\sffamily លីមីត៖}
	\begin{enumerate}[k,2]
		\item $\lim\limits_{x\to+\infty}e^{x}=+\infty$
		\item $\lim\limits_{x\to-\infty}e^{x}=0$
		\item $\lim\limits_{x\to+\infty}\frac{e^x}{x}=+\infty$		
		\item $\lim\limits_{x\to+\infty}\frac{e^x}{x^n}=+\infty$
		\item $\lim\limits_{x\to+\infty}\frac{x}{e^x}=0^+$
		\item $\lim\limits_{x\to+\infty}\frac{x^n}{e^x}=0^+$
		\item $\lim\limits_{x\to 0}\frac{e^x-1}{x}=1$
		\item $\lim\limits_{x\to +\infty}\bigg(1+\frac{1}{n}\bigg)^n=e$
	\end{enumerate}
	\item \textbf{\sffamily ដេរីវេ៖}
	\begin{center}
		$y=e^x$~~~~នោះ~~~~$y'=e^x$\\
		$y=e^{u(x)}$~~~~នោះ~~~~$y'=u'(x)e^{u(x)}$
	\end{center}
	\item \textbf{\sffamily អនុគមន៍៖}
	\begin{center}
		$y=f(x)=e^x>0~~~~\forall x\in\mathbb{R}$\\
		$e=2.7182...$
	\end{center}
	\item \textbf{\sffamily សមីការ~~វិសមីការ~~អិចស្ប៉ូណង់ស្យែល៖}
	\begin{center}
		$e^x=k$~~សមមូល~~$x=\ln k$\\
		$e^x>k$~~សមមូល~~$x>\ln k$\\
		$e^x<k$~~សមមូល~~$x<\ln k$
	\end{center}\newpage
	\item \textbf{\sffamily រូបមន្តសំខាន់ៗ៖}
	\begin{multicols}{3}
		\begin{itemize}
			\item $e^0=1$
			\item $e^m\times e^n=e^{m+n}$
			\item $\frac{e^m}{e^n}=e^{m-n}$ 
		\end{itemize}
	\end{multicols}
	\item \textbf{\sffamily របៀបរក~~និង~អាស៊ីមតូតទ្រេត៖}
	\begin{center}
		$f(x)=a\pm e^{-x}$ ~~ ត្រូវរកអាស៊ីមតូតដេកខាង $+\infty$\\
		$f(x)=a\pm e^{x}$ ~~ ត្រូវរកអាស៊ីមតូតដេកខាង $-\infty$\\
		$f(x)=ax+b\pm e^{-x}$ ~~ ត្រូវរកអាស៊ីមតូតទ្រេតខាង $+\infty$\\
		$f(x)=ax+b\pm e^{x}$ ~~ ត្រូវរកអាស៊ីមតូតទ្រេតខាង $-\infty$
	\end{center}
	\item \textbf{\sffamily  អត្រាការប្រាក់៖}
		\begin{enumerate}[k]
			\item ការប្រាក់សមាស $P=P_0\bigg(1+\frac{r}{n}\bigg)^{nt}$
			\item ការប្រាក់បន្តបន្ទាប់ $P=P_0e^{rt}$
		\end{enumerate}
		\begin{multicols}{2}
			\begin{itemize}
				\item $P$ ប្រាក់សរុប
				\item $P_0$ ប្រាក់ដើម
				\item $n$ ចំនួនដងនៃការទូទាត់ការប្រាក
				\item $r$ អត្រាការប្រាក់ 
				\item $t$ រយៈពេលគិតជាឆ្នាំ
			\end{itemize}
		\end{multicols}
		\item \textbf{\sffamily លក្ខណៈទូទៅនៃស្វ័យគុណ៖}
		\begin{enumerate}[k, 3]
			\item $(a^n)^m=a^{n\cdot m}$
			\item $a^m\times a^m=a^{m+n}$
			\item $\frac{a^m}{a^n}=a^{m-n}$
			\item $\bigg(\frac{a}{b}\bigg)^n=\bigg(\frac{b}{a}\bigg)^{-n}$
			\item $\sqrt[n]{a^m}=a^{\frac{m}{n}}$
			\item $\frac{1}{a^n}=a^{-n}$
		\end{enumerate}
	\begin{center}
		\sffamily\color{black}
		សូមសំណាងល្អ!
	\end{center}
\end{enumerate}
\newpage
\begin{center}
	\sffamily\color{black}
	\circled{០២}\\
	សិក្សាអនុគមន៍អិចស្ប៉ូណង់ស្យែល(រូបមន្តសុទ្ធ)
\end{center}
\begin{enumerate}[m]
	\item \textbf{\sffamily លីមីត៖}
	\begin{enumerate}[k,2]
		\item $\lim\limits_{x\to+\infty}\ln x=+\infty$
		\item $\lim\limits_{x\to 0^+}\ln x=-\infty$
		\item $\lim\limits_{x\to+\infty}\frac{\ln x}{x}=0^+$		
		\item $\lim\limits_{x\to0^+}x\ln x=0^-$
		\item $\lim\limits_{x\to+\infty}\frac{\ln x}{x^n}=0^+$
		\item $\lim\limits_{x\to0^+}x^n\ln x=0^-~~(n>0)$
	\end{enumerate}
	\item \textbf{\sffamily ដេរីវេ៖}
	\begin{center}
		$y=\ln x$~~~~នោះ~~~~$y'=(\ln x)'=\frac{1}{x}$\\
		$y=\ln u(x)$~~~~នោះ~~~~$y'=[\ln u(x)]'=\frac{u'(x)}{u(x)}$
	\end{center}
	\item \textbf{\sffamily សមីការ~~វិសមីការ~~លោការីតនេពែ៖}
	\begin{center}
		$\ln x=k$~~សមមូល~~$x=e^k$\\
		$\ln x>k$~~សមមូល~~$x>e^k$\\
		$\ln x<k$~~សមមូល~~$0<x<e^k$
	\end{center}
	\item \textbf{\sffamily រូបមន្តសំខាន់ៗ៖}
	\begin{multicols}{3}
		\begin{itemize}
			\item $\ln1=0$				
			\item $\ln e=1$
			\item $\ln x^\alpha=\alpha\ln x$ 
			\item $\ln u\cdot v=\ln u+\ln v$
			\item $\ln \frac{u}{v}=\ln u-\ln v$
			\item $e^{\ln k} =k$
		\end{itemize}
	\end{multicols}\newpage
	\item \textbf{\sffamily របៀបរក~~និង~អាស៊ីមតូតទ្រេត៖}
	\begin{center}
		$f(x)=a\pm \frac{\ln x}{x}$ ~~ ត្រូវរកអាស៊ីមតូតដេក និងឈរ\\
		$f(x)=ax+b\pm \frac{\ln x}{x}$ ~~ ត្រូវរកអាស៊ីមតូតឈរ និងទ្រេត
	\end{center}
	\item \textbf{\sffamily របៀបរកដែនកំណត់នៃអនុគមន៍៖}
	\begin{enumerate}[k]
		\item អនុគមន៍ $y=f(x)=ax^n\pm bx^{n-1}\pm cx^{n-2}\pm\cdots$ ជាអនុគមន៍ពហុធាមានន័យគ្រប់ $x\in\mathbb{R}$
		\item $y=f(x)=e^x$ មានន័យ គ្រប់ $x\in\mathbb{R}$
		\item $y=f(x)=e^{-x}$ មានន័យ គ្រប់ $x\in\mathbb{R}$
		\item $y=f(x)=\frac{P(x)}{Q(x)}$ មានន័យលុះត្រាតែ $Q(x)\neq0$
		\item $y=f(x)=\sqrt[2n]{P(x)}$ មានន័យលុះត្រាតែ $P(x)\geq0$
		\item $y=f(x)=\frac{P(x)}{\sqrt[2n]{P(x)}}$ មានន័យលុះត្រាតែ $P(x)>0$
		\item $y=f(x)=\ln x$ មានន័យលុះត្រាតែ $x>0$
		\item $y=f(x)=\ln P(x)$ មានន័យលុះត្រាតែ $P(x)>0$
		\item $y=f(x)=\frac{P(x)}{\ln Q(x)}$ មានន័យលុះត្រាតែ 
		$\begin{cases}
			\text{$Q(x)>0$}\\
			\text{$\ln Q(x)\neq0$}	
		\end{cases}$
		\item $y=f(x)=\log_{\alpha(x)}P(x)$ មានន័យលុះត្រាតែ 
		$\begin{cases}
			\text{$\alpha(x)>0$}\\
			\text{$\alpha(x)\neq1$}\\
			\text{$P(x)>0$}
		\end{cases}$		
	\end{enumerate}
	\begin{center}
		\sffamily\color{black}
		សូមសំណាងល្អ!
	\end{center}
\end{enumerate}
\end{document}