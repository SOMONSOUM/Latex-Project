\documentclass[12pt, a4paper]{article}
\usepackage[top=0.5cm, left=1cm, bottom=1.5cm, right=1.5cm]{geometry}
%%import package named hightest
\usepackage{hightest}
\usepackage{comment}
\usepackage{chemfig}
\usepackage[version=3]{mhchem}
\header{រៀនគណិតវិទ្យាទាំងអស់គ្នា}{ គីមីវិទ្យា}{\khmerdate}
\footer{រៀបរៀង និងបង្រៀនដោយ ស៊ុំ សំអុន}{ទំព័រ \thepage}{០៩៦ ៩៤០ ៥៨៤០}
\everymath{\protect\displaystyle\protect\color{black}}
\begin{document}
	\begin{center}
		\sffamily\color{black}
		\circled{០១}\\
		ជំពូក៣~~សូលុយស្យូងទឹក និង~$\ce{pH}$
	\end{center}
\maketitle
\begin{enumerate}[m]
	\item គណនា $\ce{pH}$ នៃសូលុយស្យុងអាស៊ីតខាងក្រោម៖
	\begin{enumerate}[k,3]
		\item $0.05M$ នៃ $\ce{HCl}$
		\item $0.04M$ នៃ $\ce{H2SO4}$
		\item $0.001M$ នៃ $\ce{HNO3}$
	\end{enumerate}
	\item គណនា $pH$ នៃសូលុយស្យុងបាសខាងក្រោម៖
	\begin{enumerate}[k, 3]
		\item $\ce{0.02M}$ នៃ $\ce{NaOH}$
		\item $\ce{0.002M}$ នៃ $\ce{Ca(OH)2}$
		\item $\ce{0.15M}$ នៃ $\ce{Ba(OH)2}$
	\end{enumerate}
	\item គណនាកំហាប់ $\left[H_3O^{+}\right]$ និង $\left[OH^{-}\right]$ នៃសូលុយស្យុងខាងក្រោម៖
	\begin{enumerate}[k,3]
		\item $pH=3.6$
		\item $pH=4.0$
		\item $pH=12.6$
	\end{enumerate}
	\item អង្គធាតុរាវសាប៊ូមានកំហាប់អុីយ៉ុងអុីដ្រុកសុីតស្មើនឹង $6.8\times 10^{-5}mol\cdot L^{-1}$ នៅ $25^{0}C$ ។
	\begin{enumerate}[k]
		\item តើសាប៊ូនេះជាអាសុីត បាស ឬណឺត ?
		\item គណនាកំហាប់អុីយ៉ុងអុីដ្រូញ៉ូម ។
		\item តើ $pH$ និង $pOH$ នៃសាប៊ូស្មើប៉ុន្មាន?
	\end{enumerate}
	\item នៅសីតុណ្ហភាពធម្មតានៃសារពាង្គកាយ $37^{0}C$ មានតម្លៃ $K_w$ ចំបោះទឹក $2.5\times 10^{-14}$ ។ គណនា $\left[H_3O^{+}\right]$ និង $\left[OH^{-}\right]$ នៅសីតុណ្ហភាពនេះ ។ តើទឹកនៅសីតុណ្ហភាព $37^{0}C$ ជាសូលុយស្យុងអាសុីត សូលុយស្យុងបាស ឬសូលុយស្យុងណឺត?
	\item ភាគសំណាកនៃម្សៅសូដាត្រូវបានរំលាយទៅក្នុងទឹក ហើយ​ $pOH$ នៃសូលុយស្យុងត្រូវបានរកឃើញមានតម្លៃស្មើ $5.8$ នៅ $25^{0}C$ ។
	\begin{enumerate}[k]
		\item តើសូលុយស្យុងនេះ ជាអាសីុត បាស ឬណឺត?
		\item គណនា $\left[H_3O^{+}\right]$ និង $\left[OH^{-}\right]$ នៃសូលុយស្យុង។
	\end{enumerate}
	\item សូលុយស្យុងអាសុីតនីទ្រិច $\left(HNO_3\right)$ មួយមានកំហាប់ $4.0\times 10^{-4}M$
	\begin{enumerate}[k]
		\item គណនាកំហាប់អុីយ៉ុងអុីដ្រូញ៉ូម ។
		\item គណនាកំហាប់អុីយ៉ុងអុីដ្រុកសុីត ។
		\item គណនា $pH$ នៃសូលុយស្យុង ។
	\end{enumerate}
	\item គេមានបរិមាណ $1.72g$ នៃ $Ca\left(OH\right)_{2}$ ទៅរំលាយក្នុងទឹកចំនួន $500mL$ ។
	\begin{enumerate}[k]
		\item សរសេរសមីការបំបែកនៃ $Ca\left(OH\right)_{2}$ ក្នុងទឹក ។
		\item គណនាកំហាប់ម៉ូឡារីតេនៃសូលុយស្យុង ។
		\item គណនា $pH$ នៃសូលុយស្យុង ។$\left(Ca=40~;~O=16~;~H=1\right)$
	\end{enumerate}
	\item គេរំលាយបរិមាណ $2.5 \times 10^{-3} mol$ នៃអាស៊ីតប្រូមីឌ្រីច $\left(HBr\right)$ ទៅក្នុង $1000 cm^{3}$ នៃទឹក គេទទួលបានសូលុយស្យុងមួយមាន $pH=2.6$ ។
	\begin{enumerate}[k]
		\item ចូរសរសេរសមីការតាងប្រតិកម្ម $HBr$ ជាមួយទឹក ។
		\item គេដឹងថា $HBr$ ជាអាសុីតខ្លាំង។ ចូរបង្ហាញ។
		\item តើគេប្រើ $pH$ សម្រាប់វាស់អ្វី? ។
	\end{enumerate}
	\item តើល្បាយ $100mL$ នៃសូលុយស្យុង $0.2M~~NaOH$ និង $200mL$ នៃសូលុយស្យុង $0.1M~~HNO_{3}$ មាន $pH$ ស្មើនឹងប៉ុន្មាន? 
	\item តើកំហាប់ម៉ូឡារីតេនៃអុីយ៉ុងសូដ្យូមនៅក្នុងសូលុយស្យុងត្រូវបានទង្វើដោយការពង្រាវ $250mL$ នៃ $0.55M~~Na_{2}SO_{4}$ ទៅជា $1.25L$ ស្មើនឹងប៉ុន្មាន?
	\item \begin{enumerate}[k]
		\item គេពង្រាវ $100$ដង នៃសូលុយស្យុងអាស៊ីតនីទ្រិចដែលមាន $pH=2.8$។ គណនា $pH$ ក្រោយពង្រាវនៃសូលុយស្យុងនេះ។  
		\item សូលុយស្យុងសូល្យូមអុីដ្រុកសុីតមួយមាន $pH=12.3$ ចំនួន $10mL$ គេចាក់បន្ថែម $990mL$ នៃសូល្យូមក្លរួដែលមានកំហាប់ $0.1M$ ។ គណនា $pH$ នៃល្បាយទទួលបាន។
	\end{enumerate}
	\item \begin{enumerate}[k]
		\item សូលុយស្យុងបាសមួយមាន $pH=12.6$ ។ គណនា $\left[OH^{-}\right]$ ដែលមានក្នុងសូលុស្យុងនេះ ។
		\item គេយក $10mL$ នៃសូលុយស្យុងបាសខាងលើនេះទៅលាយជាមួយ $990mL$ នៃទឹក ។\\
		ចូរគណនាចំនួនដងនៃការពង្រាវរបស់សូលុយស្យុងបាសនេះ។
		\item គណនា $\left[OH^{-}\right]_{f}$ និង $pH_{f}$ នៃសូលុយស្យុងបាសនេះ ។
		\item ចូរធ្វើសេចក្តីសន្និដ្ឋានភាពប្រែប្រួលនៃ $\left[OH^{-}\right]$ និង $pH$ កាលណាគេពង្រាវសូលុយស្យុងបាសនេះ។ 
	\end{enumerate}
	\item គេយកសូលុយស្យុងអាសុីតក្លរីឌ្រិចដែលមានកំហាប់ $0.03M$ ត្រូវនឹងមាឌ $30mL$ ទៅពង្រាវចំនួន $3$ ដង រួចទទួលបានសូលុយស្យុងថ្មីមួយតាងដោយសូលុយស្យុង $\left(S_{1}\right)$ ។
	\begin{enumerate}[k]
		\item គណនា $pH$ នៃសូលុយស្យុង $S_1$ ។
		\item គណនាមាឌទឹកចំបាច់ដែលត្រូវថែម ។
	\end{enumerate}
	\item ភាគសំណាក $40.0mL$ នៃ $0.25M~~KOH$ ត្រូវបន្ថែមទៅក្នុង $60.omL$ នៃ $0.15M$ សូលុស្យុង $Ba\left(OH\right)_{2}$ ។
	\begin{enumerate}[k]
		\item តើកំហាប់ជាម៉ូល $\left[OH^{-}\right]$ ក្នុងសូលុស្យុងទទួលបានស្មើប៉ុន្មាន?
		\item ទាញរកតម្លៃ $pH$ ។
	\end{enumerate}
	\item នៅពេល $25mL$ នៃ $0.10mol\cdot L^{-1}~~HBr\left(aq\right)$ ត្រូវបានលាយជាមួយ $25mL$ នៃ $0.20mol\cdot L^{-1}~~KOH\left(aq\right)$ ។ តើ $pH$ នៃសូលុស្យុងចុងបញ្ចប់ស្មើប៉ុន្មាននៅ $25^{0}C$ ។
	\item គេឲ្យផលគុណអុីយ៉ូនិចរបស់ទឹកនៅសីតុណ្ហភាព $\ce{0^\circ C}$ គឺ $\ce{Kw=10^{-15}}$
	\begin{enumerate}[k,2]
		\item គណនា $\ce{pKw}$ របស់ទឹកសុទ្ធនេះ
		\item គណនា $\ce{[H3O+]}$ និង $\ce{OH-}$ របស់ទឹកសុទ្ធ
		\item គណនា $\ce{pH}$ របស់ទឹកសុទ្ធ ។
	\end{enumerate}
	\item តើអ្វីទៅដែលហៅថាប្រតិកម្មស្វ័យអុីយ៉ុងកម្មនៃទឹក? ផលគុណអុីយ៉ុងកម្មនៃទឹក?
	\item តើទំហំ $pH$ និងកំហាប់ $\left[H_{3}O^{+}\right]$ មានទំនាក់ទំនង់គ្នាដូចម្តេច?   
	\begin{center}
		\sffamily\color{black}
		សូមសំណាងល្អ!
	\end{center}\newpage
		\begin{center}
			\sffamily\color{black}
			\circled{០២}\\
			ជំពូក៣~~សូលុយស្យូងទឹក និង~$\ce{pH}$
		\end{center}
	\item គេលាយ $50cm^3$ នៃសូលុយស្យុង $NaOH$ កំហាប់ $C_B=1.4mol.L^{-1}$ និង $50cm^3$ នៃសូលុយស្យុងអាស៊ីត $HCl$ កំហាប់ $C_A=1mol.L^{-1}$ ។
	\begin{enumerate}[k]
		\item តើប្រតិកម្មអ្វីកើតឡើង? ចូរឲ្យសមីការតុល្យការ។
		\item តើសូលុយស្យុងដែលទទួលបានក្រោយប្រតិកម្មស្ថិតក្នុងមជ្ឍដ្ឋានអ្វី?\\
		គណនា $pH$ សូលុយស្យុងដែលទទួលបាននេះ?
	\end{enumerate}
	\item -ក្នុងកែវបេស៊ែរមួយមានសូលុយស្យុងអាស៊ីតក្លរីឌ្រិច($H_3O^+, Cl^-$) នៅកំហាប់ $C_A=1\times10^{-2}M$ និងមាឌ $V_A=20mL$។
	-ក្នុងប៊ុយរ៉ែតក្រិតមួយមានសូលុយស្យុង $NaOH$ កំហាប់ $C_B=1\times10^{-2}M$ និងមាឌ $V_B$ ។\\
	គេបានធ្វើការសំរក់សូលុយស្យុង $NaOH$ ខាងលើនេះទៅក្នុងកែវបេស៊ែរនោះ ។
	\begin{enumerate}[k]
		\item សរសេរសមីការតុល្យការតាងប្រតិកម្មដែលកើតមាន?
		\item គណនា $pH$ សូលុយស្យុងអាស៊ីត $HCl$ មុនពេលសំរក់សូលុយស្យុង $NaOH$ ចូរ?
		\item គណនា $pH$ នៃសូលុយស្យុងដែលទទួលបានក្រោយពេលសំរក់សូលុយស្យុង $NaOH~~10mL$ ។ 
	\end{enumerate}
	\item គេរំលាយឧស្ម័នអ៊ីដ្រូសែនក្លរួ ($HCl$) $1.12L$ ក្នុងទឹកសុទ្ធ$1L$ ។
	\begin{enumerate}[k]
		\item សរសេរសមីការអ៊ីយ៉ុងកម្មនៃ $HCl$ ក្នុងទឹក ។
		\item គណនា $C_A$ កំហាប់ជាម៉ូលនៃសូលុយស្យុងអាស៊ីត $HCl$ ដែលទទួលបាន ?
		\item គេយកសូលុយស្យុងអាស៊ីត $HCl$ នេះ $10mL$ ចាក់ទៅក្នុងសូលុយស្យុង $KOH$ កំហាប់ $C_B=2\times10^{-2}M~;~V_B=25mL$ ។
		\begin{enumerate}[m]
			\item ឲ្យសមីការតុល្យការតាងប្រតិកម្មដែលកើតមានឡើង។
			\item តើសូលុយស្យុងដែលទទួលបានជា អាស៊ីត, បាស ឬណឺត?\\
			កំណតតម្លៃ $pH$ សូលុយស្យុងដែលទទួលបាន
		\end{enumerate}
	\end{enumerate}
	\item គេមានសូលុយស្យុង$HNO_3$ មួយនៅកំហាប់ $C_A=5\times10^{-2}M$ មាឌ $V_A=25cm^3$។
	តើគេត្រូវប្រើសូលុយស្យុង $KOH$ នៅកំហាប់ $C_B=2\times10^{-2}M$ ប៉ុន្មាន $cm^3$ ដើម្បីបន្សាបអាស៊ីត $HNO_3$ ខាងលើនេះឲ្យសាប់អស់? 
	\item សូ.អាស៊ីតក្លរីឌ្រិច ($HCl$) មួយមានកំហាប់ $C_A=5\times10^{-3}M$ ។\\
	គណនា $pH$ នៃសូលុយស្យុងនេះ? គេឲ្យ៖ $log5=0.7$\\
	ចម្លើយ៖ $pH=2.3$ 
	\item គេរំលាយឧស្ម័នអ៊ីដ្រូសែន $HCl$ $0.56L$ ទៅក្នុងទឹកសុទ្ធគេទទួលបានសូលុយស្យុងអាស៊ីតក្លរីឌ្រិច $1L$ ។
	\begin{enumerate}[k]
		\item គណនាកំហាប់ $C_A$ នៃសូលុយស្យុងដែលទទួលបាន? 
		\item គណនា $pH$ នៃសូលុយស្យុង? គេឲ្យ៖ $V_m=22.4L/mol~,~log25=1.4$ (ចម្លើយ $C_A=25\times10^{-3}~;~pH=1.6$)
	\end{enumerate}
	\item គេរំលាយក្រាម $NaOH~~3.2g$ ក្នុងទឹកសុទ្ធ $500mL$ នៅ $25^\circ C$ ។
	\begin{enumerate}[k]
		\item គណនា $C_B$ កំហាប់ជាម៉ូលនៃសូលុយស្យុង $KOH$ ទទួលបាន?
		\item កំណត់តម្លៃ $pH$ នៃសូលុយស្យុងខាងលើនេះ? \\ចម្លើយ $C_B=4\times10^{-1}~~;~~pH=13.6$
	\end{enumerate}
	\item គេរំលាយក្រាម $KOH~~0.2mol$ ក្នុងទឺកសុទ្ធ គេទទួលបានសូលុយស្យុង $KOH~~500mL$ នៅ $25^\circ C$ ។
	\begin{enumerate}[k]
		\item គណនាកំហាប់ជាម៉ូលនៃសូលុយស្យុងនេះ?
		\item កំណត់ $pH$ នៃសូលុយស្យុងខាងលើនេះ? \\ចម្លើយ $C_B=4\times10^{-1} M~~,~~pH=13.6$ 
	\end{enumerate}
	\item គេចង់ធ្វើសូលុយស្យុងស៊ូត ($NaOH$) មួយដែលមាន $pH=12.5$ ។
	\begin{enumerate}[k]
		\item គណនា $[OH^-]$ ដែលមានក្នុងសូលុយស្យុងនេះ?
		\item តើគេយក $NaOH$ ប៉ុន្មានក្រាមដើម្បីធ្វើសូលុយស្យុង $NaOH$ នេះ $1L$ ?\\ ចម្លើយ $[OH^-]=3.2\times10^{-2}M;~m=1.28g$
	\end{enumerate}  
	\item គេរំលាយក្រាមស៊ូតកាត់ ($NaOH$) ទៅក្នុងទឹកសុទ្ធដើម្បីទទួលបានសូលុយស្យុងស៊ូត ($S_1$) មួយមានកំហាប់ស្មើនឹង $4\times10^{-2}mol.L^{-1}$ និងមានមាឌចំនួន $200mL$ ។
	\begin{enumerate}[k]
		\item សរសេរសមីការអ៊ីយ៉ុងកម្មនៃសូដ្យូមអ៊ីដ្រុកស៊ីតក្នុងទឹកសុទ្ធ។
		\item គណនាម៉ាសក្រាមស៊ូតត្រូវរំលាយ។
		\item គណនា $pH$ នៃសូលុយស្យុងស៊ូតខាងលើ ។ \\គេឲ្យ៖ $Na=23, O=16, H=1, log4=0.6$
	\end{enumerate}
	\item នៅពេល $25 mL$ នៃ $0.10mol.L^{-1}$ $HBr$ ត្រូវបានលាយជាមួយ $25mL$ នៃ $0.20mol.L^{-1} KOH$ ។ តើ $pH$ នៃសូលុយស្យុងចុងបញ្ចប់ស្មើប៉ុន្មាននៅ $25^\circ C$ ?
	\item ភាគសំណាក $40.0mL$ នៃ $0.25M~~KOH$ ត្រូវបានបន្ថែមទៅក្នុង $60.0mL$ នៃ $0.15 M$ សូលុយស្យុង $Ba(OH)_2$ ។
	\begin{enumerate}[k]
		\item តើកំហាប់ជាម៉ូល $[OH^-]$ ក្នុងសូលុយស្យុងទទួលបានស្មើប៉ុន្មាន?
		\item ទាញរកតម្លៃ $pH$ ។
	\end{enumerate}
	\item គេចង់ធ្វើសូលុយស្យុងអាសុីតក្លរីឌ្រិច $\ce{(HCl)}$ ដែលមានមាឌ $\ce{700cm^3}$ កំហាប់ $2 \times 10^{-2} \ce{mol\cdot L^{-1}}$ ។
	\begin{enumerate}[k]
		\item សរសេរសមីការតាងប្រតិកម្មរវាងអាសុីតនេះជាមួយទឹក
		\item គណនាមាឌឧស្ម័ន $HCl$ ចាំបាច់ដែលប្រើ។ បើ $\ce{Vm=24 L\cdot mol^{-1}}$
		\item គណនា $\ce{pH}$ នៃសូលុយស្យុងនេះ ។
	\end{enumerate}
	\begin{center}
		\sffamily\color{black}
		សូមសំណាងល្អ!
	\end{center}\newpage
	\begin{center}
		\sffamily\color{black}
		\circled{០៣}\\
		ជំពូក៣~~សូលុយស្យូងទឹក និង~$\ce{pH}$
	\end{center}
	\item គេចង់ទង្វើសូលុយស្យុងមួយដែលមាន $pH=10.6$ ដោយការរំលាយក្រាម $Ba\left(OH\right)_{2}$ ទៅក្នុងទឹក។
	\begin{enumerate}[k]
		\item គណនាកំហាប់ជាម៉ូលជាម៉ូលនៃសូលុយស្យុងទទួលបាន។
		\item គណនាម៉ាស $Ba\left(OH\right)_2$ ចាំបាច់ដើម្បីទង្វើសូលុយស្យុងខាងលើ $250mL$ ។
	\end{enumerate}
	\item គេរំលាយ $Ba\left(OH\right)_{2}$ ទៅក្នុងទឹកគេទទួលបានសូលុយស្យុង $S_{1}$ ដែលមានមាឌ $1L$។ បើគេយក $10mL$ នៃសូលុយស្យុង $S_{1}$ ទៅលាយជាមួយ $40mL$ នៃទឹកសុទ្ធ គេទទួលបានសូលុយស្យុង $S_{2}$ ដែលមាន $pH=9.4$ ។ 
	\begin{enumerate}[k]
		\item គណនា $pH$ នៃសូលុយស្យុង $S_{1}$ ។
		\item គណនាម៉ាស $Ba\left(OH\right)_{2}$ ចាំបាច់ដែលត្រូវប្រើដើម្បីទង្វើសូលុយស្យុង $S_{1}$ ។
	\end{enumerate}
	\item គេចង់ទង្វើ $100mL$ នៃសូលុស្យុង $HCl$ ដែលមាន $pH=3.2$ ដោយការរំលាយឧស្ម័នអុីដ្រូសែនក្លរួក្នុងទឹក ។
	\begin{enumerate}[k]
		\item គណនាមាឌឧស្ម័នអុីដ្រូសែនក្លរួដែលត្រូវប្រើនៅលខ្ឌ័ណធម្មតា
		\item គេលាយសូលុយស្យុងដែលទទួលបានខាងលើជាមួយ $100mL$ នៃសូលុយស្យុង $NaOH$ ដែលមាន $pH=12.5$\\
		ចូរកំណត់ $pH$ នៃល្បាយសូលុយស្យុងដែលទទួលបាន។
	\end{enumerate}
	\item គេលាយ $50mL$ នៃសូលុយស្យុង $HCl$ ដែលមាន $pH=2$ ជាមួយ $70mL$ នៃសូលុយស្យុង $NaOH$ គេទទួលបានល្បាយសូលុយស្យុងដែលមាន $pH=4.4$។ គណនា $pH$ នៃសូលុយស្យុង $NaOH$ ខាងលើ ។
	\item គេចង់ទង្វើ $100mL$ នៃសូលុយស្យុងមួយដែលមាន $pH=9$ ដោយការលាយបញ្ចូលនៃសូលុយស្យុង $HCl~~pH=4$ ជាមួយសូលុយស្យុង $NaOH~~pH=10$។ ចូរកំណត់មាឌសូលុយស្យុងនីមួយៗដែលត្រូវប្រើ។
	\item ចូរគណនាកំហាប់ $\left[H_{3}O^{+}\right]$ និង $\left[OH^{-}\right]$ នៅក្នុងសូលុយស្យុងមួយដែលត្រូវបានគេរៀបចំចេញពី $0.025mol$ នៃបារ្យូមអុីដ្រុកសុីត $\left(Ba\left(OH\right)_{2}\right)$ ដែលត្រូវបំបែកក្នុងទឹក $105mL$ ។
	\item ចូរគណនាកំហាប់ $\left[H_{3}O^{+}\right]$ និង $\left[OH^{-}\right]$ នៅក្នុងសូលុយស្យុងមួយដែលត្រូវបានគេរៀបចំចេញពី $0.005mol$ នៃអាស៊ីតក្លរីឌ្រិច $\left(HCl\right)$ ដែលត្រូវបំបែកក្នុងទឹក $1L$ ។
	\item ចូរគណនាកំហាប់ $\left[H_{3}O^{+}\right]$ និង $\left[OH^{-}\right]$ នៅក្នុងសូលុយស្យុងមួយដែលត្រូវបានគេរៀបចំចេញពី $10g$ នៃសូល្យូមអុីដ្រុកសុីត $\left(NaOH\right)$ ដែលត្រូវបំបែកក្នុងទឹក $375L$ ។
	\item គេរំលាយឧស្ម័នអុីដ្រូសែនក្លរួ $2.5L$ ទៅក្នុងទឹក $2.5L$ គេទទួលបានសូលុយស្យុង អាស៊ីតក្លរីឌ្រិចដែលមាន $pH=1.6$។
	\begin{enumerate}[k]
		\item តើអាសុីតក្លរីឌ្រិចជាអាសុីតខ្លាំង ឬខ្សោយ?
		\item សរសេរសមីការតាងប្រតិកម្មដែលកើតមាន ។ $\left(V_{m}=25mol\cdot L^{-1}\right)$
	\end{enumerate}\newpage
	\item គេរំលាយ $2g$ នៃសូលុយស្យុងសូល្យូមអុីដ្រុកសុីតសុទ្ធទៅក្នុងទឹកគេទទួលបាន សូលុយស្យុងស៊ូត $1L$ និងមាន $pH=12.7$ ។
	\begin{enumerate}[k]
		\item តើសូលុយស្យុងស៊ួតជាបាសខ្លាំង ឬខ្រោយ?
		\item សរសេរសមីការតាងប្រតិកម្មដែលកើតមាន ។
	\end{enumerate}
	\item សូលុយស្យុងស៊ូតមួយមានកំហាប់ $0.2M$ និងមាឌ $50mL$។ នៅសីតុណ្ហភាព $25^{o}C$ គេយកសូលុយស្យុងនេះទៅពង្រាវដោយថែមទឹកបិតដើម្បីទទួលបានសូលុយស្យុងថ្មីមួយមាន $pH=12$ ។
	\begin{enumerate}[k]
		\item ចូរគណនា $pH$ នៃសូលុយស្យុងស៊ូតមុនថែមទឹក។
		\item ចូរគណនាមាឌទឹកដែលត្រូវថែម។ គេឲ្យ៖ $log2=0.3$
	\end{enumerate}
	\item គេយកសូលុយស្យុង $HCl$ មានកំហាប់ $0.1M$ និងមាឌ $20mL$ ចាក់ចូរទៅក្នុងសូលុយស្យុង $HNO_{3}$ ដែលមានកំហាប់ $0.2M$ និងមាឌ $40mL$ គេទទួលបានល្បាយសូលុយស្យុងមួយថ្មី។
	\begin{enumerate}[k]
		\item ចូរសរសេរសមីការតាងប្រតិកម្មនៃ $HCl$ និង $HNO_{3}$ ជាមួយទឹក។
		\item ចូរគណនា $pH$ នៃសូលុយស្យុងនីមួយៗមុនពេលចាក់ចូរគ្នា។
		\item ចូរគណនា $pH$ នៃល្បាយសូលុយស្យុងថ្មី។ $log2=0.3~;~log1.6=0.2$ ។
	\end{enumerate}
	\item គេបំបែក $\ce{0.2g}$ នៃសូល្យូមអុីដ្រុកសុីតទៅក្នុងទឹកសុទ្ធគេទទួលបាន សូលុយស្យុងមានមាឌ $\ce{2L}$ ។
	\begin{enumerate}[k]
		\item សរសេរសមីការតាងប្រតិកម្មនៃការបំបែកអង្គធាតុរឹងក្នុងទឹក 
		\item គណនា $\ce{pH}$ នៃសូលុយស្យុងនេះ
		\item គណនាកំហាប់ប្រភេទគីមីនីមួយៗដែលមានវត្តមាននៅសូលុយស្យុង 
		\item គណនាមាឌទឹកដែលត្រូវចាក់ចូរទៅក្នុង $\ce{20 mL}$ នៃសូលុយស្យុងខាងលើដើម្បីទទួលបានសូលុយស្យុងថ្មីមាន $\ce{pH=11}$ ។
	\end{enumerate}
	\item គេរំលាយឧស្ម័នអុីដ្រូសែនប្រូមួ $\ce{(HBr)}$ ចំនួន $\ce{1.2L}$ ទៅក្នុងទឹក $\ce{5L}$ គេទទួលបានសូលុយស្យុងដែលត្រូវនឹងវា។
	\begin{enumerate}[k]
		\item សរសេរមីការតាងប្រតិកម្មដែលកើតមាន
		\item គណនាកំហាប់ប្រភេទគីមីដែលមានវត្តមាននៅក្នុងសូលុយស្យុង
		\item គណនា $\ce{pH}$ នៃសូលុយស្យុងដែលទទួលបាន។ បើ $\ce{Vm = 24L\cdot mol^{-1}}$
	\end{enumerate}
	\item សូលុយស្យុងអាសុីតក្លរិច $\ce{(HCl)}$ មួយមាន $\ce{pH=3.4}$ ចំនួន $\ce{10mL}$ ។ គេចាក់បន្ថែម $\ce{90mL}$ នៃសូលុយស្យុង $\ce{KCl}$ ដែលមានកំហាប់ $\ce{0.1M}$ ។ គណនា $\ce{pH}$ នៃល្បាយដែលទទួលបាន ។
	\item គេយក $\ce{20mL}$ នៃសូលុយស្យុងស៊ូតដែលមានកំហាប់ $\ce{0.3M}$ ទៅលាយជាមួយ $\ce{30mL}$ នៃសូលុយស្យុងស៊ូតមួយទៀតដែលមានកំហាប់ $\ce{0.1M}$។ គណនា $\ce{pH}$ នៃល្បាយសូលុយស្យុងដែលទទួលបាន ។
	\begin{center}
		\sffamily\color{black}
		សូមសំណាងល្អ!
	\end{center}\newpage
	\begin{center}
		\sffamily\color{black}
		\circled{០៤}\\
		ជំពូក៣~~សូលុយស្យូងទឹក និង~$\ce{pH}$
	\end{center}
	\item គេរំលាយឧស្ម័នអុីដ្រូសែនក្លរួចំនួន $\ce{0.56L}$ ក្នុងទឹកបិតគេទទួលបានសូលុយស្យុងអាសុីតក្លរីឌ្រិចដែលមានមាឌ $\ce{500mL}$។
	\begin{enumerate}[k]
		\item សរសេរសមីការតាងប្រតិកម្មរវាង $\ce{HCl}$ ជាមួយទឹក 
		\item គណនាកំហាប់ប្រភេទគីមីដែលមានវត្តមានក្នុងសូលុយស្យុងនេះ
		\item គណនា $\ce{pH}$ របស់សូលុយស្យុងអាសុីតនេះ។ បើ $\ce{Vm=22.4L\cdot mol^{-1}}$
	\end{enumerate}
	\item គេរំលាយក្រាមសូល្យូមអុីដ្រុកសុីត $\ce{(NaOH)}$ ចំនួន $\ce{1.6g}$ ក្នុងទឹកគេទទួលបានសូលុយស្យុង $\ce{S1}$ ចំនួន $\ce{250mL}$។ គេបន្ថែម សូលុយស្យុងប៉ូតាស្យូមអុីដ្រុកសុីត $\ce{(KOH)}$~ $\ce{S2}$ ដែលមាន $\ce{pH=12}$ ចំនួន $\ce{500mL}$ ទៅលើសូលុយស្យុង $\ce{S1}$ គេទទួលបានសូលុយស្យុង $\ce{S3}$។
	\begin{enumerate}[k]
		\item គណនាបរិមាណអុីយ៉ុងអុីដ្រុកសុីតដែលមាននៅក្នុងសូលុស្យុង $\ce{S3}$
		\item គណនា $\ce{pH}$ នៃសូលុយស្យុង $\ce{S3}$
	\end{enumerate}
	\item សូលុយស្យុងអាសុីតក្លរីឌ្រិចមួយមាន $\ce{pH=2}$ (សូលុយស្យុង $\ce{S1}$)។ សូលុយស្យុងអាសុីតក្លរីឌ្រិចមួយទៀតមាន $\ce{pH=4}$ (សូលុយស្យុង $\ce{S2}$)។ គេយក $\ce{50mL}$ នៃសូលុយស្យុង $\ce{S1}$ ទៅលាយជាមួយ $\ce{50mL}$ នៃសូលុយស្យុង $\ce{S2}$។\\
	 គណនា $\ce{pH}$ នៃល្បាយសូលុយស្យុងដែលទទួលបានក្រោយការលាយនេះ ។
	 \item គេឲ្យឧស្ម័នអុីដ្រូសែនក្លរួចំនួន $6\times10^{-3}\ce{mol}$ ទៅក្នុងទឹក $\ce{2L}$។
	 \begin{enumerate}[k]
	 	\item គណនា $\ce{pH}$ នៃសូលុយស្យុងអាសុីតដែលទទួលបាន
	 	\item បើគេយក $\ce{100mL}$ នៃសូលុយស្យុងអាសុីតខាងលើទៅចាក់បញ្ចូលក្នុងសូលុយស្យុងអាសុីតនីឌ្រិច $\ce{(HNO3)}$ ដែលមានកំហាប់ $5\times10^{-3}\ce{mol\cdot L^{-1}}$ ចំនួន $\ce{100mL}$។ គណនា $\ce{pH}$ នៃល្បាយសូលុយស្យុងថ្មី ។
	 	\item គណនាមាឌទឹកដែលត្រូវថែមទៅលើសូលុយស្យុងថ្មីដើម្បីទទួលបានសូលុយស្យុងថ្មីមួយទៀតមាន $\ce{pH=3}$ ។
	 \end{enumerate}
 	\item គេឲ្យឧស្ម័នអុីដ្រូសែនក្លរួ $\ce{5L}$ ឆ្លងកាត់ទឹក $\ce{2L}$ គេទទួលបានសូលុយស្យុងដែលមាន $\ce{pH=1}$។
 	\begin{enumerate}[k]
 		\item តើសូលុយស្យុងអាសុីតក្លរីឌ្រិចនេះជាអាសុីតខ្លាំង ឬខ្សោយ?
 		\item សរសេរសមីការអាសុីតនេះជាមួយទឹក
 		\item គណនាមាឌទឹកដែលត្រូវប្រើដើម្បីបន្ថែមទៅលើ $\ce{50mL}$ នៃសូលុយស្យុងអាសុីតខាងលើដើម្បីឲ្យគេទទួលបានសូលុយស្យុងអាសុីតថ្មីមាន $\ce{pH=1.3}$ ។
 	\end{enumerate}
 	\item គេឲ្យ $\ce{Kw=2.5}\times10^{-13}$ នៅសីតុណ្ហភាព $\ce{80^\circ C}$។ នៅសីតុណ្ហភាពនេះ សូលុយស្យុងទឹកមួយមាន $\ce{pH=6.5}$។\\
 	តើសូលុយស្យុងនេះមានធម្មជាតិជា អាសុីត បាស ឬណឺត?
 	\item ចូរសរសេររូបមន្តដើម្បីគណនាកំហាប់ អុីយ៉ុងអុីដ្រុកសុីត $\ce{[OH^-]}$ និង អុីយ៉ុងអុីដ្រូញ៉ូម $\ce{[H3O+]}$ នីមួយៗ ឧ្យបាន$\ce{3}$យ៉ាង។\newpage
 	\item គេយក $\ce{10g}$ នៃល្យាយសូល្យូមក្លរួ និងស៊ូត $\ce{(NaOH)}$ ទៅរំលាយក្នុងទឹកគេទទួលបាន $\ce{1L}$ សូលុយស្យុងមួយដែលមាន $\ce{pH=13}$។
 	\begin{enumerate}[k]
 		\item គណនាសមាសភាពជាម៉ាសនៃល្បាយដើម
 		\item គេយកសូលុយស្យុងខាងលើទៅពង្រាវ $\ce{100}$ដង។ គណនា $\ce{pH}$ នៃល្បាយសូលុយស្យុងដែលទទួលបាន។
 	\end{enumerate}
 	\item សូលុយស្យុងអាសុីតផួរិច $\ce{(H2SO4)}$ មួយមាន $\ce{pH=3.7}$។ គេឧបមាថា អាសុីតស៊ុលផួរិចបំបែកទាំងស្រុងជា $\ce{H3O+}$ និង $\ce{SO4^{2-}}$។
 	\begin{enumerate}[k]
 		\item ចូរសរសេរសមីការតាងប្រតិកម្មរវាងអាសុីតនេះជាមួយទឹក
 		\item គណនាកំហាប់ប្រភេទគីមីដែលមានវត្តមាននៅក្នុងសូលុយស្យុងនេះ
 		\item ទាញរកកំហាប់ដើមនៃសូលុយស្យុងអាសុីតនេះ។
 	\end{enumerate}
 	\item គេរៀបចំសូលុយស្យុងទីផ្សារនៃអាសុីតក្លរីឌ្រិច $\ce{(HCl)}$ មួយដែលមាន $\ce{35\%}$ ជាម៉ាសនិងមានដង់សុីតេធៀបនឹងទឹក $\ce{d=1.15}$។
 	\begin{enumerate}[k]
 		\item គណនាកំហាប់នៃសូលុយស្យុងទីផ្សារនេះ
 		\item គេចង់រៀបចំ $\ce{1L}$ នៃសូលុយស្យុងអាសុីតក្លរីឌ្រិចដែលមានកំហាប់ $2\times10^{-2}\ce{M}$។\\ គណនាមាឌសូលុយស្យុងទីផ្សារដែលត្រូវប្រើ។
 	\end{enumerate}
 	\item គេរៀបចំសូលុយស្យុងទីផ្សារនៃសូលុយស្យុងស៊ូត $\ce{(NaOH)}$ មួយដែលមាន $\ce{35\%}$ ជាម៉ាសនិងមានដង់សុីតេធៀបនឹងទឹក $\ce{d=1.38}$។
 	\begin{enumerate}[k]
 		\item គណនាកំហាប់នៃសូលុយស្យុងទីផ្សារនេះ
 		\item គណនាមាឌ $\ce{V1}$ នៃសូលុយស្យុងនេះដែលត្រូវគេលាយដោយទឹកសុទ្ធ ដើម្បីទទួលបាន $\ce{1L}$ នៃសូលុយស្យុងមាន $\ce{pH=12.4}$
 		\item គេចាក់ $\ce{5mL}$ នៃសូលុយស្យុងទីផ្សារនេះទៅក្នុងទឹកបាន $\ce{1L}$។
 		គណនា $\ce{pH}$ សូលុយស្យុងដែលទទួលបាន។
 	\end{enumerate}
 	\item ក្រោយពីការពង្រាវ $\ce{50mL}$ នៃសូលុយស្យុងស៊ូតដែលមាន $\ce{pH=12}$ គេទទួលបានសូលុយស្យុងថ្មីមាន $\ce{pH^{'}=10.7}$។
 	\begin{enumerate}[k]
 		\item តើគេត្រូវពង្រាវសូលុយស្យុងដើមប៉ុន្មានដង?
 		\item គេថែមស៊ូត $\ce{m_{(g)}}$ ចូរក្នុងសូលុយស្យុងក្រោយពង្រាវនេះ គេទទួលបានសូលុយស្យុងថ្មីមួយទៀតមាន $\ce{pH=11}$។ គណនាម៉ាសស៊ូត $\ce{m_{(g)}}$ ។
 	\end{enumerate}
	\begin{center}
		\sffamily\color{black}
		សូមសំណាងល្អ!
	\end{center}\newpage
\end{enumerate}	
\end{document}