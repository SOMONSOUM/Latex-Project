\documentclass[12pt, a4paper]{article}
\usepackage[landscape, top=0.5cm, left=1cm, bottom=1.5cm, right=1.5cm]{geometry}
%%import package named hightest
\usepackage{hightest}
\usepackage{comment}

\setlength{\columnseprule}{1pt}
\def\columnseprulecolor{\color{magenta}}
%\usepackage{mathpazo}% change math font
%\usepackage[no-math]{fontspec}% font specfication
\header{រៀនគណិតវិទ្យាទាំងអស់គ្នា}{ គីមីវិទ្យា}{១០/០៣/២០១៨}
\footer{រៀបរៀង និងបង្រៀនដោយ ស៊ុំ សំអុន}{ទំព័រ \thepage}{០៩៦ ៩៤០ ៥៨៤០}
\everymath{\protect\displaystyle\protect\color{black}}
\begin{document}
	\begin{center}
		\sffamily\color{black}
		\circled{០១}\\
		កំណត់ $pH$ នៃសូលុយស្យូង (លំហាត់សុទ្ធ)
	\end{center}
\maketitle
\begin{enumerate}[m, 2]
		\item គេលាយ $50cm^3$ នៃសូលុយស្យុង $NaOH$ កំហាប់ $C_B=1.4mol.L^{-1}$ និង $50cm^3$ នៃសូលុយស្យុងអាស៊ីត $HCl$ កំហាប់ $C_A=1mol.L^{-1}$ ។
		\begin{enumerate}[k]
			\item តើប្រតិកម្មអ្វីកើតឡើង? ចូរឲ្យសមីការតុល្យការ។
			\item តើសូលុយស្យុងដែលទទួលបានក្រោយប្រតិកម្មស្ថិតក្នុងមជ្ឍដ្ឋានអ្វី?\\
			គណនា $pH$ សូលុយស្យុងដែលទទួលបាននេះ?
		\end{enumerate}
		\item -ក្នុងកែវបេស៊ែរមួយមានសូលុយស្យុងអាស៊ីតក្លរីឌ្រិច($H_3O^+, Cl^-$) នៅកំហាប់ \\$C_A=1\times10^{-2}M$ និងមាឌ $V_A=20mL$។
		-ក្នុងប៊ុយរ៉ែតក្រិតមួយមានសូលុយស្យុង $NaOH$ កំហាប់ $C_B=1\times10^{-2}M$ និងមាឌ $V_B$ ។\\
		គេបានធ្វើការសំរក់សូលុយស្យុង $NaOH$ ខាងលើនេះទៅក្នុងកែវបេស៊ែរនោះ ។
		\begin{enumerate}[k]
			\item សរសេរសមីការតុល្យការតាងប្រតិកម្មដែលកើតមាន?
			\item គណនា $pH$ សូលុយស្យុងអាស៊ីត $HCl$ មុនពេលសំរក់សូលុយស្យុង $NaOH$ ចូរ?
			\item គណនា $pH$ នៃសូលុយស្យុងដែលទទួលបានក្រោយពេលសំរក់សូលុយស្យុង $NaOH~~10mL$ ។ 
		\end{enumerate}
		\item គេរំលាយឧស្ម័នអ៊ីដ្រូសែនក្លរួ ($HCl$) $1.12L$ ក្នុងទឹកសុទ្ធ$1L$ ។
		\begin{enumerate}[k]
			\item សរសេរសមីការអ៊ីយ៉ុងកម្មនៃ $HCl$ ក្នុងទឹក ។
			\item គណនា $C_A$ កំហាប់ជាម៉ូលនៃសូលុយស្យុងអាស៊ីត $HCl$ ដែលទទួលបាន ?
			\item គេយកសូលុយស្យុងអាស៊ីត $HCl$ នេះ $10mL$ ចាក់ទៅក្នុងសូលុយស្យុង $KOH$ កំហាប់ $C_B=2\times10^{-2}M~;~V_B=25mL$ ។
			\begin{enumerate}[m]
				\item ឲ្យសមីការតុល្យការតាងប្រតិកម្មដែលកើតមានឡើង។
				\item តើសូលុយស្យុងដែលទទួលបានជា អាស៊ីត, បាស ឬណឺត?\\
				កំណតតម្លៃ $pH$ សូលុយស្យុងដែលទទួលបាន
			\end{enumerate}
		\end{enumerate}
		\item គេមានសូលុយស្យុង$HNO_3$ មួយនៅកំហាប់ $C_A=5\times10^{-2}M$ មាឌ \\$V_A=25cm^3$។\\
		តើគេត្រូវប្រើសូលុយស្យុង $KOH$ នៅកំហាប់ $C_B=2\times10^{-2}M$ ប៉ុន្មាន $cm^3$ ដើម្បីបន្សាបអាស៊ីត $HNO_3$ ខាងលើនេះឲ្យសាប់អស់? 
		\item សូ.អាស៊ីតក្លរីឌ្រិច ($HCl$) មួយមានកំហាប់ $C_A=5\times10^{-3}M$ ។\\
		គណនា $pH$ នៃសូលុយស្យុងនេះ? គេឲ្យ៖ $log5=0.7$\\
		ចម្លើយ៖ $pH=2.3$ 
		\item គេរំលាយឧស្ម័នអ៊ីដ្រូសែន $HCl$ $0.56L$ ទៅក្នុងទឹកសុទ្ធគេទទួលបានសូលុយស្យុងអាស៊ីតក្លរីឌ្រិច $1L$ ។
		\begin{enumerate}[k]
			\item គណនាកំហាប់ $C_A$ នៃសូលុយស្យុងដែលទទួលបាន? 
			\item គណនា $pH$ នៃសូលុយស្យុង? គេឲ្យ៖ $V_m=22.4L/mol~,~log25=1.4$ (ចម្លើយ $C_A=25\times10^{-3}~;~pH=1.6$)
		\end{enumerate}
		\begin{center}
			To be continued\\
			\sffamily\color{black}
			សូមសំណាងល្អ!
		\end{center}\newpage
		\vspace{-1em}
		\begin{center}
		\sffamily\color{black}
		\circled{០២}\\
		កំណត់ $pH$ នៃសូលុយស្យូង (លំហាត់សុទ្ធ)
		\end{center}
		\item គេរំលាយក្រាម $NaOH~~3.2g$ ក្នុងទឹកសុទ្ធ $500mL$ នៅ $25^\circ C$ ។
		\begin{enumerate}[k]
			\item គណនា $C_B$ កំហាប់ជាម៉ូលនៃសូលុយស្យុង $KOH$ ទទួលបាន?
			\item កំណត់តម្លៃ $pH$ នៃសូលុយស្យុងខាងលើនេះ? \\ចម្លើយ $C_B=4\times10^{-1}~~;~~pH=13.6$
		\end{enumerate}
		\item គេរំលាយក្រាម $KOH~~0.2mol$ ក្នុងទឺកសុទ្ធ គេទទួលបានសូលុយស្យុង $KOH~~500mL$ នៅ $25^\circ C$ ។
		\begin{enumerate}[k]
			\item គណនាកំហាប់ជាម៉ូលនៃសូលុយស្យុងនេះ?
			\item កំណត់ $pH$ នៃសូលុយស្យុងខាងលើនេះ? \\ចម្លើយ $C_B=4\times10^{-1} M~~,~~pH=13.6$ 
		\end{enumerate}
		\item គេចង់ធ្វើសូលុយស្យុងស៊ូត ($NaOH$) មួយដែលមាន $pH=12.5$ ។
		\begin{enumerate}[k]
			\item គណនា $[OH^-]$ ដែលមានក្នុងសូលុយស្យុងនេះ?
			\item តើគេយក $NaOH$ ប៉ុន្មានក្រាមដើម្បីធ្វើសូលុយស្យុង $NaOH$ នេះ $1L$ ?\\ ចម្លើយ $[OH^-]=3.2\times10^{-2}M;~m=1.28g$
		\end{enumerate}  
		\item គេរំលាយក្រាមស៊ូតកាត់ ($NaOH$) ទៅក្នុងទឹកសុទ្ធដើម្បីទទួលបានសូលុយស្យុងស៊ូត ($S_1$) មួយមានកំហាប់ស្មើនឹង $4\times10^{-2}mol.L^{-1}$ និងមានមាឌចំនួន $200mL$ ។
		\begin{enumerate}[k]
			\item សរសេរសមីការអ៊ីយ៉ុងកម្មនៃសូដ្យូមអ៊ីដ្រុកស៊ីតក្នុងទឹកសុទ្ធ។
			\item គណនាម៉ាសក្រាមស៊ូតត្រូវរំលាយ។
			\item គណនា $pH$ នៃសូលុយស្យុងស៊ូតខាងលើ ។ \\គេឲ្យ៖ $Na=23, O=16, H=1, log4=0.6$
		\end{enumerate}
		\item នៅពេល $25 mL$ នៃ $0.10mol.L^{-1}$ $HBr$ ត្រូវបានលាយជាមួយ $25mL$ នៃ $0.20mol.L^{-1} KOH$ ។ តើ $pH$ នៃសូលុយស្យុងចុងបញ្ចប់ស្មើប៉ុន្មាននៅ $25^\circ C$ ?
		\item ភាគសំណាក $40.0mL$ នៃ $0.25M KOH$ ត្រូវបានបន្ថែមទៅក្នុង $60.0mL$ នៃ $0.15 M$ សូលុយស្យុង $Ba(OH)_2$ ។
		\begin{enumerate}[k]
			\item តើកំហាប់ជាម៉ូល $[OH^-]$ ក្នុងសូលុយស្យុងទទួលបានស្មើប៉ុន្មាន?
			\item ទាញរកតម្លៃ $pH$ ។
		\end{enumerate}
	\begin{center}
		To be continued\\
		\sffamily\color{black}
		សូមសំណាងល្អ!
	\end{center}\newpage
\end{enumerate}
	
\end{document}