\documentclass{beamer}
\mode<presentation>{
\usetheme{Madrid}
\usefonttheme[onlymath]{serif}
%\usecolortheme{beaver}
}
%\usepackage[utf8]{inputenc}
%\usepackage{default}
\usepackage{varwidth}
\usepackage{pgfplots}
\pgfplotsset{/pgf/number format/use comma,compat=newest}
\usepackage{color}
\usepackage{amsmath,amsfonts,amssymb}
\usepackage{hyperref}
\usepackage{tikz}
\usepackage{enumerate}
\usepackage{fontspec}
\newcommand{\en}{\fontspec[Script=Khmer]{Ubuntu}}
\newcommand{\Kml}{\fontspec[Script=Khmer]{Khmer OS Muol Light}}
\newcommand{\Kmp}{\fontspec[Script=Khmer]{Khmer  OS Muol Pali}}
\newcommand{\Kh}{\fontspec[Script=Khmer]{Ang DaunTep}}
\newcommand{\Kos}{\fontspec[Script=Khmer]{Khmer OS }}
\title{ {\Kml គណិតវិទ្យា ថ្នាក់ទី១២}}\vspace{1cm}
\author{ \bf {\Kmp បង្រៀនដោយ: ស្រ៊ុន ប៉េងហួរ}}
%\institute[]{\includegraphics[height=1in]{logo.png}{\vspace{.3cm}}\\
%\bf Chapter 2: Analytic Functions {\vspace{.3cm}} \\ Lecture by
%: {\vspace{.3cm}}\\ {\bf Ph.D SEAM NGONN }}
\date{\en \today} 

\begin{document}
\begin{frame}
\titlepage
\begin{center}
\Kmp វិញ្ញាសាត្រៀមប្រលងឆមាសទី ២
\end{center}
\end{frame}
\begin{frame}
\frametitle{\Kml វិញ្ញាសាទី១  ត្រៀបប្រលងឆមាសទី ២}
\begin{block}{\centering \Kml លំហាត់ទី ១}
\Kh គណនាអាំងតេក្រាល $I=\frac{1}{e^x+1}$ ដែល $I(0)=\ln 4$
\end{block}
\pause
	\begin{center}
	\Kml ដំណោះស្រាយ
	\end{center}
\pause
	\begin{flalign*}\quad 
	\text{\Kh គេមាន}\quad I=\int\frac{1}{e^x+1}dx
	&=\int \frac{1+e^x-e^x}{e^x+1}x&\\
	&=\int\left(1-\frac{e^x}{e^x+1}\right)dx&\\
	&=\int dx-\int\frac{e^x}{e^x+1}dx&\\
	&=x-\ln \left|e^x+1\right|+c&\\
	\end{flalign*}
\end{frame}
\begin{frame}
	\begin{flalign*}\qquad\qquad
	\text{\Kh ដោយ}\;\; I(0)=\ln 4\;\; \text{\Kh តែ}\;\; I(0)&=0-\ln2+c&\\
	&=-\ln 2+c&\\
	\text{\Kh នាំអោយ}\;\; -\ln 2+c&=\ln 4&\\
	\text{\Kh នោះ}\;\; c&=\ln 4+\ln 2\\
	&=2\ln 2+\ln2&\\
	&=3\ln 2
	\end{flalign*}
\pause
	\qquad\qquad\Kh ដូចនេះ $I=\int \frac{1}{e^x+1}dx=x-\ln\left|e^x+1\right|+3\ln 2$
\end{frame}
\begin{frame}
\frametitle{\Kml វិញ្ញាសាទី១  ត្រៀបប្រលងឆមាសទី ២}
\begin{block}{\centering \Kml លំហាត់ទី ២}
\Kh គេអោយសមីការឌីផេរ៉ងស្យែល  $(E):\;y''+2y'+y=x^2+2x-2$
	\begin{enumerate}
	\item[\Kh ក.] រកពហុធាដឺក្រេទីពីរ $P(x)$ ដែលជាចម្លើមពិសេសមួយនៃ $(E)$  
	\item[\Kh ខ.] ដោះស្រាយសមីការ $(E'):\;y''+2y'+y=0$  រួចទាញរកអនុគមន៍   $f$ ដែលជាចម្លើយទូទៅនៃសមីការ $(E)$
	\item[\Kh គ.] រកចម្លើយ $f$  មួយនៃ $(E)$  ដែលខ្សែកោង $(C)$ តាងអនុគមន៍កាត់តាមចំណុច $M(0,1)$ ហើយបន្ទាត់ប៉ះនឹង $(C)$
 ត្រង់ចំណុចនោះនោះស្របទៅនឹងបន្ទាត់ $(l):\;y=-2x$
	\end{enumerate}
\end{block}
\pause
	\begin{center}
	\Kml ដំណោះស្រាយ
	\end{center}
\end{frame}
\begin{frame}
	\begin{enumerate}
	\item[\Kh ក.] \Kh រកពហុធាដឺក្រេទីពីរ $P(x)$ 
	\end{enumerate}
\pause
	\begin{flalign*}\qquad\qquad
	\text{\Kh តាង}\;\;&P(x)=ax^2+bx+c\;\; \text{\Kh ជាចម្លើយពិសេសនៃសមីការ} (E)&\\
	\text{\Kh គេបាន}\;\;&P\;'(x)=2ax+b&\\
	&P\;''(x)=2a&
	\end{flalign*}
\pause
	\qquad\qquad\Kh ដោយ $P(x)$ ជាចម្លើយពិសេសនៃសមីការ $(E)$
\pause
	\begin{flalign*}\qquad\qquad
	\text{\Kh គេបាន}\;\; P\;''(x)+2P\;'(x)+P(x)&=x^2+2x-2&\\
	2ax+2\left(2ax+b\right)+ax^2+bx+c&=ax^2+bx+c&\\
	ax^2+(4a+b)x+(2a+2b+c)&=x^2+2x-2&
	\end{flalign*}
\pause
	\qquad\qquad\Kh គេបាន
	$\begin{cases}
	a=1\\
	4a+b=2\\
	2a+2b+c=-2
	\end{cases}$
\pause
	\Kh សមមូល
	$\begin{cases}
	a=1\\
	b=-2\\
	c=0
	\end{cases}$\\
\pause
	\qquad \Kh ដូចនេះ $P(x)=x^2-2x$ ជាចម្លើយនៃសមីការ $(E)$
\end{frame}
\begin{frame}
	\begin{enumerate}
	\item[\Kh ខ.] \Kh ដោះស្រាយសមីការ $(E\;'):\;y\;''+2y\;'+y=0$  
	\end{enumerate}
\pause
	\begin{flalign*}\qquad\qquad
	\text{\Kh គេបានសមីការសម្គាល់}\;\;m^2+2m+1&=0&\\
	(m+1)^2&=0&\\
	m&=-1\quad \text{\Kh ឫសឌុប}
	\end{flalign*}
\pause
	\begin{flalign*}\qquad\qquad
	\text{\Kh គេបានចម្លើយនៃសមីការ $(E\;')$ គឺ}\;\;y&=(Ax+B)e^{mx}&\\
	&=(Ax+B)e^{-x}&\\
	\text{\Kh ដូចនេះ ចម្លើយនៃសមីការ $(E\;')$ គឺ}\;\;&=(Ax+B)e^{-x}&\\
	&\quad\text{\Kh ដែល}\;\; A\;,\;B\in\mathbb R&
	\end{flalign*}
\end{frame}
\begin{frame}
	\Kh រួចទាញរកអនុគមន៍   $f$ ដែលជាចម្លើយទូទៅនៃសមីការ $(E)$
\pause
	\begin{flalign*}\qquad\qquad
	\text{\Kh ចម្លើយទូទៅនៃសមីការ $(E)$ គឺ}\;\;f&=y+P(x)&\\
	&=(Ax+B)e^{-x}+x^2-2x&\\
	\text{\Kh ដូចនេះ ចម្លើយទូទៅនៃសមីការ $(E)$ គឺ}\;\;f&=(Ax+B)e^{-x}+x^2-2x&\\ 		&\quad\text{\Kh ដែល}\;\; A\;,\;B\in\mathbb R&
	\end{flalign*}
	\begin{enumerate}
	\item[\Kh គ.] \Kh រកចម្លើយ $f$  មួយនៃ $(E)$\\[5pt]
\pause
	\Kh តាមបំរាប់ គេបាន
	$\begin{cases}
	f(0)=1\\[5pt]
	f\;'(0)=-2
	\end{cases}$
	\end{enumerate}
\end{frame}
\begin{frame}
	\begin{flalign*}\qquad\qquad
	\text{\Kh គេមាន}\quad  f(x)&=(Ax+B)e^{-x}+x^2-2x&\\
	f\;'(x)&=Ae^{-x}-(Ax+B)e^{-x}+2x-2&\\
	&=(A-Ax-B)e^{-x}+2x-2&
	\end{flalign*}
	\begin{center}
	 \Kh គេបាន 
	$\begin{cases}
	f(0)=B&\\[5pt]
	f\;'(0)=A-B-2
	\end{cases}$ \Kh តែ 
	$\begin{cases}
	f(0)=1\\[5pt]
	f\;'(0)=-2
	\end{cases}$\\[10pt]
	\Kh សមមូល
	$\begin{cases}
	B=1&\\[5pt]
	A-B-2=-2
	\end{cases}$ \Kh គេបាន
	$\begin{cases}
	B=1\\[5pt]
	A=1
	\end{cases}$\\[10pt]
\Kh ដូចនេះ ចម្លើយទូទៅនៃសមីការ $(E)$ គឺ $f=(x+1)e^{-x}+x^2-2x$
	\end{center}
\end{frame}
\begin{frame}
\frametitle{\Kml វិញ្ញាសាទី១  ត្រៀបប្រលងឆមាសទី ២}
\begin{block}{\centering \Kml លំហាត់ទី ៣}
\Kh នៅក្នុងប្រអប់មួយមានប៊ិចខៀវ $8$ និង ប៊ិចក្រហម $4$ ដែលដូចគ្នាទាំងទំហំ និងគំរូ។ គេដកប៊ិច $5$ ដើមព្រមគ្នាដោយចៃដន្យ
	\begin{enumerate}
	\item[\Kh ក.] រកប្រូបាបដែលគេបានប៊ិចទាំង $5$ ជាប៊ិចខៀវ
	\item[\Kh ខ.] រកប្រូបាបដែលគេចាប់បានប៊ិចក្រហម $3$
	\item[\Kh គ.] រកប្រូបាបដែលគេចាប់បានប៊ិចក្រហមមួយយ៉ាងតិច
	\item[\Kh ឃ.] រកប្រូបាបដែលគេចាប់បានប៊ិចក្រហម $2$ ឫ $3$
	\end{enumerate}
\end{block}
\pause
	\begin{center}
	\Kml ដំណោះស្រាយ
	\end{center}
\end{frame}
\begin{frame}
	\begin{enumerate}
	\item[\Kh ក.] \Kh រកប្រូបាបដែលគេបានប៊ិចទាំង $5$ ជាប៊ិចខៀវ\\
	\pause
	\Kh តាង $A$ ជាព្រឹត្តិការណ៍ដែលគេបានប៊ិចទំាង $5$ ជាប៊ិចខៀវ
	\pause
	\begin{flalign*}\qquad 
	\text{\Kh គេបាន}\quad n(A)&=C(8,5)=\frac{8!}{3!5!}=56&\\
	\text{\Kh ហើយ}\quad n(S)&=C(12,5)=\frac{12!}{5!7!}=792&\\
	\text{\Kh នោះ}\;\;P(A)&=\frac{n(A)}{n(S)}=\frac{56}{792}\\
	&=\frac{7}{99}
	\end{flalign*}
	\Kh ដូចនេះ ប្រូបាបដែលគេបានប៊ិចទាំង $5$ ជាប៊ិចខៀវគឺ $P(A)=\frac{7}{99}$
	\end{enumerate}
\end{frame}
\begin{frame}
	\begin{enumerate}
	\item[\Kh ខ.] \Kh រកប្រូបាបដែលគេចាប់បានប៊ិចក្រហម $3$\\
	\pause
	\Kh តាង $B$ ជាព្រឹត្តិការណ៍ដែលគេបានប៊ិចក្រហម $3$\\
	\pause
	\Kh មានន័យថា គេចាប់បានប៊ិចក្រហម $3$ និង ខៀវ $2$
	\pause
	\begin{flalign*}\qquad 
	\text{\Kh គេបាន}\quad n(B)&=C(4,3)\times C(8,2)=\frac{4!}{1!3!}\times \frac{8!}{6!2!}&\\
	n(B)&=112&\\
	\text{\Kh នោះ}\;\;P(B)&=\frac{n(B)}{n(S)}=\frac{112}{792}\\
	&=\frac{14}{99}
	\end{flalign*}
	\Kh ដូចនេះ ប្រូបាបដែលគេចាប់បានប៊ិចក្រហម $3$ គឺ $P(B)=\frac{14}{99}$
	\end{enumerate}
\end{frame}
\begin{frame}
	\begin{enumerate}
	\item[\Kh គ.] \Kh រកប្រូបាបដែលគេចាប់បានប៊ិចក្រហមមួយយ៉ាងតិច\\
	\pause
	\Kh តាង $C$ ជាព្រឹត្តិការណ៍ចាប់បានប៊ិចក្រហមមួយយ៉ាងតិច\\
	\pause
	ដោយព្រឹត្តិការណ៍ $C$ ជាព្រឹត្តិការណ៍ផ្ទុយពីព្រឹតិ្តការណ៍ $A$
	\pause
	\begin{flalign*}\qquad
	\text{\Kh គេបាន}\;\; P(A)+P(C)&=1&\\
	P(C)&=1-P(A)&\\
	&=1-\frac{7}{99}&\\
	&=\frac{92}{99}
	\end{flalign*}
	\end{enumerate}
	\Kh ដូចនេះ ប្រូបាបដែលគេចាប់បានប៊ិចក្រហមមួយយ៉ាងតិចគឺ$P(C)=\frac{92}{99}$
\end{frame}
\begin{frame}
	\begin{enumerate}
	\item[\Kh ឃ.] \Kh រកប្រូបាបដែលគេចាប់បានប៊ិចក្រហម $2$ ឫ $3$\\
	\pause
	\Kh តាង $D$ ជាព្រឹត្តិការណ៍ដែលគេចាប់បានប៊ិចក្រហម $2$ ឫ $3$\\
	\pause
	មានន័យថា គេចាប់បានប៊ិចក្រហម $2$ និង ខៀវ $3$ \\
	ឬ ក្រហម $3$ និង ខៀវ $2$
	\pause
	\begin{flalign*}\qquad 
	\text{\Kh គេបាន}\quad n(B)&=C(4,2)\times C(8,3)+C(4,3)\times C(8,2)&\\
	&=\frac{4!}{2!2!}\times \frac{8!}{5!3!}+\frac{4!}{1!3!}\times \frac{8!}{6!2!}=448&\\
	\text{\Kh នោះ}\;\;P(B)&=\frac{n(D)}{n(S)}=\frac{448}{792}\\
	&=\frac{56}{99}
	\end{flalign*}
	ដូចនេះ ប្រូបាបដែលគេចាប់បានប៊ិចក្រហម $2$ ឫ $3$គឺ $P(D)=\frac{56}{99}$
	\end{enumerate}
\end{frame}
\end{document}
