\documentclass[11pt, a5paper]{article}
\RequirePackage{geometry}
\geometry{
left=0.7cm,%
right=0.7cm,%
top=1.5cm,%
bottom=1.5cm}
\usepackage[no-math]{fontspec} % Package សម្រាប់សរសេរភាសាខ្មែរ
\setmainfont{Khmer OS Battambang} % Set roman font ជា font Khmer OS Battambang
\setsansfont{Khmer OS Bokor}
\setmonofont{Khmer OS System}
\newfontfamily{\kbk}{Khmer OS Bokor}
\newfontfamily{\kml}{Khmer OS Muol Light}
\XeTeXlinebreaklocale "kh"
\XeTeXlinebreakskip = 0pt plus 1pt minus 1pt% line break skip
\begin{document}
	\section*{\centering \LaTeX (\kml{ឡាតិច})}
	\quad\LaTeX~ជាភាសាកុំព្យួទ័រដែលអនុញ្ញាតឲ្យយើងប្រើប្រាស់ដើម្បីសរសេរឯកសារបច្ចេកទេស ដែលមានគុណភាពខ្ពស់ ។ \LaTeX~បង្កើតឡើងដើម្បីជួយសម្រួលដល់ការសរសេរអត្ថបទ ដែលមាននិម្មិតសញ្ញាគណិតវិទ្យា រូបមន្តគីមី ស្រៈព្យព្ជានៈ ជាពិសេសការរៀបចំឯកសារ ដើម្បីបោះពុម្ភផ្សាយផ្សេងៗ។ ម្យ៉ាងវិញទៀត បច្ចុប្បន្ននេះនៅតាមសាកលវិទ្យាល័យល្បីៗមួយចំនួនលើសាកលលោក គេតម្រូវឲ្យនិសិ្សតសរសេរនិក្ខេបបទ
\end{document}