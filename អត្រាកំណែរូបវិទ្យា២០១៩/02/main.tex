\documentclass{officialexam} 
\usepackage{circuitikz}
\everymath{\color{blue}}
\usepackage{graphicx}
\graphicspath{ {./images/} }
\begin{document}
	\maketitle\\
	\borderline{ប្រធាន}
	\begin{enumerate}[I]
		\item (៤ ពិន្ទុ) តើទ្រឹស្តីសុីនេទិចនៃឧស្ម័នសិក្សាអំពីអ្វី?
		\item (៦ ពិន្ទុ) គេផ្ទុកកុងដង់សាទ័រមួយដែលមានកាប៉ាសុីតេ $C=2.0\mu F$ ក្រោមតង់ស្យុង $V=5.0V$។\\ គណនាថាមពលអគ្គិសនីដែលផ្ទុកក្នុងកុងដង់សាទ័រ។
		\item (១០ ពិន្ទុ) គណនាមាឌឧស្ម័នអុកសុីសែន $6.4g$ ដែលផ្ទុកក្នុងធុងក្រោមសម្ពាធ $1.0\times10^{5}Pa$ និងសីតុណ្ហភាព $27^\circ C$ ។ \\គេឱ្យ $R=8.31J/mol\cdot K$ និងម៉ាសម៉ូលេគុលឧស្ម័នអុកសុីសែន $M(O_2)=32g/mol$
		\item (១០ ពិន្ទុ) គេធ្វើកម្មន្ត $20kJ$ លើប្រព័ន្ធឧស្ម័នបិទជិតមួយ។ ក្រោយមកកម្តៅ $4190J$ បានភាយចេញពីប្រព័ន្ធ។ \\គណនាបម្រែបម្រួលថាមពលក្នុងនៃប្រព័ន្ធ។
		\item (១០ ពិន្ទុ) ម៉ាសុីនមួយមានទិន្នផលកម្តៅ $0.40$ គណនា៖
		\begin{enumerate}[k]
			\item កម្មន្តដែលបានធ្វើ ប្រសិនបើវាស្រូបកម្ដៅ $4000J$ ពីធុងក្តៅ។
			\item កម្តៅភាយចេញពីធុងត្រជាក់។
		\end{enumerate} 
		\item (១០ ពិន្ទុ) សូលេណូអុីតមួយមានប្រវែង $l=40.0cm$ និងមានកាំ $R=2.0cm$ ត្រូវបានរុំជា​​ស្ពៀ​​​​ជាប់ៗ​​គ្នាចំនួន $2000$ ស្ពៀ។
		\begin{enumerate}[k]
			\item គណនាអាំងឌុចតង់នៃសូលេណូអុីតនេះ។
			\item គណនាថាមពលម៉ាញេទិច បើមានចរន្តប្រែប្រួល $i$ ឆ្លងកាត់បូប៊ីនមានតម្លៃ $i=2.0A$ ។\\ គេឱ្យ $~\pi^2=10$ និងជំរាបដែនម៉ាញេទិចក្នុងសុញ្ញាកាស $\mu_0=4\pi\times10^{-7}T\cdot m/A$
		\end{enumerate} 
	\end{enumerate}
{\color{white}.}\dotfill\\
{\color{white}.}\dotfill\\
{\color{white}.}\dotfill\\
{\color{white}.}\dotfill\\
{\color{white}.}\dotfill\\
{\color{white}.}\dotfill\\
{\color{white}.}\dotfill\\
{\color{white}.}\dotfill\\
{\color{white}.}\dotfill\\
{\color{white}.}\dotfill\\
{\color{white}.}\dotfill\\
{\color{white}.}\dotfill\\
{\color{white}.}\dotfill\\
{\color{white}.}\dotfill\\
{\color{white}.}\dotfill\\
\newpage
\borderline{\bigg[អត្រាកំណែវិញ្ញាសា រូបវិទ្យា\bigg]}\\
\begin{enumerate}[I]
	\item ទ្រឹស្តីសុីនេទិចនៃឧស្ម័នសិក្សាអំពីចលនារបស់ម៉ូលេគុលឧស្ម័នស្ថិតនៅក្នុងធុងដែលផ្ទុកវា។
	\item គណនាថាមពលអគ្គិសនីដែលផ្ទុកក្នុងកុងដង់សាទ័រ
	\begin{align*}
		\text{តាមរូបមន្ត}\ & E_C=\frac{1}{2}CV^2\\
		\text{ដោយ}\ & C=2.0\mu F = 2.0\times10^{-6}F\\
					\ & V= 5.0V\\
		\text{គេបាន}\ & E_C=\frac{1}{2}\left(2.0\times10^{-6}\right)\left(5.0\right)^2=25.0\times10^{-6}J\\
		\text{ដូចនេះ}​​\quad & \fbox{$E_C=25.0\times10^{-6}J$}
	\end{align*}
	\item គណនាមាឌឧស្ម័នអុកសុីសែន
	\begin{flalign*}
	\text{តាមរូបមន្ត}\quad & PV=nRT\\
	\Rightarrow			  \quad & V= \frac{nRT}{P}\\
	\text{ដោយ}\quad & R=8.31J/mol\cdot K\\
				\quad & T=27+273K=300K\\
				\quad & P=1.0\times10^{5}Pa\\
				\quad & m=6.4g\\
				\quad & M\left(O_2\right)=32g/mol\\
	\text{តែ}\quad & n=\frac{m}{M}=\frac{6.4}{32}=0.2mol\\
	\Rightarrow \quad & V=\frac{0.2\times8.31\times300}{1.0\times10^{5}}=4986\times10^{-6}m^{3}\simeq50\times10^{-6}m^{3}\\
	\text{ដូចនេះ}​​\quad & \fbox{$V=4986\times10^{-6}m^{3}\simeq50\times10^{-6}m^{3}$}
	\end{flalign*}
	\item គណនាបម្រែបម្រួលថាមពលក្នុងនៃប្រព័ន្ធ
	\begin{flalign*}
		\text{តាមរូបមន្ត}\quad & Q=\Delta U + W\\
		\Rightarrow\quad & \Delta U = Q-W\\
		\text{ដោយ}\quad & W=-20kJ=-20\times10^{3}J=-20000J\quad (\text{ប្រព័ន្ធ​ឧស្ម័នរងកម្មន្ត})\\
		\quad & Q=-4190J\quad (\text{ប្រព័ន្ធបំភាយកម្តៅ})\\
		\Rightarrow\quad & \Delta U = -4190-\left(-20000\right)=-4190+20000=15810J\\
		\text{ដូចនេះ}\quad & \fbox{$\Delta U = 15810J$}
	\end{flalign*}
	\item \begin{enumerate}[k]
		\item គណនាកម្មន្តដែលបានធ្វើ ប្រសិនបើវាស្រូបកម្ដៅ $4000J$ ពីធុងក្តៅ​
		\begin{flalign*}
		\text{តាមរូបមន្ត}\quad & e=\frac{W}{Q_h}\\
		\Rightarrow\quad & W=e\times Q_h\\
		\text{ដោយ}\quad & e=0.40\\
					\quad & Q_h=4000J\\
		\Rightarrow\quad & W=0.40\times4000=1600J\\
		\text{ដូចនេះ}\quad & \fbox{$W=1600J$}
		\end{flalign*}
		\item គណនាកម្តៅភាយចេញពីធុងត្រជាក់
		\begin{flalign*}
			\text{តាមរូបមន្ត}\quad & W=Q_h-Q_c\\	
			\Rightarrow\quad & Q_c=Q_h-W\\
			\text{ដោយ}\quad & W=1600J\\
			\quad & Q_h=4000J\\
			\Rightarrow\quad & Q_c=4000-1600=2400J\\
			\text{ដូចនេះ}\quad & \fbox{$Q_c=2400J$}
		\end{flalign*}
	\end{enumerate}
	\item\begin{enumerate}[k]
		\item គណនាអាំងឌុចតង់នៃសូលេណូអុីតនេះ
		\begin{flalign*}
			\text{តាមរូបមន្ត}\quad & L=\mu_0\frac{N^2A}{l}\\	
			\text{ដោយ}\quad & \mu_0=4\pi\times10^{-7}T\cdot m/A\\
					   \quad & N=2000=2\times10^{3}\text{ស្ពៀ}\\
					   \quad & l=40.0cm=40.0\times10^{-2}m=4.0\times10^{-1}m\\
					   \quad & R=2.0cm=2.0\times10^{-2}m\\
			\text{តែ}\quad & A=\pi R^{2}=\pi\left(2.0\times10^{-2}\right)^{2}=4\pi\times10^{-4}m^{2}\\
			\Rightarrow\quad & L=4\pi\times10^{-7}\frac{\left(2\times10^{3}\right)^2\times4\pi\times10^{-4}m^{2}}{4.0\times10^{-1}}=16\pi^{2}\times10^{-4}\\
			\quad & L =16\times10\times10^{-4}=16.0\times10^{-3}=16mH\\
			\text{ដូចនេះ}\quad & \fbox{$L=16.0\times10^{-3}H \quad\text{ឬ}\quad L=16mH$}
		\end{flalign*}
		\item គណនាថាមពលម៉ាញេទិច បើមានចរន្តប្រែប្រួល $i$ ឆ្លងកាត់បូប៊ីនមានតម្លៃ $i=2.0A$
		\begin{flalign*}
			\text{តាមរូបមន្ត}\quad & E_L=\frac{1}{2}Li^{2}\\
			\text{ដោយ}\quad & L=16\times10^{-3}H\\
						\quad & i=2.0A\\
			\Rightarrow \quad & E_L=\frac{1}{2}\left(16.0\times10^{-3}\right)\left(2.0\right)^{2}=32.0\times10^{-3}J\\
			\text{ដូចនេះ}\quad & \fbox{$E_L=32\times10^{-3}J\quad\text{ឬ}\quad E_L=32mJ$}
		\end{flalign*}
	\end{enumerate} 
\end{enumerate}
\end{document}