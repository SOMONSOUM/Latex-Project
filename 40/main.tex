\documentclass[12pt, a4paper]{article}
%%import package named hightest
\usepackage{hightest}
\usepackage{tikz}
%\usepackage{mathpazo}% change math font
%\usepackage[no-math]{fontspec}% font specfication
\header{រៀនគណិតវិទ្យាទាំងអស់គ្នា}{គណិតវិទ្យា}{\khmerdate}
\footer{រៀបរៀង និងបង្រៀនដោយ ស៊ុំ សំអុន}{ទំព័រ \thepage}{០៩៦ ៩៤០ ៥៨៤០}
\everymath{\protect\displaystyle\protect\color{black}}
\begin{document}
\maketitle
\begin{enumerate}[m]
	\item ចូរសិក្សាអថេរភាព និងសង់ក្រាបនៃអនុគមន៍ខាងក្រោម៖
	\begin{enumerate}[k,3]
		\item $y=f(x)=2-\frac{\ln x}{x}$
		\item $y=f(x)=\frac{\ln x}{x^2}$
		\item $f(x)=1-x\ln x$
		\item $f(x)=x-1+\frac{\ln x}{x}$
		\item $f(x)=x^2\ln x-1$
		\item $f(x)=x-\frac{\ln x}{x}$
	\end{enumerate}
	\item 
\end{enumerate}
	\begin{center}
		\sffamily\color{black}
		សូមសំណាងល្អ!
	\end{center}\newpage
\end{document}