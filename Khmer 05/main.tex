\documentclass[12pt,a4paper]{article}
\usepackage{geometry}
\geometry{
	left=0.7cm,%
	right=0.7cm,%
	top=1.5cm,%
	bottom=1.5cm
}
\usepackage{graphicx}
\usepackage{tikz}
\usepackage{wasysym}
\usepackage{mathpazo}% change math font
\usepackage{enumitem}% change list environment like enumerate, itemize and description
\usepackage{multicol}% multi columns
\usepackage{tikz}% graphic drawing
\usepackage[no-math]{fontspec}% font specfication
\RequirePackage{unicode-math}
%\setmathfont{Latin Modern Math}
\setmathfont{TeX Gyre Pagella Math}
\setmainfont{Khmer OS Battambang}% set default font to Khmer OS
\setsansfont[Ligatures=TeX,AutoFakeBold=0,AutoFakeSlant=0.25]{Khmer OS Bokor}% sans serif font
\setmonofont[AutoFakeBold=0.25,AutoFakeSlant=0.25]{Khmer OS Muol Light}% sans serif font
\XeTeXlinebreaklocale "kh"
\XeTeXlinebreakskip = 0pt plus 1pt minus 1pt% line break skip

\setmathrm{Times New Roman}
\newcommand{\kos}{\fontspec[Scale = 0.875, Script=Khmer]{Khmer OS System}\selectfont}
\newcommand{\en}{\fontspec{Times New Roman}\selectfont}
\newcommand{\km}{\fontspec[Scale = 0.875, Script=Khmer]{Khmer OS Muol}\selectfont}
\newcommand{\kml}{\fontspec[Scale = 0.875, Script=Khmer]{Khmer OS Muol Light}\selectfont}

\newcommand{\heart}{\ensuremath\heartsuit}
\newcommand{\butt}{\rotatebox[origin=c]{180}{\heart}}
\newcommand*\circled[1]{\tikz[baseline=(char.base)]{
\node[shape=circle,draw,inner sep=2pt] (char) {#1};}}
%
\SetEnumitemKey{I}{%
leftmargin=*,
label={\protect\tikz[baseline=-0.9ex]\protect\node[draw=gray,thick,circle,minimum height=.65cm,inner sep=1pt,text=black,fill=white]{\Roman*};},%
font=\small\sffamily\bfseries,%
labelsep=1ex,%
topsep=0pt}
%
\SetEnumitemKey{a}{%
leftmargin=*,%
label={\protect\tikz[baseline=-0.9ex]\protect\node[draw=gray,thick,circle,minimum height=.5cm,inner sep=1pt,text=blue,fill=magenta!5!white]{\alph*};},%
font=\small\sffamily\bfseries,%
labelsep=1ex,%
topsep=0pt}
%
\SetEnumitemKey{1}{leftmargin=*,%
label={\protect\tikz[baseline=-0.9ex]\protect\node[draw=gray,thick,circle,minimum height=.5cm,inner sep=1pt,text=black,fill=cyan!20!white]{\arabic*};},%
font=\small\sffamily\bfseries,%
labelsep=1ex,%
topsep=0pt}
%
\def\hard{\leavevmode\makebox[0pt][r]{\large\ensuremath{\star}\hspace{2em}}}
%
\def\hhard{\leavevmode\makebox[0pt][r]{\large\ensuremath{\star\star}\hspace{2em}}}
%
\everymath{\protect\displaystyle\protect\color{blue}}
\makeatletter 
% khmer number
\def\khmer#1{\expandafter\@khmer\csname c@#1\endcsname}
\def\@khmer#1{\expandafter\@@khmer\number#1\@nil}
\def\@@khmer#1{%
	\ifx#1\@nil% terminate when encounter @nil
	\else%
	\ifcase#1 ០\or ១\or ២\or ៣\or ៤\or ៥\or ៦\or ៧\or ៨\or ៩\fi%
	\expandafter\@@khmer% recursively map the following characters
	\fi}
\def\khmernumeral#1{\@@khmer#1\@nil}
\def\alpkh#1{\expandafter\@alpkh\csname c@#1\endcsname}
\def\@alpkh#1{%
	\ifcase#1% zero ->none
	\or ក\or ខ\or គ \or ឃ\or ង%
	\or ច\or ឆ\or ជ\or ឈ\or ញ%
	\or ដ\or ឋ\or ឌ\or ឍ\or ណ%
	\or ត\or ថ\or ទ\or ធ\or ន%
	\or ប\or ផ\or ព\or ភ\or ម%
	\or យ\or រ\or ល\ro វ\or ស%
	\or ហ\or ឡ\or អ%
	\else%
	\@ctrerr % Otherwise counter error
	\fi
}
\def\khmershortdate{%
	\ifnum\day<10\@khmer{0\day}\else\@khmer\day\fi%
	/\@khmer\month%
	/\@khmer\year}
\def\khmerdate{%
	\ifnum\day<10\@khmer{0\day}\else\@khmer\day\fi~%
	\ifcase\month\or%
	មករា\or កុម្ភៈ\or មិនា\or មេសា\or%
	ឧសភា\or មិថុនា\or កក្កដា\or សីហា\or%
	កញ្ញា\or តុលា\or វិច្ឆិកា\or ធ្នូ\fi~%
	\@khmer\year}
\makeatother
\AddEnumerateCounter{\alpkh}{\@alpkh}{ឈ}
\AddEnumerateCounter{\khmer}{\@khmer}{៣}
% enumerate keys
%\SetEnumitemKey{1}{leftmargin=*,labelsep=1ex,itemsep=1ex,label=\arabic*.}
%\SetEnumitemKey{a}{leftmargin=*,labelsep=1ex,itemsep=1ex,label=\alph*.}
\SetEnumitemKey{A}{leftmargin=*,labelsep=1ex,itemsep=1ex,label=\Alph*.}
\SetEnumitemKey{i}{leftmargin=*,labelsep=1ex,itemsep=1ex,label=\roman*.}
%\SetEnumitemKey{I}{leftmargin=*,labelsep=1ex,itemsep=1ex,label=\Roman*.}
\SetEnumitemKey{k}{leftmargin=*,%
	label={\protect\tikz[baseline=-0.9ex]\protect\node[draw=gray,thick,circle,minimum height=.5cm,inner sep=1pt,text=blue,fill=magenta!5!white]{\alpkh*};},%
	font=\small\sffamily\bfseries,%
	labelsep=1ex,%
	topsep=0pt}
%
\SetEnumitemKey{m}{leftmargin=*,%
	label={\protect\tikz[baseline=-0.9ex]\protect\node[draw=gray,thick,circle,minimum height=.5cm,inner sep=1pt,text=magenta,fill=cyan!20!white]{\khmer*};},%
	font=\small\sffamily\bfseries,%
	labelsep=1ex,%
	topsep=0pt}
%

\DeclareMathSizes{12pt}{13pt}{11pt}{10pt}


\pagenumbering{khmer}
%
\pagecolor{cyan!1!white}
%
\begin{document}
\begin{center}
\kml\color{blue}
\circled{១}\\
\heart គន្លឹះសិក្សាអនុគមន៍ និង ខ្សែកោង\heart \\
រៀបរៀងដោយ៖ ស៊ុំ សំអុន\\
\phone ទូរស័ព្ទលេខៈ ០៩៦ ៩៤០ ៥៨៤០\phone
\end{center}
\begin{enumerate}[m]
\item ចូររកដែនកំណត់នៃអនុគមន៍ខាងក្រោម៖
	\begin{multicols}{3}
		\begin{enumerate}[k]
		\item $f(x)=\frac{x+1}{x-1}$
		\item $f(x)=\frac{2-3x}{x^2-3x+2}$
		\item $f(x)=\frac{x^2+x+1}{x^2-x+1}$
		\item $f(x)=\ln\left( x+1\right)+e^{2x}$
		\item $f(x)=x+1+\ln\left(\frac{3+x}{3-x}\right)$
		\item $f(x)=x+1+\ln\left(\frac{x+2}{x-2}\right)$
		\end{enumerate}
	\end{multicols}
\item រកសមីការបន្ទាត់ប៉ះ $T$ ដែលប៉ះនឹងខ្សែកោង៖
	\begin{enumerate}[k]
		\item $C: f(x)=x^2+1$ ត្រង់ចំណុចដែលមានអាប់ស៊ីស $x_0 =1$
		\item $C: f(x)=1-x\ln x$ ត្រង់ចំណុចដែលមានអាប់ស៊ីស $x_0 =1$
		\item $C: f(x)=\frac{e^x}{1-\sin x}$ ត្រង់ចំណុចដែលមានអាប់ស៊ីស $x_0 =0$
		\item $C: f(x)=e^x+\frac{e^x+1}{e^x-1}$ ត្រង់ចំណុចដែលមានអាប់ស៊ីស $x_0 =\ln2$
	\end{enumerate}
\item គេមានអនុគមន៍ $f$ កំណត់ដោយ $y=f(x)=1+\frac{\ln x}{x}$ និងមានខ្សែកោង $H$ ។
	\begin{enumerate}[k]
	\item សរសេរសមីការបន្ទាត់ $d$ ដែលប៉ះខ្សែកោង $H$ ត្រង់ចំណុច $A(1, 1)$ ។
	\item គេឲ្យខ្សែកោង $K$ តាងអនុគមន៍ $y=g(x)=e^{3x}+x-e^6$ ។ \\
	ចូរកំណត់កូអរដោនេនៃចំណុចប្រសព្វ $B$ រវាងបន្ទាត់ $d$ និងខ្សែកោង $K$ តាង $g$ ។
	\end{enumerate}
\item រកតម្លៃបរមានៃអនុគមន៍ខាងក្រោម៖
	\begin{multicols}{3}
		\begin{enumerate}[k]
		\item $y=\frac{x^2-x-2}{x+2}$
		\item $y=\frac{x^2-3x+6}{x-2}$
		\item $y=\frac{x^2+x+2}{x-1}$
		\end{enumerate}
	\end{multicols}
\item គេឲ្យអនុគមន៍ $f(x)=\frac{ax^2+bx+c}{x-2}$ ។ រកតម្លៃមេគុណ $a, b$ និង$c$ ដើម្បីឲ្យអនុគមន៍ $f$ មានតម្លៃស្មើ $-1$ ចំពោះ $x=1$ ហើយមានតម្លៃបរមាស្មើ $8$ ត្រង់$x=4$ ។
\item គេឲ្យអនុគមន៍ $f(x)=\frac{ax^2+bx+c}{x}$ ។ រកតម្លៃមេគុណ $a, b$ និង$c$ ដើម្បីឲ្យអនុគមន៍ $f$ មានតម្លៃស្មើ $8$ ចំពោះ $x=1$ ហើយមានតម្លៃអតិបរមាស្មើ $-1$ ត្រង់$x=-2$ ។
\item គេឲ្យអនុគមន៍ $g(x)=ax+a+\frac{b}{x+2}$ ចំពោះ $x\neq -2$ ។ រកតម្លៃមេគុណ $a$ និង $b$ ដើម្បីឲ្យអនុគមន៍ $g$ មានតម្លៃអប្បរមាស្មើ $2$ ចំពោះ $x=1$ ហើយមានតម្លៃអតិបរមាស្មើ $-1$ ត្រង់$x=0$ 
\item រកសមីការអាស៊ីមតូតនៃក្រាបតាងអនុគមន៍នីមួយៗដូចខាងក្រោម៖
	\begin{multicols}{3}
		\begin{enumerate}[k]
		\item $y=f(x)=\frac{x^2+x+1}{x-1}$
		\item $y=f(x)=\frac{x^2+2x-3}{x+2}$
		\item $y=f(x)=\frac{3x^2+6x+3}{x^2+2}$
		\end{enumerate}
	\end{multicols}
\item គេឲ្យអនុគមន៍ $f$ កំណត់ដោយ $y=f(x)=\frac{x^2-x+1}{x-1}$ និងមានក្រាប $C$។
\begin{enumerate}[k]
\item រកសមីការអាស៊ីមតូតឈរ និងអាស៊ីមតូតទ្រេតរបស់ក្រាប $C$ ។
\item បង្ហាញថាចំណុច $I(1, 1)$ ជាផ្ចិតឆ្លុះរបស់ក្រាប $C$ ។
\end{enumerate}
\item គេមានអនុគមន៍ $f$ កំណត់ដោយ $y=f(x)=\frac{x}{x^2+1}$ និងមានក្រាប $C$ ។
	\begin{enumerate}[k]
		\item រកសមីការអាស៊ីមតូតរបស់ក្រាប $C$ ។
		\item សិក្សាភាពគូរ-សេស រួចទាញថា គល់ $O$ នៃតម្រុយជាផ្ចិតឆ្លុះនៃក្រាប $C$ ។
	\end{enumerate}
\item សិក្សាអថេរភាព និងសង់ក្រាបនៃអនុគមន៍ខាងក្រោម៖ 
	\begin{multicols}{3}
		\begin{enumerate}[k]
		\item $f(x)=\frac{x^2+x+1}{x+1}$
		\item $f(x)=\frac{x^2-2x-3}{x-1}$
		\item $f(x)=\frac{x^2-3x+2}{x+2}$
		\end{enumerate}
	\end{multicols}
\item សិក្សាអថេរភាព និងសង់ក្រាបនៃអនុគមន៍ខាងក្រោម៖ 
	\begin{multicols}{3}
		\begin{enumerate}[k]
		\item $f(x)=\frac{x^2+x+1}{x^2+1}$
		\item $f(x)=\frac{x^2-2x+1}{x^2-2x}$
		\item $f(x)=\frac{3x^2+6x+3}{x^2+2}$
		\end{enumerate}
	\end{multicols}

\item អនុវត្តន៍ $ f $ កំណត់ដោយ $f(x)=x+2-\frac{4}{x-1}$ និងមានខ្សែកោង $C$ ។
	\begin{enumerate}[k]
		\item រកដែនកំណត់នៃអនុគមន៍ $f$ ។ គណនា និងសិក្សាសញ្ញាដេរីវេ $f'(x)$ ។
		\item រកតម្លៃអតិបរមា និងអប្យបរមានៃ $f$ ។
		\item កំណត់សមីការនៃអាស៊ីមតូតឈរ និងទ្រេតនៃខ្សែកោង $C$ ។
		\item សិក្សាទីតាំងធៀបរវាងអាស៊ីមតូតទ្រេត និងខ្សែកោង $C$ ។
		\item សង់តារាងអថេរភាពនៃអនុគមន៍ $f$ និងសង់ខ្សែកោង $C$ ។
	\end{enumerate}
\item អនុគមន៍ $f$ កំណត់ចំពោះគ្រប់ $x\neq 1$ ដោយ $f(x)=\frac{x^2-3x+6}{x-1}$ និងមានក្រាប $C$ ។
	\begin{enumerate}[k]
		\item រកចំនួនពិត $a, b$ និង $c$ ដើម្បីឲ្យ $f(x)=ax+b+\frac{c}{x-1}$ ចំពោះគ្រប់ $x\neq1$ ។
		\item រកតម្លៃអតិបរមា និងអប្បបរមានៃ $f$ ។
		\item រកសមីការនៃអាស៊ីមតូតឈរ និងទ្រេតនៃខ្សែកោង $C$ ។
		\item សិក្សាទីតាំងធៀបរវាងអាស៊ីមតូតទ្រេត និងខ្សែកោង $C$ ។
		\item សង់តារាងអថេរភាពនៃអនុគមន៍ $f$ និងសង់ខ្សែកោង $C$ ។
	\end{enumerate}
\item  គេមានអនុគមន៍ $f$ កំណត់ដោយ $f(x)=\frac{x^2+5x+15}{x+3}$ មានខ្សែកោង $C$ ។
	\begin{enumerate}[k]
		\item សិក្សាអថេរភាព និងគូសខ្សែកោងនៃអនុគមន៍ $f$ ។
		\item រកគ្រប់ចំណុចនៅលើខ្សែកោងនៃ $f$ ដែលមានកូអរដោនេជាចំនួនគត់រ៉ឺឡាទីប ។
	\end{enumerate}
\end{enumerate}
\begin{center}
\sffamily\color{blue}
សូមសំណាងល្អ!
\end{center}
\end{document}