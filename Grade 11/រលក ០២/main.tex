\documentclass[12pt, a4paper]{article}
%%import package named techno
\usepackage{techno}
\usepackage[export]{adjustbox}
\usepackage{wrapfig}
\usepackage{tkz-tab}
%សរសេរគីមីវិទ្យា
\usepackage[version=3]{mhchem}
%\usepackage{mathpazo}% change math font
%\usepackage[no-math]{fontspec}% font specfication
\everymath{\protect\displaystyle\protect\color{blue}}
\begin{document}
\maketitle\koc
	\begin{enumerate}[m]
		\item ចល័តមួយចងភ្ជាប់ទៅនឹងរ៉ឺសរបញ្ឃរសន្ធឹងមួយប្រវែង $10cm$ ត្រូវបានទាញចុះក្រោមប្រវែង $A=5.0cm$ រួចលែងដោយគ្មានល្បឿនដើម។ គេឲ្យ $k=29.4N/m$ និងពុលសាស្យុង $\omega = 9.90rad/s$ ។ រកប្រកង់ និងខួបនៃចលនា។
		\item លំយោលសុីនុយសូអុីតមួយមានអំព្លីទុត $5cm$ និងខួប $2s$។ នៅខណៈដើមពេលភាគល្អិតស្ថិតត្រង់ទីតាំង $25cm$ ។ ចូរកំណត់សមីការនៃបម្លាស់ទីភាគល្អិតជាអនុគមន៍នៃពេល។
		\item អង្គធាតុមួយត្រូវបានគេចងព្យួរទៅនឹងរ៉ឺសរមួយ។ គេទាញវាចុះក្រោមរួចប្រលែង នៅខណៈនោះវាផ្លាស់ទីបានអំព្លីទុត $A=50cm$ ។ គេឲ្យពុលសាស្យុង $\omega=10rad/s$ ។
		\begin{enumerate}[k]
			\item គណនាប្រេកង់នៃចលនា។
			\item គណនាខួបនៃចលនា។
			\item សរសេរសមីការនៃលំយោល។
		\end{enumerate}
		\item ប៉ោលរ៉ឺសរដងដេកមួយធ្វើឡើងពីរ៉ឺសរមានថេរកម្រាញ $k=29.4N/m$ និងភ្ជាប់ដោយម៉ាសមួយ $m=0.30kg$ ។ ចូររកខួប និងប្រកង់នៃលំយោល។
		\item ឃ្លីមួយត្រូវបានចងភ្ជាប់ជាមួយនឹងខ្សែ $l=1.6m$ ព្យួរទៅនឹងបង្គោលឈរដោយដែកគោលរួចហើយធ្វើឃ្លីឲ្យវិលជាចលនាវង់ស្មើរាល់មួយវិនាទី $24$ ជុំ។ គណនាសំទុះរបស់គ្រាប់ឃ្លី ដោយគម្លាតមុំ របស់ខ្សែ $\alpha = 30^\circ$ ។
		\item អង្គធាតុមួយធ្វើចលនាអាកម៉ូនិចលើគន្លងត្រង់មួយជុំវិញទីតាំងលំនឹង $O$ ជាមួយនឹងខួប $T=0.3s$ ដោយដឹងថា $t=0$ អង្គធាតុមានអេឡុងកាស្យុង $x=-9cm$ ជាមួយនឹងល្បឿនដើមស្មើសូន្យ។
		\begin{enumerate}[k]
			\item សរសេរសមីការលំយោល។
			\item គណនាល្បឿនអតិបរមា។
		\end{enumerate}
		\item ប៉ោលរ៉ឺសរមួយយោលដោយអំព្លីទុត $4cm$ និងខួប $T=0.1s$។ សរសេរសមីការលំយោលរបស់ប៉ោលនោះ បើនៅខណៈពេល $t=0s$ ប៉ោលរ៉ឺសរនោះមានអេឡុង​​កា​​ស្យុង $x=2cm$។ គណនារយៈពេលខ្លីបំផុតដើម្បីឲ្យប៉ោលយោលពី $x_1=2cm$ ទៅ $x_2=4cm$ ។
		\item សមីការរបស់រួបធាតុមួធ្វើលំយោលអាកម៉ូនិចមានទម្រង់ $x=10\sin\left(5\pi + \frac{\pi}{6}\right)$ ។
		\begin{enumerate}[k]
			\item កំណត់ខួប ប្រេកង់មុំ អំព្លីទុត និងផាសដើមរបស់លំយោល។
			\item កំណត់អេឡុងកា​​​ស្យុង $x$ នៅពេលខណៈ $t=0.4s$។
			\item គណនាអេឡុងកា​​​ស្យុងពេលដែលផាសយោលបាន $-\frac{\pi}{4}$ ។
		\end{enumerate}
		\item គេចងព្យួរប៉ោលទី១ មានប្រវែង $l_1$ និងខួប $T_1=0.3s$ ហើយប៉ោលទី២ មានប្រវែង $l_2$ និងខួប $T_2=0.4s$។ ចូរគណនាខួបនៃប៉ោលទោលដែលមានប្រវែង $\left(l_1+l_2\right)$ នៅត្រង់កន្លែងនោះ។
		\item សរសេរសមីការផ្គួបនៃចលនាលំយោលអាកម៉ូនិចពីរដែលមានសមីការ $x_1=10\sin\left(\omega t -\frac{\pi}{6}\right)$ និង $x_2=10\sin\left(\omega t+ \frac{\pi}{3}\right)$ ដែល $x$ គិតជា $cm$ និង $t$ គិតជា $s$ ។ គេឲ្យ៖ ពុល​​សា​​ស្យុង$\omega = 50 rd/s$
	\end{enumerate}
	\borderline{ចម្លើយ}\\
	{\color{white}.}\dotfill\\
	{\color{white}.}\dotfill\\
	{\color{white}.}\dotfill\\
	{\color{white}.}\dotfill\\
	{\color{white}.}\dotfill\\
	{\color{white}.}\dotfill\\
	{\color{white}.}\dotfill\\
	{\color{white}.}\dotfill\\
	{\color{white}.}\dotfill\\
	{\color{white}.}\dotfill\\
	{\color{white}.}\dotfill\\
	{\color{white}.}\dotfill\\
	{\color{white}.}\dotfill\\
	{\color{white}.}\dotfill\\
	{\color{white}.}\dotfill\\
	{\color{white}.}\dotfill\\
	{\color{white}.}\dotfill\\
	{\color{white}.}\dotfill\\
	{\color{white}.}\dotfill\\
	{\color{white}.}\dotfill\\
	{\color{white}.}\dotfill\\
	{\color{white}.}\dotfill\\
	\begin{center}
		\sffamily\color{blue}
		សូមសំណាងល្អ!
	\end{center}\newpage
\end{document}