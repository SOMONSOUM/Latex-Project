\documentclass[a4paper,12pt]{book}
\usepackage{polyglossia}
\newfontfamily\khmerfont[%
Script=Khmer,
BoldFont={Khmer M1},
ItalicFont=*,
ItalicFeatures={FakeSlant=.2},
Scale=0.95]{Khmer OS}
\setmainlanguage{khmer}
%%
\usepackage{titlecolor}
%%
\newtheorem{definition}{និយមន័យ}[chapter]
%%
\begin{document}
	\frontmatter
	\pagenumbering{alph}
	\tableofcontents
	\mainmatter
	\chapter{ត្រីកោណមាត្រ}
	ផលធៀបត្រីកោណមាត្រ
	\section{ស៊ីនុស}
	\begin{definition}[ស៊ីនុស]
		\upshape
		គេឲ្យត្រីកោណ $ \triangle ABC $ មួយកែងត្រង់ $ C $។ \emph{ស៊ីនុស}នៃមុំ $ A $ ជាផលធៀបរង្វាស់ជ្រុងឈមនឹងអ៊ីប៉ូតេនុស។
		\[ \sin A=\frac{BC}{AB}=\frac{\textnormal{ប្រវែងជ្រុងឈម}}{\textnormal{ប្រវែងអ៊ីប៉ូតេនុស}} \]
	\end{definition}
	\section{កូស៊ីនុស}
	\begin{definition}[កូស៊ីនុស]
		\upshape
		គេឲ្យត្រីកោណ $ \triangle ABC $ មួយកែងត្រង់ $ C $។ \emph{កូស៊ីនុស}នៃមុំ $ A $ ជាផលធៀបរង្វាស់ជ្រុងជាប់នឹងអ៊ីប៉ូតេនុស។
		\[ \cos A=\frac{AC}{AB}=\frac{\textnormal{ប្រវែងជ្រុងជាប់}}{\textnormal{ប្រវែងអ៊ីប៉ូតេនុស}} \]
	\end{definition}
	\section{តង់សង់}
	\begin{definition}[តង់សង់]
		\upshape
		គេឲ្យត្រីកោណ $ \triangle ABC $ មួយកែងត្រង់ $ C $។ \emph{តង់សង់}នៃមុំ $ A $ ជាផលធៀបរង្វាស់ជ្រុងឈមនឹងជ្រុងជាប់។
		\[ \tan A=\frac{BC}{AC}=\frac{\textnormal{ប្រវែងជ្រុងឈម}}{\textnormal{ប្រវែងជ្រុងជាប់}} \]
	\end{definition}
	\section{កូតង់សង់}
	\begin{definition}[កូតង់សង់]
		\upshape
		គេឲ្យត្រីកោណ $ \triangle ABC $ មួយកែងត្រង់ $ C $។ \emph{កូតង់សង់}នៃមុំ $ A $ ជាផលធៀបរង្វាស់ជ្រុងជាប់នឹងជ្រុងឈម។
		\[ \cot A=\frac{BC}{AC}=\frac{\textnormal{ប្រវែងជ្រុងជាប់}}{\textnormal{ប្រវែងជ្រុងឈម}} \]
	\end{definition}
	\appendix
	\chapter{ចំណងជើង}
	តើមានអ្វីប្លែក
	\section{ចំណងជើង}
	តើមានអ្វីប្លែក
	\backmatter
	\begin{thebibliography}{2}
		\bibitem{key} author name,
		\newblock\emph{book title},
		\newblock publisher.
	\end{thebibliography}
\end{document}